\chapter*{\textsf{Lot of preparedness}}
\addcontentsline{toc}{chapter}{Lot of preparedness}

T\textsc{he} organization of war is like the organization of a Mills bomb inasmuch as it is fashioned to destroy itself in its act of life. The supreme moment of a soldier's life is the moment in which he dies.

The Battalion was in Romarin on the eve of Messines. Life, like a candle in a draught, was flickering fitfully before being extinguished. The mechanism of hell had been wrought to a state of smooth, yet violently running, perfection. And men had been adapted to be catapulted into violence. When the Battalion had departed for St. Omer the gun concentration had been impressive. On return the vast had become still more vast, so that soldiers beheld crouching might with worshipful awe. Someone unaware of the impotence of words hazarded a phrase or two.

``Lot of guns, Guy.''

``Lot of guns. Yes. Lot of guns.''

Lot of guns. By Almighty that phrase failed no more than any other. A verse of ``Mademoiselle,'' unusually clever and obscene and sung gustily, might have suggested the concentration. Might have, for that song could act as a vehicle for every soldierly emotion. But the silent awe of worshipper and victim was the sincerest tribute. As a mountain peak might in its grandeur catch the breath, so did preparation in its cumulative horror. ``Lot of guns.'' What a puny phrase from one whose crimson blood might yet almost splash their very mouths.

``Lots of guns. The Pacific is a lot of water.''

Humans had lined the rearward areas with roads and railways, pygmies had massed mechanical Gargantua. There were armour-piercing guns with long barrels and low trajectories, aristocratic thoroughbreds of destruction that could accurately drop steel pills twenty miles away. There were howitzers so short and broad that they squatted like frogs on the earth, guns that coughed shells up into the blue, thrown away from earth to fall beyond the clouds into Messines, from which city these thunderbolts sent up clouds of dust like the wind-swept spray of an iron-bound coast. He stood behind a squat howitzer and watched the shell going to heaven, listened to the puffing and wobbling sound as if it were a departing train.

``They fall down from heaven.''

`` `And this inverted bowl we call the sky,' '' said Joe, `` `beneath which, crawling, cooped, we live and die. Lift not your hand to it for help for-----' ''

``Jesus, Joe. Forget Omar for a little while. Lot of guns, Joe.''

That phrase hung in his mind.

``And a lot of shells too.''

``Yes. Lot of guns and lot of shells and lot of dead men.''

Hundreds of thousands of shells, light, heavy, corpulent, and slender, with and without copper garlands, covered the earth where grass had grown, until it appeared that spring had sprouted blunt-nosed shell. Dumps of shells, fields of shells, dumps of roading, of trench material, advanced and unoccupied gun positions. Dumps of everything in sufficiency. Dumps duplicating and dumps re-duplicating sufficiency. Death took few chances.

``Lot of material.''

``Yes. Lot of material.''

Expenditure in material was to be lavish, prodigal, exceeded in the carnival of Mars only by expenditure of men. For despite all the dumps and accumulations, there was lacking in the seats of the mighty the one commodity essential to economical victory, genius and directive ability. Thinking was mechanical and the mechanism was an antedated war manual. The inheritors of ``Lot of Preparedness'' were beggar-my-neighbour generals, believers in attrition. How much blood was pulped on the bloody anvil mattered not so long as the German pile of corpses grew simultaneously or even in lesser proportion. Attrition was attractive when the General wasn't cannon fodder.

Lots of everything, even correspondents. For correspondents gathered around holocaust like vultures around a dying man, ready to pick propaganda from the flesh of the dead. Perhaps they had concentrated too. Maybe they had dumps of words ready, new synonyms for jaded readers, spare fountain-pens, dumps of note-books. War correspondents were the captains of verbiage, were the artists unique in their command of butchery's terminology. Readers had to be catered for by means of a fresh verbal tactic. Bungling had to be written up eloquently. Like the prostitute in St. Omer, the correspondent had an excuse. He had to win a living.

``Lot of men too.''

``Yes. Lot of men too.''

Hosts camped at every vantage point, awaiting the order to advance and fill assembly trenches. Rumour had it that cavalry was gathered in the rear ready to essay advance when the bending line broke, permitting allied troops to spew through into Belgium. Oh, optimism of brazen heads in brass hats!

A complete bird's eye model of Messines surrounded by a gangway was approached and viewed by all who were to advance upon Messines Ridge, although on the day Messines was to bear no relationship to the model. For in the advance infantrymen only attained a worm's-eye view.

Underneath the German line shafts had been driven, mines had been stored with explosive, miners had been withdrawn, so that Fritz, prior to falling into hell-fire and brimstone, might be elevated to catch a tantalizing view of heaven. Above in the sky a score of balloons floated, each balloon giving the appearance of a hive because of the swarm of protective planes that hovered or throbbed about. Anti-aircraft guns pockmarked the blue with angry black-and-white bursts. Far above, out of sight or sound, to the edge of the ether, other planes climbed to swoop and tear at curious eyes with nickelled talons.

Because of the lot of guns, civilians were getting ready to evacuate the back areas, for Fritz had a lot of guns too and would retaliate. Shopkeepers were selling off their goods while men remained who spent freely. And unlike the forced sale of peace-time, the goods were fetching enhanced prices. The very reason for evacuation had caused concentration of spendthrift customers. Each home had its store and each home waxed fatter in the last hours of trade. Prices went up with the velocity of shell from a howitzer, went up until just within reach of clamouring heroes who wanted to purchase before they were separated from wallets by high explosive. Men cursed and paid. There was no time for argument.

Civilians departed with ill-concealed and grudging reluctance. Brave Belges had to be combed away like lice from the carcases upon which the battened. They were fearful lest on return the tide of war might have advanced a little. They wanted war to stabilize at the point of maximum profit. The concentration added many a coin to the already overfilled stocking. Profiteering knows patriotism only to a certain metallic clink, a crispness of bank-notes. The heroes to be were the shopkeeper's opportunity. Brave Belges and grasping Frenchmen squeezed their client\`{e}le, made dying an expensive matter, later were to charge survivors the cost of graveyard plots. Doomed to be rat and cannon fodder, men felt in advance the sharp teeth of gluttonous Belge and of the descendants of the compatriots of Joan of Arc.

Profit came in a final flood. Nevertheless profiteers strode on their way muttering curses at the temporary check to rapacity, greed only excited by harvests already reaped. Their thirst grew as does an alcoholic. They scrambled for their victim's flesh at the eve of Calvary.

Guy sat in beer-dens where beer had been eked out tremendously by the water-pipe, where the tariff had been advanced until only an overseas soldier could relax. Poor British Tommy on his pay had to pucker his stomach with water. The tariff mounted as the supplies vanished, as zero hour drew near. When the soldier had exhausted the diluted beer and wine of \textit{estaminet} and farm shelves, dust-covered bottles were brought up from cellars and sold at handsome prices. Who wanted money in battle? Notes had no currency in hell. Men might as well be skinned by brave Belges before being spitted by Huns.

``Lots of drunkards.''

``Yes. Lot of drunkards. 'I often wonder what the vintners buy one half so precious as the goods they sell.' ''

``You'll die spouting Omar, Joe.''

``Perhaps.''

They sat near the corner of Romarin road with hundreds of their comrades, who lined the dusty grass between road and ditch. Beer poured down thirsty throats and a vocal clamour rang from the side of the ditch. And except for roadside revelry, Romarin had gone quiet. The stream of traffic no longer poured along the roads. With everything ready and all the springs oiled, the sector hushed except for men's voices, calm before crescendo. Terrific preparation awaited a sign. Traffic would soon be a jostling stream again, but a stream to the rear, a stream of ambulance and stretcher, of walking wounded. Traffic slowed to a trickle as beer flowed to a torrent. Guy sipped of a bottle that was handed from man to man, a bottle that once was beer, that once was wine, that next time was beer again. The law against giving soldiers bottles had fallen down in the stampede for profit.

``Here.''

``Thanks.''

None swallowed more than a fair portion. Each tilted the bottle, passed it on, wiped his lips. Men near unto death caressed the long cool neck of the bottle, as good a way of saying good-bye as any. Stories, course stories, rang across the road, told not in subdued whisper but in oratorical voice, greeted with loud approval. The soldier was as far from home psychologically as physically. Or maybe was pretending to be to seem a soldier.

``'Ere's luck.''

``Good luck.''

Joe was tilting the bottle and swallowing, too happy almost to even want to quote Omar. He had the broad grin of intoxication. His face was dark with whiskers.

``Great fellows. Good beer.''

``Listen to the \textit{estaminet} pump.''

For the rascally brave Belges were adding water.

``You want a shave,'' Guy continued.

``Get a few close ones soon.''

``Yes. I suppose so.''

Ribaldry held royal court in the roadside ditch. Secretive souls were infected and unburdened themselves of long dormant profanities. Wine and high fever and uncertainty wiped out the need for timidity.

``Great night.''

``After the Lord Mayor's Show comes the sanitary cart.''

``Exactly. But as Omar says, 'One moment in------' ''

`` `Tell me the old, old story,' '' someone started to sing, but Joe didn't mind.

The smutty soldier songs were sung and resung. Sentiment was walled away behind obscenity. Yet of a sudden sentiment swept the wall away. Perhaps it was that twilight gave a measure of concealment to the face of each. The high explosive soldier song was miraculously no more, and the road rang with harmonics of peace-time days. Was it intoxication or was it the gathering darkness which allowed soldiers to wear a human expression on their faces which safeguarded the emotions of each from prying eyes? For on the eve of inferno none wanted to be human, and yet suddenly were all caught up in sentimentality. Husky voices harmonized. ``Good Old Jeff.'' ``Poor Old Joe.'' The songs of childhood welled up throatily, and for a moment New Zealand was very near. They sang till voices tired, and then gradually dispersed.

``Lot of guns. Lot of beer. Lot of bloody fools. Let's go to our huts. We want a lot of sleep.''

`` `What but thee sleep.' ''

``Jesus, Joe, that's a miracle. You quoted from someone other than Omar.''