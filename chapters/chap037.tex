\chapter*{\textsf{Rest}}
\addcontentsline{toc}{chapter}{Rest}

D\textsc{ulieu} would have refreshed mind and body if the air had not been thick with the talk of other offensives, if Brass Hats had relented a little. Winning the war by rigid attention to saluting and other irritants left an objectionable flavour on the most patriotic palate. Rankers could not see that an army that refused to salute on Monday would end the week by having scruples about dying. But if there were irritants Dulieu had billets deep in straw that was only moderately lousy, a Y.M.C.A. with writing material, and a few books and magazines. Shopkeepers and householders abounded, eager, at contraband prices, to do illegal trade in civilian bread; women there were with coffee, and even love, for sale.

The billet of John Guy was near a deep, warm, willow-shaded lagoon and the platoon rejoiced in the luxury of a morning swim. And the naked swimmer could lie concealed in yellowing wheat or raid the rows of growing peas in a nearby field. A tree or two yielded luscious cherries if money was plentiful and stolen cherries to the impoverished. There was a regular delivery of all London papers and careful buying could equip a section with \textit{The Times}, \textit{Chronicle}, \textit{Mail}, and even the \textit{Pictorial}. A rich Sunday stream offered choice between Garvin and Bottomley.

And war was interesting via the linotype. High strategy was entertaining if it flowed from the pen, though devastating if belched from a cannon mouth. Guy has a craving for everything readable. He read each paper from cover to cover, read music criticism although he knew nothing about music, read the stock exchange list as though he had money to invest. Swimming in the lagoon, stealing peas and cherries, getting a regular morning paper, he might have recuperated had it not been for pipe-claying and eye-wash. Sometimes he could sneak into the wheat field with his \textit{Mirror of the Sea} and dream himself into another world until a breath of air brought the swelling and falling of the faraway gun orchestration.

New reinforcements brought the Battalion to full strength again. Promotion became a daily affair. Privates became corporals, corporals became sergeants, sergeants left for Officer's Training Corps. Promotion brought responsibility but responsibility brought pay, satisfied an odd mortal who saw promotion as a career.

``Here,'' said the Sergeant-Major, calling on Guy. ``Put this up.''

``What up?''

``Bit of decoration. Congratulations.''

``Came up with the rations.'' He took the ribbon.

But if he joked he was pleased in his soul. Yet his pleasure was accompanied with astonishment. He was more out of harmony with the military machine than any man in his Battalion, and yet of all he got a decoration for the offensive. The Captain cornered him and congratulated him; the Colonel did too. The Captain said something about promotion, but Guy side-stepped the discussion as he would have side-stepped a bayonet. Yet he availed himself of certain privileges the ribbon brought him. Recognition of martial virtue made officers more tolerant when he loafed. Poorer soldiers had to be more industrious. Authority sometimes endured the idiosyncrasies of those tough in battle.

For a brief moment he knew wealth. Although he had lost all his watches he had managed to retain two pairs of binoculars. In abject poverty, and hungry for eggs, wine, cherries, newspapers, he raffled a pair among new-comers. The funds made life less arduous.


