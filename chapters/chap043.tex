\chapter*{\textsf{Good sport}}
\addcontentsline{toc}{chapter}{Good sport}

W\textsc{hen} Guy awakened the Colonel and his entourage had gone away to comfortable quarters in the rear. Back to the cook-house where monotony of toast and boiled eggs could be varied. With a companion and a useless phone Guy was left alone for a day or two, days during which the abominable rain of the advance gave way to brilliant sunshine. Since the pill-box was no longer headquarters he was not troubled to repair the disconnected line but just loafed, happy to be forgotten.

The craters dried out in the summer's warmth. The company in La Bass\'{e}e Ville was relieved. Strong points were formed although a few communication trenches were cut. Fritz, expecting a fresh attack, put down a terrible bombardment twice a day, swelling his hate to barrage fury at Stand To and Stand Down. Guy kept well out of the way, nursing his disabled phone and only appearing when the ration fatigue arrived. Once more for a day or two there were dead men's rations to eat. But the Colonel was his undoing. He sent a message forward to the company in La Bass\'{e}e Ville, a message required to be delivered in the brilliant sunshine of morning. The Captain remembered Guy and sent for him.

``Take this message across. Don't know why the hell the Colonel can't wait until to-night. It's nothing of importance.''

The Colonel had forgotten that La Bass\'{e}e Ville was out of bounds in daylight. The sun was shining brightly on the drying field as Guy set out.

German gunners were about and alert that morning. In early morning the zeal of gunners could be phenomenal. If gunners have to fire a few rounds why not fire them during the best part of the day. Rounds fired at night were fired at fixed targets, artillerymen could not witness the results. Guns fired at targets by advanced observers in daylight yielded the thrill that comes from sport. Artillerists are human and find it fun to terrorize a visible target, to lounge and peer through binoculars an laugh at the incongruous flounderings of an enemy. Guy himself had leaned against trenches with a pair of binoculars to his eyes and had laughed to see isolated German wayfarers pursued overland by British shells. He had watched men in grey plunging to the right and to left, falling to earth, to avoid shell which appeared to pitch at their very feet. He had found the spectacle of a solitary man being pursued by shell very funny. Sniping a man with a light field gun was as entertaining as going after wild fowl, called for high skill. The antics of the poor wretch could make the gun-pit or trench roar with laughter. In the mood, German and Britisher were both sporting.

So on this brilliant morning when a pair of German eyes saw John Guy step off from a trench to run a message across to La Bass\'{e}e Ville, a German throat chuckled and the guns were trained. Unconscious of the honour he was about to be accorded, Guy sped rapidly. He thought he would be all right, never believing a whole battery could waste time and shell on his progress. That doesn't mean he lingered to enjoy the beauties of the morning.

Whizz Bang! A shell exploded thirty yards behind him as he ducked down into a shell-hole, an isolated range finding shell that he in no way connected with his own presence. He jumped and hurried again.

Whizz Bang!

He seemed to nearly run into a shell that pitched thirty paces in front of him. But still he did not credit the shelling with any special design on himself although he ran when he jumped.

Whizz Bang!

He nearly ran into the third which seemed to pitch at his feet. He got a whiff of its acrid smoke and the earth fell around him like rain. And then he knew he was the target, that German gunners were having a game, laughing at the acrobatics of his fear. Grown cunning as always when animated, he stayed where he was.

Whizz Bang!

A shell exploded where he would have been had he run forward. He ran twenty paces to the right and a shell headed him off. The battery was skilful. He ran left and a shell turned him back. He ran forward and again filled his lungs with smoke and felt the rain of pattering earth. The softness of the earth was his salvation. Whizz Bang! Bang! Bang! Bang!

A whole salvo seemed to box him in and confused him. There was nothing for it but to run and he ran on. Aloud he revenged himself on Fritz.

``Dirty bloody bastards.''

Whizz Bang!

A dozen balls of shrapnel whistled around as Fritz tried to bag him with a canister of pellets. He found a deep hole and lay in it while Fritz landed a few more shells. Then the shelling paused. Perhaps they would let him alone now. He jumped again. Fritz was ready and a salvo burst at well-spaced intervals in front. He flopped to earth.

``Think you're bloody well clever, Fritz.''

His mind exaggerated the danger. A moving man with shell-holes to fall into is not easy to snipe with whizz-bangs. But he drove himself along. To the right, to the left, to the right, to the left, forward, forward, forward. Fritz knew the way.

Whizz Bang! Bang! Bang!

The smoke of a dozen shells was wafted to his lungs. Fritz was laughing. Better to chase a man than shell a blind target. Sweating, swearing, peering anxiously to ascertain if the last blast was nearer than the second blast. Zigzagging, he reached at last the cobbled road but away to the left of the refinery.

Whizz Bang! Bang!

The cobbled road was more dangerous. Shell detonated near the surface and had greater speed for splinters. The moving man had a good chance but the battery had a chance too.

Whizz Bang!

There were deep places in the old ditch beside the road and he rested again until the shelling ceased. But the ditch had been blown about by British guns and here and there Guy had to run across the gaps. Fritz knew the range to an inch and chased him along the road. Bang! Bang! Bang! Fritz had light shell to burn.

Salvo after salvo. The better shelter made progress surer but slower. A spent splinter of something fell against his helmet, hard enough to cut his flesh but not enough to impress the thin steel. On he went, crawling, jumping. But two shells did come that seemed to have his measure. Two shells with a peculiar intimacy, shells that seemed hungry for flesh, that seemed for an awful instant to be tunnelling their way to the very centre of his cringing personality so as to disintegrate the world from its centre. The shells had the sound known only to the ears of dying men miraculously saved. What it was that was different in their yell he never could explain, but their scream contained a quantity that the scream of other shells lacked. His knees bent, he subsided to earth, shrunken and folded almost as if he were to be returned to the womb. Mud splattered, fumes filled his lungs. But the shells had found soft filth just beyond the road, not the cobbles, and the intimate burst had been upward and not outward. He was left trembling, half shell-shocked. He rested on his right hand and knee and peered for the next explosion. If Fritz had not changed his barrels slightly he must have bagged his man, but the next shell anticipated movement and fell twenty yards in front. Guy decided to rest for a few minutes.

And only then it dawned upon him that his left hand was on something soft and hideous, something that gave beneath his fingers. As the odour of explosion drifted away the sickening stench of that something crowded up toward his nostrils. Slowly he turned his eyes to look. His hand was on a dead German who had lain in the rain and sun until he had become a mass of sizzling rottenness. Upon the pit of an abdomen wet with vile life his left hand rested. The shock nauseated him. He drew his hand away and fled. he couldn't rest alongside of that. He was hammered to his knees in a cleaner shell-hole.

Whizz Bang!

The hole was deep, safe. He rubbed his hand in the moist earth, drawing it backward and forward, forcing earth through his fingers to free his flesh from the taint, but in his mind the hand remained polluted until it was scoured with hot soap and water days afterwards. He wiped and wiped, but pollution seemed to have entered his soul.

Whizz Bang!

From place to place he ran, always wiping his hand against the cleaner earth, forcing the earth below his nails until they throbbed at his vigour. And then he reached a long line of safe ditch and crawled steadily toward the refinery, troubled far less by shelling than by the horror of a squirming paunch. Once he exposed himself and no shell came so he decided that Fritz had tired of the game. Actually the German observer was exercising restraint because he knew Guy's feet would lead him to a whole nest of New Zealanders. If the company headquarters could be ascertained that was a point to shell regularly, day and night. Guy jumped from his trench by the roadside, ran through the refinery, up a street, and dived down into the occupied cellar.

``Hullo!'' the Captain said. ``You look upset.''

``I've been getting hell.''

``What bloody fool sent you in daylight.''

``The Colonel.''

``No fool like a big one. We'll get hell now. We've been lying low to keep our place safe.''

``Been sniping at me with whizz-bangs.''

``Heard them and knew someone was coming. We'll get heavies.''

John Guy drank deep at a mug of water and unthinkingly wiped his caked lips with his desecrated hand. Shells started to fall around the cellar, much too close for unconcern. Heavy shells. Whizz-bangs were good fun for runners, but heavy-calibre was necessary to penetrate strong points.

``You'd better stay till dark. If Fritz sees you go he'll think there's something up and give us beans.''

Guy didn't argue. The shells were too close for dispute. Rubble fell around the cellar entrance as shell fell on the road outside.

``Damn the Colonel,'' the Captain declaimed, discipline giving way to exasperation. ``We hide and he advertises where we hide.''

``Me too,'' Guy muttered as the Captain swore.

Crash!

The shells were heavy. Every crash brought an expostulation.

``Damn the Colonel.''

When Guy remembered about his hand and his lips he used his right hand so furiously to wipe out the fancied taint that he made his lips cracked and sore. He rubbed them until they bled, the chafing only forcing them more upon his attention. He spat and spat and spat until he was parched, and when his tongue sought the sore lips he shrank from a disgusting contact. He had long grown callous to the dead, but dabbling a hand on a man's innards when half shell-shocked had been a new experience. The stain was ineradicable because it had become mental. At intervals the Captain fired his exclamation on the air:

``Damn the Colonel.''

The shelling died away without doing the cellar harm, leaving the occupants only shaken and not so sure as to the future. The sun was climbing higher and higher and artillerymen were losing the freshness of morning. For the tumbled bricks in La Bass\'{e}e Ville looked much the same to observers whether shelled or unshelled. It was better fun if not so useful to shell an intimidated runner. With the silence came sleep. The cellar was hot and the floor was hard but the inmates sprawled in ungainly attitudes, unconscious of even the flies that gathered on their grimy faces.
