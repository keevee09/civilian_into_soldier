\chapter*{\textsf{Disintegration}}
\addcontentsline{toc}{chapter}{Disintegration}

W\textsc{ith} a platoon that was almost wholly new to France he went back to Hyde Park Corner and to Red Lodge, back later to Plugstreet Wood. When they had lived on Hill 3 before Messines they had inhabited bell-tents, but now they occupied sand-bagged bivvies with much less safety. For after Messines Fritz traversed the hill day and night with light and heavy shells. Fritz would start at the end and work along, dropping shells, lengthening or shortening his range, and then for half an hour or more he might leave the hill alone.

In the day-time the shelling was merely dangerous to life. Repeated through all the hours of the night, the shelling started to disintegrate exhausted nervous systems. Sleep was too spasmodic. Lying in a flimsy bivvy with a shell exploding along the hill, with concussion going away and then creeping back shot by shot, left no period of rest to restore nervous reservoirs. No refinement of torture could exceed the periodic crash that came as one dozed in the no-man's-land between wakefulness and deep sleep, as one through tiredness had lost the preparedness that afforded partial protection.

If there had been a thousand shells their motley fury, as at Messines, would have at last sung hellish lullaby, but the recurring scream and crash of that single succession hammered nerves to pieces as scientists tell us a bridge might be knocked to pieces with synchronized hammer impacts.

Down the road between Red Lodge and Hyde Park Corner a second tenacious gun made blood flow too. The traffic always hurried, even before Messines. Now it came to Red Lodge at a mad chariot gallop, running a race with the blind periodic death that struck at the road from the traversing gun. Limbers moved slowly, unwillingly, and more unwillingly, along the road from Romarin to Red Lodge, and then the drivers yelled and whipped and steel tyres and iron hooves clattered and ground as the limber and its driver ran a race with murder. Through the night came pounding of hoof and tyre, jangling of harness, cracking of whips, cursing of drivers as limbers sped from or ran hellishly into the oncoming burst. Shells caught an odd limber, chopped horse and mule to pieces, hammered waggons to wreckage that steadied frantic team and fearful man. Fragments of flesh were thrown aside, pieces of wood, harness, spilled shells, were dumped out of the way, and other wheels were galloped in terror over the warm stains to grind away the signs of periodic butchery.

Infantrymen, hardier than limber drivers because more resigned, hastened on foot to fatigue, intimidated by wayside heaps of horse and mule flesh, by separate heaps of human remains, by patches of flesh and khaki cloth that hung from the very branches of the trees. Mechanical murder waxed in geniality and hung a rubber truck tyre high on a branch as well, as willing to play quoits with the inanimate as the animate. No wonder the cemetery beyond Red Lodge oozed blood down the hill, oozed a red gash even when earth was dry of rain; no wonder the new generation of rats grew large as rabbits as food-getting grew easier.

And John Guy of a sudden had become a casualty. The nervous threads that had held body and soul together seemed in a night to control, where once they had been controlled. It was unaccountable. No shell had torn his body limb from limb, he had no outward scratch. And yet his body had suddenly thrown down his mind. With a vile stealth concussion had worked insidious disintegration. He was a casualty and was unaware that he was a casualty. He put his new condition down to a strengthening of cowardice. He became increasingly susceptible to sounds, to all sounds. He twitched at sounds that were dangerous and at sounds that were harmless. And above all he found himself listening at the silences. Listening at the silence. How foolish that sounds people who have never used their ears to fathom the void. And all because his nervous responses no longer distinguished with their previous speed what was dangerous and what was not. So that he had to listen for all, to wait and see, to endure dissipating suspense. His power of automatic decision impaired, he quailed before everything. The imagination that had initially granted him swift adaption now turned and helped complete his downfall. Every sound that came from everything that could kill seemed to be coming from something with intent to kill him. Leisurely thought said ``You fool,'' after instinctive alarm had made him flinch, but instinct continued to outdistance lumbering thought.

He started again to wonder if he was yellow, to envy men of the new reinforcements, despite their outward timidity. For theirs was the timidity of inexperience, something to be outgrown with adaption. And his cringing was deep-rooted and, truly he sensed it, ineradicable for the war's duration. New-comers twitched openly but his twitching was furtive, curiously enough accompanied by an outward placidity. For he dissembled. He spied upon himself but tried to avoid the prying eyes of others. He was afraid to parade his fear, more afraid of his fear than of the shell which caused the shrinking. He did not want to advertise his state. And, like all men who cherish a demon, furtiveness nourishes the fiend. Confession, open confession, is the extreme exorciser. Struggling to dissemble, he exaggerated his state. Why had this thing come upon him? Pursued by his weakening nerves, the struggle was uneven. Relentlessly iron concussion undermined frail self-control. Soon he knew that at some date he was bound to go under nervously unless he was reprieved from strain. His sang-froid became so much military camouflage, good enough to deceive but not strong enough to resist attack. He railed at himself as far-away bursts caused wilting. Who is less the coward that the fearful who makes brave face? But reason of that sort was beyond him. He had grown too ill to be that rational.

To be a man. To be a man. To be a man. How important that was. How much more important than winning the war. To steel himself not to quake, that was a campaign to fight, alas, a losing battle. To show a brave face until a shell got him. To die at least sane. All this was of greater importance than winning the war. Hanging the Kaiser. Making Germany pay. Dictating peace in Berlin. What were all these if a man became a furtive idiot? But the furtive beast was fed on a profuse diet of shell detonation, the human was friendless. He asked himself one question a million times. Was he grown yellow because his reactions had become as mechanized, as inhuman as the gun explosions, because high explosive made rare fun of his grey matter? If he had had a friend with whom to commune what self-pity and contempt he might have saved himself.

Crash! a shell would go a hundred yards away and his heart would leap.

Crash! again and he would mentally cower, showing placidity to the world.

The thunder of guns made men imbecile. His mental processes were being devastated. His mind was limping to conclusions too painfully to save him from humiliation.

How was he to know that the physical illness of years allied to the stresses and strains of the war had made him at last nervously ill. He thought sickness was funk. He was travelling the road all surviving infantrymen at last travelled, but he was going farther along the road because he was underpowered physically and overpowered emotionally. Even his ribbon brought trouble for he felt the label attached was attached to a sham. He did not understand that the caress of barrage and the surge of advance, anything beyond dread waiting would scourge him again to hardihood. Because his nerves quaked in monotonous hours he thought his knees would sag in violent ones. His habit of dramatizing his each experience, of living and reliving each one a hundred times with a vivid, hysterical consciousness possessing him each time, meant that experiences never ceased taking a toll of energy, sapping and draining until the pool ran dry. And the ribbon on his breast made him fear that his furtive cringings were witnessed by each new chum. Thinking back he remembered his early scorn for men grown nervy under concussion.

He was nervously an old-timer and he had only endured for eight months. He did not understand that length was determinable more by the incidents survived than by the passage of time. For in that eight months all had gone from the platoon except very few men, so that he was in truth an old-timer. He had long ago surrendered any hope of outlasting the war with his physical faculties unimpaired. Survival from wounds or death was only reprieve. A succession of stunts loomed up in the years ahead. But what troubled him now was that mental disintegration might precede physical death. That something worse than death might lay hold of him. He might become a living imbecile.

And after Messines he hated the High Command with unhealthy venom. He never recovered from sight of horses and men plunging in shell-holes as they were mown down. He knew that the General did not know the nature of his own hell. And when a sick soldier loses faith in the cause and in the Generals and in himself, he starts to fall to pieces swiftly. If he could have given meaning to life, meaning to death, there would have been a rallying-point for his faculties. But as the world crumbled around him so he started to crumble inside. His spiritual state took on the semblance of his material environment. When faith in himself vanished his condition became pitiful. Yet it was that his comrades were unaware that he was sick, so did he conceal his state.


