\chapter*{\textsf{The attack on the pay office}}
\addcontentsline{toc}{chapter}{The attack on the pay office}

S\textsc{itting} on his bed dressing himself, he ransacked his brains for a story that might seduce the pay-office, but his mind seemed more fertile on creating difficulties than in evolving fabrications. Each spear-point he marshalled seemed to conjure up a hundred enemies. He knew himself for a poor liar, one of that tribe that is disarmed in advance by a sense of guilt. Finally he decided to make his attack without any preparation, trusting to his audacity and power of improvisation to save the situation. He would go briskly up the Row and in through the door in a vigorous movement and then he would adapt his strategy to the circumstances. The more virgin his tactic the less aware he would be of its weakness. He would beard the Colonel and romance for his life. How could there be elaborate rehearsing for a stunt of that sort? He was wise in his decisions, for he was always nimble-witted under pressure.

``Mademoiselle from Armenti\`{e}res, Parley Voo.'' He simulated gaiety as though he could humbug himself.

Therein again he was wise, for if he could once have himself he could have squeezed gold out of the vaults of the Bank of England. His confidence could become armour-plated. But there was a tinny quality about the tone of his voice. He lacked the conviction that he had pretended.

He was soon off, off to storm the exchequer. As if hard granite could be conquered by undaunted resolution, he squared his shoulders and stepped valiantly along. He held his head high, he swung his arms from the shoulder. At this moment when he required the greatest amount of individuality he looked most like the barrack-square soldier. A martinet would have surveyed him with chesty approval. He was concerned about not having delayed his shave until the morning, but he expected his chin to pass muster. He deliberately kept his mind from his task lest the internal cynic devastate his every subterfuge in advance. He marched on the granite building with his body and not according to plan. ``A good soldier marches on his stomach.'' Believing that there was something in the axiom, he paused and bankrupted himself, exchanging his coppers for two apples. He stood in a doorway and devoured them, skin, core, pips, flesh. The fruit was good. Two big apples on top of the bedside meal dulled the apprehension of his mind. He wiped his lips with the back of his hand. And then again he forged impetuously along.

``What's wanted,'' he said half aloud as he viewed the open door, ``is cheek. Damned cheek.''

As at zero he was suddenly exulted by the idea that none could resist the surging tide of his personality. All he had to do was to persuade the Colonel he wouldn't be killed until his account was squared, that the office would hold no mortgage against a wooden cross. Exultantly he launched himself up the few stone steps and at the door, but as he was about to cross the threshold resolution evaporated. He turned hurriedly away, dominated by sickly failure. He could have wished that the granite building were only a spitting machine-gun so that he could risk his life in a physical struggle and succeed or fall dead. He walked around the block containing the building like a sentry on duty, round toward Red Lion Lane, back to the Row. Three times he passed the open door and each time grew more irresolute, each time the interior seemed grimmer. ``Christ,'' he said to himself, ``if I keep on going I'll fall down from exhaustion and my teeth will start chattering.''

Yes, he was having a route-march all to himself and dinner-time would come around, and the office would close, and he would have to return and tackle the job on a sinking stomach. As it was, thanks to the hospitality of a comrade and two apples, he was physically O.K. He would never be better, however long he delayed. Feeling anything but a hero, he hurried in the door at last as a respectable citizen might for the first time cross the threshold of a pawnshop. The Sergeant behind the grill assessed him at a glance.

``Leave?''

``Yes.''

``Take your pay-book upstairs and have it checked at the ledgers. Come back here for the money when you get the warrant.''

``But I've just arrived. I haven't got a cent and I've lost my pay-book.''

``It's a wonder you haven't got syphilis and given birth to a pair of twins. You seem to have done everything else.''

``Surely it's not that bad,'' Guy grinned sombrely.

``It's worse.''

``But I haven't got a feather.''

``It's a case for the Colonel.''

``When can I see the Colonel?''

``To-morrow. He's not down to-day.''

``But I've got no rations.''

``You've got plenty of nerve.''

Guy was pleased at the assurance. Personally he felt very cheap, as cheap as his wallet.

``Who is second in command?''

``The Major. He'll be down in twenty minutes. Take a seat and we'll see what can be done.''

Maybe it was part of the Sergeant's duty to make supplicants modest in advance. Guy took a seat and grew as timid as a boy might while waiting in a dentist's anteroom. Suppose they insisted on his return to France? He fretted.

The great man was an officer detailed for pay-department service on account of sickness and wounds received in battle zones, a cheerful soldier in spite of, or because of, his disabilities. He lived in London as sumptuously as his pay and job would allow him to live. Like most paymasters he was badly overdrawn. Probably he had enjoyed his breakfast and was in a good mood as he came beaming up the steps to smile a ``Good morning'' to the Pay-Sergeant, to the men awaiting an interview. Gaily he strode to his office. He was a human first, a soldier incidentally. Nevertheless his presence failed to uplift Guy. Indeed his robust breeziness suggested ability to withstand assault.

The Sergeant disappeared to give the Major an inkling of the business of the privates who awaited him.

``You're last,'' the Sergeant said to Guy as he came out.

``You're cheering,'' Guy replied.

Buy his turn came far too rapidly, it seemed.

``Go on,'' ordered the Sergeant.

``Come in,'' yelled the voice inside, as Guy knocked timidly on the door.

He entered, halted in front of a desk, saluted, stood on a thick carpet in a luxuriously furnished den. Behind the desk the officer sat writing. He did not lift his head but spoke with good humour.

``Sit down.''

Guy sat down and an orderly appeared with morning tea.

``Another cup,'' the Major commanded, continuing to write.

Guy gasped. What sort of a Major was this? Tea for two in the pay-office! He surveyed the walls. They were covered with war trophies, souvenirs so varied and so splendidly exhibited as to command attention. The Major looked up and caught the appreciation in Guy's face.

``Good show.''

``Yes, Sir. Best I have seen.''

The Major handed him a plate of scones, well buttered.

``Anything I haven't got?'' cocksureness in the voice.

``Yes, Sir, there is.''

``What is it?'' There was incredulity in the voice.

``A German Iron Cross, first class.''

``Who the hell has?''

Iron Crosses were very rare in those days.

``Well, as a matter of fact I have.''

``You have?'' How eager the Major's interest became.

``Yes. A beauty.''

``Got it on you?''

``Sure.''

``How did you get it?''

Guy handed it over. The Major touched it and spoke about it as an awed collector might touch and discuss a scared masterpiece. He put it in the centre of his palm, turned it over, ran his thumb-nail around the silvered edges. He looked at his collection, he looked at Guy, he sighed. He placed it in the centre of the table on a white piece of paper.

``Have another cup of tea?''

``Thanks. I will.''

The Sergeant came to announce another man anxious to see the Major.

``In half an hour,'' the Major answered. ``And he's hopelessly overdrawn already,'' referring to the business that was waiting.

Guy lounged in the chair and sipped at hot tea while the Major continued to gaze upon the Cross as though it were a magician's crystal. An atmosphere of friendship was in the office. Private and Major had a common meeting ground. Again the Major sighed as he pawed at the Cross. Then without warning he came to business.

``Lost your pay-book.'' The words were not hostile because they were addressed to the Cross.

``Yes, Sir.'' Guy used the Sir again.

``How?'' Again the Major talked at the black medal with its silvered rim.

``Ah------''

But the Major interrupted and answered his own query.

``Here last night. Home with a Flossie. Not enough to eat. Too much to drink. Wake up in the morning with a sore head, watch gone, money gone, tart gone, and if you haven't been along for early treatment, health bloody well gone too. You haven't been near the dispensary?''

Guy shook his head.

``Better hurry when you leave here. Too many Minnie casualties in France for you to succumb to Flossie in London. This morning you feel a bit of a fool without a cent, so now you want enough to do it all over again.''

Guy was staggered at the rapidity of the faulty diagnosis which might have suited the average, but certainly not himself, not yet. What could he say which would not sound foolish alongside the Major's ready-made excuse? He did mutter something, the truth, and the truth seemed so damned lame and silly that he joined in the ironical laughter.

``As a matter of fact all I did last night was to go for a quiet walk.''

The Major roared and Guy roared with him.

``In the Y.M.C.A., I suppose, writing home to Mother first.''

Fortunately the Major's eyes kept returning to the Iron Cross, first class.

``What made you think you could get around me? What do you think I am?'' The Major heartily asked.

``Your reputation for softness,'' Guy answered the first part of the question.

``Not bad. Not bad,'' the Major said after he had laughed again. ``My reputation for softness. That's a new one.''

``Give us this day our daily bread,'' Guy want on, ``for thine is the kingdom, the power and------''

``Ha! ha! ha! ha!'' The Major was happy.

Guy realized that the moment had arrived to push home his attack. He abandoned all excuses. He wanted money and didn't want to queer the pitch with futile argument.

``Look. You wouldn't see a fellow down on his uppers in London. We all make mistakes.''

The Sergeant came in with a new pay-book showing Guy's credit at the moment. It was actually a few shillings.

``Well, I'm damned! You want an advance on this?''

``Yes, Sir.''

``And you'll be pushing up daisies before you get your account squared. I'll be asked to please explain------''

``I'm hard to kill.''

``You're tough, all right. But we're hard to persuade. Do you know you've got guts.'' For the first time the Major saw the ribbon. ``With your front you should have earned the V.C. four times. But bailing up a machine-gun is bravery and holding up the pay-office is highway robbery.''

The Major was pleased with himself and his utterance was jovial.

Guy was nonplussed but not defeated. His spirits were high.

``Have another cup of tea.'' The Major's eyes were again on the Cross.

``No, thanks.''

``Eight days to go and you're penniless. The Y.M.C.A. are arranging visits to country homes. Why not go down into the country for a week? It would do you good. Young ladies to entertain you, but not as you were entertained last night, not so exciting, so exhausting, but good for the health. You'll be among good people with sons at the war. You've got a good medal, bound to be promoted, commission not impossible. Chance of V.C. instead of V.D. A man who gets in such a helpless mess first night can't be trusted with much money.''

The Major had to be more than a martinet. He had to be a confessor and adviser, a welfare officer.

``Look here, take my advice.''

Guy interrupted him deliberately. He had to make an effort before the Major solved all his problems for him.

``Can't live on air wherever I go. Besides, I might sell the Iron Cross.''

That intimation gave Guy the initiative once more. The dialogue could never have occurred between an ordinary major and an ordinary private. But no ordinary major would have entertained a private to morning tea, and no ordinary private would have swooped down on London with an empty purse.

``Suppose the Hun got it for bayoneting a Britisher,'' the Major guessed.

``I got the decoration for kicking its owner on the backside.''

``It's a splendid souvenir.''

``And you've a fine collection.''

``Nothing like this?''

``No, hard luck.''

They got up and walked around the room and the Major related the history of many of the souvenirs. He carried the Iron Cross in front of him on his flat palm.

``Leave the Cross with me for safe keeping. I'll give you a receipt for it. You can have it \emph{apr\`{e}s la guerre}.''

``No. I want money. I'll sell it to you.''

``And go on the burst again.''

``Look here, you're a sport.''

Guy told the Major the truth about his arrival. Could he be expected to turn down his leave? He was listened to and believed.

``You're tough all right. You should wear the Cross yourself for cheek.''

``I'll sell it for a tenner. If I had time to find a buyer I'd beat that.''

``It's worth it, but I live in London and I'm overdrawn.''

``Bit of a boy yourself,'' Guy couldn't help risking.

``Living's expensive.''

``Well, I'm in a bloody hole. If I could give you the Cross I would.''

Guy meant what he said and the Major knew he meant it.

``I like you. You advance me a fiver on account and give me three quid for the Cross. I'm bound to die worth a fiver. You'll be safe and I'll have a pound a day to spend.''

``Done,'' said the Major. ``But I'll have to overdraw two accounts---yours and mine.''

``What's a few quid against the Cross?''

``I'm not squealing. It's worth more.''

They shook hands on the bargain. Guy wrote the story of the Cross on a piece of paper.

``You won't come again? I've gone the limit.''

``Not for morning tea?'' Guy queried.

``Good,'' the Major said, and laughed again.

``No. I won't come again.''

``I wish you luck. I'm sorry I couldn't offer you more.''

``You're a good sport.''

``Good-bye.'' They shook hands---the private and the Major.

Guy took his authority into the Pay-Sergeant and sang:

``The infantry have a jolly good time, Parley Voo.''

``What did you do in civilian life:'' the Sergeant asked.

``What do you mean?''

``You weren't a bushranger, by any chance?''

Guy beamed and took the money and a new pay-book and walked out over the stone steps. He walked on air. He held his head high and swung his arms again. In his office the Major was finding a position of honour for the new souvenir.
