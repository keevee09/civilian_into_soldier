\chapter*{\textsf{On the road gaily without a crust of bread}}
\addcontentsline{toc}{chapter}{On the road gaily without a crust of bread}

``D\textsc{o} want to go on leave? Let me know in two hours.''

Guy walked to the bell-tent and looked at his pay-book, although he knew the answer in advance. Divisional rules compelled a leave credit of ten pounds, and his book showed a minus. His last pair of binoculars had been raffled and the proceeds had been squandered. Damn and damn and damn. His luck for sure. Everyone on reserve was jumping at the chance. The Division was assembling to take part in an offensive in the salient. But he had no funds, no prospects, not a feather. London leave was rest from discipline, from detonation. And he was poor and leave was for the wealthy. Equality of opportunity had no meaning unaccompanied by equality of funds. War would go on until he rotted, and he would never get to London.

``Off to London! Off to London!'' A mate came bounding joyously into the tent. ``Man, it's a great war.''

``Is it?''

``Why so gloomy? You've been in France longer than I have. You're due for leave too.''

``Where do we have to draw out cash?''

``In the Base pay-office in London.''

``Are you sure?'' Guy's voice was losing its gloominess.

``The Sergeant-Major says so.''

That news gave him an idea. If he faked his pay-book he could get to London. He would be penniless in London as a result, buy being penniless in London was no worse than being miserable in France. And one never knew.

``There's a tide in the affairs of men.''

``What the hell are you talking about?''

He shrugged his shoulders. Yes, once in London he might cajole a tenner from the pay-office. If he could only first secure the leave. Anyway, why should he be denied leave because of penury, merely because his allotted shilling a day was nearly all spent on newspapers? He had heard that the pay-office would not advance money because a dead man was poor security for debt, because a carcase yielded nothing to a distress warrant. But he'd be damned if he'd let them take him down for his legitimate leave, poverty or no poverty.

He found a quiet place an erased the pay entries from his book. Like many a great statesman he achieved a paper surplus. He would be no worse off in England than in France. He would go to London if he had to sleep on the Embankment. As he completed his alterations another joyful figure came bounding into sight.

``London leave. Flossies. Spring beds. Ham and eggs.''

``Me too,'' he said.

``I thought you were broke.''

``I'm Get Rich Quick Wallingford.''

``How did you manage it?''

``Figures cannot lie, but liars sometimes figure,'' he repeated the political platitude.

``But how?''

``I've got a Scotch Jew relative in the Battalion and I borrowed a few bob.''

Guy's mate grinned and went on his cheerful way. On the leave train Guy did meet a Scot and soon they were chums. The Scot had grown shrewd in lessening the weight of the yoke.

``Here.'' The Scot rummaged in his kit. ``Out these sergeant's stripes up. You can take them off on the boat.''

``Why put 'em up?''

``The sergeant's have a jolly good time, parley voo. Sergeant's mess in Boulogne. Three or four in a tent instead of sixteen, plenty to eat without scrambling at the common trough.''

Guy put up the stripes. The Scot was convincing.

``Nobody knows who's who on leave.''

``You're right, Scotty.''

``You'd no like to come to Scotland wi' me?''

``No, Scotty.''

``Bonnie girls in Scotland. Keen, too. Everyone goes to London. Shop in the best market.''

``They've got to be very cheap to suit me,'' Guy joked as he thought of his finances.

``I'm thinking it's Palestine you should be going to.''

Crossing the Channel Guy took down the stripes in the boat latrine. After he boarded the train at Dover he tore his pay-book into minute fragments and threw the pieces out of the window. He took the French equivalent of his few shillings out of his pocket and grinned at its meagreness. Leave and hardly a feather to bloody well fly with. He was sure the train wheels were yelling a cynical refrain at him. ``No money to spend! No money to spend!'' But what did it matter? He was for London.

``Have a good time,'' Scotty advised.

``I'm a modern Dick Whittington.''

``You don't look like a Lord Mayor.''

``More like what comes after him, Scotty. I'm going to London broke.''

``But you fellows get good pay.''

``I've been listening to the wheels sing a song.''

``Turn again, Whittington?''

``No. No money to spend.''

``You'll need to be Dick Turpin, not Dick Whittington.''

``Dick Whittington, Dick Turpin---I wish I were Joan of Arc, I'd go on the streets.''

``Make a profit out of the trip. You seem cheerful about it.''

``Laughter in a cemetery.''

``I wish you were coming north, I like you.''

``Fifty-fifty, Scotty.''

At Waterloo an orderly from the New Zealand Soldiers' Club at Russell Square greeted the New Zealanders and conducted them to a waiting conveyance which ran them across the river an up Southampton Row and dropped them at the door of the Club. From the vestibule Guy saw that the evening meal was in progress. He fingered his few bob and hesitated.

``Here goes!'' An arm at his back thrust him aside and forced a way to the dining-room.

He followed the speaker in. The tables were covered with clean linen. Lady volunteer workers bustled industriously about with orders, or away with grubby plates. He stood for a moment looking for the quietest corner and ended by sitting abruptly in the nearest vacant chair. He toyed with the clean cutlery, read the trademark on the knives and spoons and forks, managed to give a very frugal order to a waitress. He listened to happy conversation, noted the relaxed comfort of men lounging in decent chairs who had a few hours before been squatted in a vile billet.

``You're just over?'' his neighbour queried.

``Arrived a few minutes ago.''

``I'm going back to-night. Have a good time.''

He commenced by ordering little because he could pay for little. Then he found the flavours so inviting that his hungered palate caused him to outrage discretion. He ordered extras and ate with a will. He remembered his old dead friend who used to bombard him with Omar and thought, ``Ah, take the cash in hand and waive the rest.'' He would eat, drink and be merry, and to-morrow he would lie. He grinned at the happiness of his thoughts. As he became increasingly self-satisfied he found more time to devote to his surroundings, to other men. At first he had eyes only for his plate.

``Do you know London?'' his table-mate was asking.

``The Embankment,'' he answered.

``Take my tip and find a good girl to-night and stay with her all the time. Better than one-nighters.''

``If I can find a girl to keep me I'll marry her.''

His companion ignored his joking, thrust back his seat, stood erect, called ``Horray!'' and was gone.

The New Zealanders in the Club dining-room were mostly pay clerks and Base Headquarters men, and men of the latest reinforcement from New Zealand on their London leave. Many of the regular customers were clever malingerers who found Base Headquarters pleasanter than France. Leave men were different and had a habit of taking a meal or two and a bath at the Club and of then disappearing until, money expired, they returned to the Club again prior to departure. Guy found joy in following the movements of volunteer women workers who hurried around carrying the plates of hot food. They were mostly New Zealanders marooned in London by the war and by desire to engage in war work, mostly young, tastefully clad, the well-nourished daughters of wealthy parents, amateur waitresses. To a soldier on active service all young woman are beautiful. To Guy from France, these English-speaking kinswomen were a miraculous improvement on the slatterns and drudges in the homes of the peasantry. Being a waitress for fun and being a waitress for necessity are responsible for different mental attitudes to the job. Volunteer waitresses were buoyant, bright, played at the job with a zest they might have attached to a game of tennis. If life had been compounded of the certainty of long years of plate carrying for a meagre wage, they would not have made the room and the soldiers so jolly.

He feasted his eyes on these women as the bustled about. Their presence, their attractiveness, their Anglo-Saxon characteristics, their pure English, catered for a something in the soldier that had been stifled, a something for which the sight and sound of a French slattern was a poor substitute. There was a touch of home, of New Zealand, about the Club in the heart of Empire, and reactions of subtle nature were evoked. It was as if he had caught a glimpse of a Golden Kowhai, or a tussocked hill, or a fern-clad gully in his native land. If Frenchwomen had been attractive and had been bustling about he would have been wondering about the easiest means of sexual approach and whether he could afford the price. But looking at these women he felt a different sort of affection.

Probably the contrast of sitting with his knees under linen and his hands toying with bright cutlery, and his palate tasting civilian flavours, with the squatting in a pig-sty and eating fragments from a dirty dixie with the aid of the point of his jack-knife, caused him to exaggerate the qualities of the women who hurried about. He envied them the excitement they had derived from such a happening. Bombs were so rare in London that they acted as a stimulant to patriotic effort, they made the lead-swinger and malingerer in London really feel he was at war. How delightful it would be if a front-line soldier could get a nice deep patriotic wound on a London street and be hurried along smooth roads to waiting doctors and nurses. It would be glorious to become a casualty on the London streets so long as a piece of humming steel did the job and not a diseased Flossie.

And then he started to appreciate the women with an interest more human and less patriotic. There was a tall, slender creature with fair hair who bustled about vivaciously. What a lovely girl she was. Did she go straight home at night, or did she loiter to cheer some lucky wretch on leave by a more charming intimacy than the mere carrying of a plate of food? But if she did loiter it would not be with a baggy-trousered and tuniced vagabond like himself, but with one of those fortunate mortals who tonsured and tailored and shod themselves out of private income. Was she a patriot who had refused to surrender her virginity in the firing-line? Had she been prepared to embrace an uncouth slave of Mars as well as to carry a plate of fried eggs to him?

Well, if in this day of chaos, this period of the impossibility of ordered emotions did she know fugitive passion, she was not for him. She was for some tailored, cultured son, some soldier possessed of war or pre-war rank and station. He was as much the vagabond in military as he had been in civil life. He couldn't guarantee that he could raise the price of a London Flossie until he had been to the pay-office. And it would cost more to entertain the charitable that to buy love outright. He was profoundly grateful that this beautiful woman should even fetch his plate and carry away his empty cup for a refill. She was willing to notice his order if she ignored completely its source. Women were still marvellous creatures, soft, unhelmeted. He was to keep making such primitive discoveries all the time he was in London.

His mind toyed airily with current dialogue.

``What did you do in the Great War, Mother?''

``Carried plates, my child.''

Now if she had only slept with some grubby, lousy hero.

He walked to the cash-desk and stepped away with few coins left. Commissariat, he thought, making fun of his predicament, brilliant, exchequer rotten. After the Lord Mayor's show comes the sanitary cart. But he refused to concentrate too much on the state of his purse for the time being. Downstairs he went to the basement bathroom and stripped off his lice-infested garments. He stood beneath the shower and let the water fall in its cleansing sheets. Hot water was free, soap was free, so he laved and scrubbed and watched his skin grow pink under the cleansing friction. He scrubbed with an animal zest and emotional abandon, as though by scouring his body he was washing the trench away from his soul. He found pagan rapture in the contemplation of a clean, wet, glistening skin. Where usually he had only to lower his head and sniff at his chest to fill his lungs with sweat and trench sourness, now his nostrils dilated to the odour of perfumed suds. The bath was a costless picnic which he vowed to repeat daily. The sense of cleanliness as he beheld his slender nakedness was itself worth a trip to London. Other men came to the shower-room and raced him, hurried along by other attractions. He lingered, keeping his financial worries away with a cake of perfumed soap and sheets of falling water.

``Don't use  all the water in London,'' the attendant advised, but with a good nature.

``The infantry have a jolly good time, Parley Voo,'' he sang.

He changed into a clean singlet, shirt and underdrawers, after drying himself on a warm towel. There were no lice in the seams of the clothes so he was free of livestock until a supply came through from his khaki. Fortune smiled on his penury.

``Here, take these. I am going back and tailor-mades are out of bounds in France. Chap gave them to me the night I arrived.''

``They're not crummy?''

``No.''

``Thanks.''

He put on the tunic an trousers. They were ill-fitting, made for a stout man, and Guy was a spare man, but there were no vermin. He would rather be baggy than verminous.

``Prevent you standing against all the lamp-posts to scratch your back.''

``You're right.''

But there was a hitch to the gift.

``You happen to have a few bob?''

``Too few. I haven't cashed in yet.''

``You couldn't------''

Guy surrendered a franc.

``My limit.''

``It'll do.''

``I'd like to go some more for the uniform but------''

``Don't apologize.''

Guy stored his helmet and gas-mask and webbing and rifle and discarded uniform in the basement store-room, received a receipt which he pinned inside of his pocket. He cleaned his boots. He sauntered into the barber's shop and had a shave, a shave from a blade so keen that he found sensual pleasure in stroking a silky skin. The barber advised all who were customers.

``Medical store on the left.''

``What for?''

``Early treatment. Dreadnought supplies. Open all night and day.''

As much from curiosity as from any other purpose he went to the left. Prophylaxes were of no use to him. He had the supreme disinfectant for wayward passion, an empty purse. Vice costs money. Still, if he lacked the price of sin he was entitled to his emergency medical rations. He wanted to see what was in the small cardboard box. He had normal curiosity.

``Don't be too shy to use 'em,'' the orderly advised.

``Too poor.''

``Some men are afraid the girl will be offended and they end behind the barbed wire. If you are too bashful during action come at the double and we'll treat you before you become too much of a casualty. If you get the record of treatment in your pay-book you won't get your pay stopped.''

A private at Guy's back had something to say as they walked upstairs out of the cellar.

``Take my tip and leave early treatment alone. They play a trick on a man and put him out of action for the rest of his leave.''

Guy hurried on without answering, unwilling to discuss a question like that with anyone. At the office he booked a bed in a dormitory for, despite prophylaxis, he knew he was doomed to sleep in his own bed. He had very little money and not a great deal of fortitude when his bed had been made safe for the night, but he did not permit his outward appearance to reflect his trepidations.

He turned from the dormitory to launch himself upon the dark streets. He paused a moment on the top step of the Club entrance and the cold air cut against his smooth chin. Standing on the top step like a diver about to plunge into bitter water, he chuckled grimly. What an adventurer he was, keeping the war at bay for the duration of his leave on his wits. Still he was clean, well-fed, warm. He couldn't hear the chattering of machine-guns. No explosion made him wince.

``First thing's a woman.''

The speaker was one of three. The other two assented. Guy saw the three go down the steps to the pavement, and as they moved into the shadows three woman came to greet them. There was a pause for mutual inspection, a linking of arms, a moving away. A leave man alongside of Guy spoke.

``Over the top with the best of luck, Parley Voo. Suppose they have paid for their beds here to-night. Most of 'em book in here but stay somewhere else. They want a bed-warmer.''

``So long as the heat doesn't burn.''

``Something in that. Good night.''

``Good night.''

So Guy watched another man go down and get accosted and go off arm in arm. And he fingered his few shillings and felt strangely out of it all. He wanted to have a good time too.

He walked to the corner and stood under the dim red street-lamp. Perhaps he would be as readily accosted. For the moment there were no more women about so he crossed Russell Square and walked along Southampton Row. He stayed his progress at a fruiterer's window where in a dim red light he thought he saw apples labelled ``four-pence each.'' In a bold stroke he got rid of half his remaining capital, emerging with two apples. Eightpence wouldn't postpone his evil hour for long and he might as well die hard. When he came out of the shop and stood against the curb to munch, a pretty girl came out of the night and ran deliberately against him. She was young, fresh, plump, jolly, warm, inviting.

``Good night, digger.''

``Good night, sweetheart.''

``I'll share.''

Blithely he took the second apple and surrendered it to her. Despite the gloom he saw the clean white of her teeth as she bit into the apple and her eyes were glad. Maybe she was animated at her easy catch. Certainly she was not melancholy in pursuit of her profession nor was she a slattern. She was a fresh bloom well worthy of hire purchase. Having surveyed him she was satisfied. He passed muster for she did not entice everyone. She slipped a soft arm through his and pressed herself excitingly to him. Her attitude was so friendly and trusting that he enclosed her shoulders with the arm that poised the apple so that when he reached down to bite at it her hair and her warm breath were felt against his smooth-shaved cheek. The close warm softness of her was thrilling. Out of the slums and inured to the streets, to the harshness of life, to a soldier used to mud and lice and coarseness she seemed to possess all that was clean and noble. That touch of her hair and the fanning of her breath brought back the atmosphere of southern harvest nights, of moonlight, and hay scent, of women far away. The starved mind exaggerated the softness and daintiness.

``Come on. Come on home. No good loafing in the street. I've got a good room and a comfy bed. We can have a fire if you pay for it. Supper to-night, breakfast in the morning, a good time in between.'' She pressed herself very close to clinch the bargain.

He shook his head but she paid his refusal no attention.

``When did you arrive?''

``This evening.''

She was overjoyed. Here was a man new from France, bound to be healthy in body and pocket, and he seemed to be a decent sort. If she could secure his goodwill the risks of pavement solicitation were banished for a week.

``Come on, digger. Last man stayed for the whole of his leave.''

``But I can't.''

``Why not? You're quite safe with me. Don't you like me?''

``It's not that------''

``I don't care if you've never been with a girl before. I------''

He felt old beside her but was an infant in comparison.

``I'm not shy.''

``Well, why not come home?''

``I've got no money.'' He threw away his core, not much of a core.

``Heard that story before. Come while your lucky. Here, have a bit of mine.'' She held the apple against his mouth so that instead of saying ``no'' he had to bite. ``Come on. The place is clean. I don't drink, but I'll get you a drink if you want it. I haven't had much of a run lately. Not very many on leave.''

``Honest, I haven't a cent until I get paid to-morrow.''

``Don't be scared.''

``I haven't the price, I tell you. You can search me.''

``Oh!'' She believed him at last. ``Hard luck. I could do worse.''

A group of New Zealanders were coming up the Row so she relinquished his arm.

``Sorry, but it takes money to pay the rent. Here''---she thrust half the apple back at him. ``Good luck when you get paid.'' She was gone.

He stood and started to bite her apple. Jolly girl. She would be good fun. He watched her accost a soldier, talk for a few minutes, hurry off on his arm, piloting his feet to her room. He knew that if he was still standing it was because his virtue was the concomitant of an empty purse. Of all the street women he was to see in London, she was to remain as the freshest, the most promising of life. Mayhap it took a lighted room to shatter illusion. Perhaps she was the first woman to grasp at his soldier's arms, the first to smile invitation at him, the first to press delicious warm softness against a body that had been sour with sweat and alive with vermin until less than an hour before. And maybe we value most that which is denied to us. Writers call girls of the street drabs. Why? To a soldier on leave such a girl represented all womankind. He took the coins out of his pocket and stared at them in the dim light as though will might cause them to multiply. And in the dim red light a pair of copper tails shone at him. And made him vocal.

``Jesus. Rule, Britannia.'' He looked at the dim effigy on the chariot wheels. ``I'm wearing your uniform but I can't afford one of your daughters.''

Another girl came bustling along and accosted him, a harder, older, coarser type. Perhaps it was only the contrast that turned him hastily away.

``No, no, no.''

``If you don't want the night you don't need to stay.'' She quoted a price for a shorter accommodation.

``No, no, no.''

She was disposed to bargain. To reduce her charge in an effort to meet the market. She was after business, was a hard, keen go-getter.

``No, I tell you. No.''

``Well, don't get shirty about it. You'll find a Y.M.C.A. right down the road, Algey.''

She left him. With her went the desire to be unchaste even if he had the money. She was a warning and not a promise. There was about her the roughness of a drill-master. And he didn't want love to be performed according to a military manual. But he saw her find a man and joke good-naturedly to him as on they went. Probably exaggerating the first girl's tenderness, he had been foolish about the second's hardness.

Night thrust other women up at him and made him deeply regret his penury. He oscillated between regret and repulsion as the daughters of the street were attractive or obnoxious. All he started to apparently ignore as he tramped. There were hundreds of eager women and girls along the routes tramped by the soldiers, above all for the overseas soldier. Poor Tommy was not such a catch. Tommy did not wear a peaked felt hat which advertised more generous pay. John Guy pursued a correct course, kept on an even keel by the copper ballast in his pockets, desire in his heart and a little military issue of preventatives in his pocket as he wandered the gloomy streets.

In France he had heard men tell how they would spend London leave. He had heard them again on the leave train. Tommies and Jocks were mostly bound home to wives, sisters, parents, sweethearts, at first anyhow. Those who had no homes but the money, were going to have a good time if they were average men. If they were exceptional they would continue as saints as well as heroes. He had not felt any desire to spend his leave with a woman until that jolly girl had brushed her hair against his cheek, and now he had desire without its price. For the New Zealander couldn't go home for a week and know his own. Australians, Canadians, New Zealanders, later United States soldiers, could only pass time among strangers. The exceptional men would go tripping with Y.M.C.A. groups, would go to organized entertainment. The herd, the average man, would forget France in a girl's company. Being men out of hell even London Flossie might seem heavenly. They wanted something that was the farthest possible remove from the life they had lived. They wanted to lean against a soft cheek rather than against a Y.M.C.A. hymn book. Sauntering along, he wondered at the charm of women who had sold their flesh. He did not know that the odour of trench filth, the rottenness, the continual presence of other men rough of chin and coarse of voice and phrase, the constant jarring of concussion, lent the street-walker's charm a high premium. Nor did the average soldier, for the leave period was too fleeting to shatter the glamour, too hurried to permit men to see tawdryness as such. He decided they were not so very vicious, these women who ministered unto and encouraged the appetites of the soldier. And was the desire to spend leave with a woman any more vicious than the will to spit some poor swine with a bayonet? Being penniless, a drifter and dreamer, instead of a man of action, he grew philosophical and defeatist.

Sex was a banned passion for the overseas soldier until \emph{apr\`{e}s la guerre}, except it expressed itself in the perversion of professional accommodation. But a whore was at least a step on the road to normality. Excess profits were good business, murder was nobility, bungling was generalship, getting drunk or sleeping with a girl were sin. To poke a bayonet in a wretch's ribs was uplifting because the military chaplain sanctified the contact. To lie with a woman was sin because war had voided the religious ceremony, had made such dalliance merely pleasure and only accidentally procreative. Religion was accommodating for all except the soldier. Patriotism couldn't wipe out sex hunger. For men were loyal to sex long before they had ever been loyal to a nation state. And absence didn't make the heart grow fonder. After a time it calloused the heart and men forgot their women who waited. Because those women would always wait. War was interminable. Now that jolly girl was better company than------

``Good night, digger.''

``No, thanks.''

So John Guy, unable to purchase, blundered along shadowy streets. And the denied grapes grew sweeter and not sourer axioms notwithstanding. His drifting solitariness invited glances and invitations. What right had he to prowl around unless he was anxious to become attached? His presence was unfair to the girls who plied their calling as he stifled hope of business by scarcely an answering glance. Most of the time his interest was furtive to keep them off, sometimes he gave back a blunt answering stare and had to shake a hand away from his arm. He saw soldiers and women pairing, hurrying out of the darkness to romantic lust.

And in the dim light he saw girls other than prostitutes going by in two and threes, going home from work, going later to music-halls, going to meet civilian if they were lucky. For like the young men, war denied young women of life, compelling young women to usher in a new sex convention. What soldier on leave had time to develop a leisurely passion, to take the risk of wasted labour? Life was fleeting, time was short, the prostitute required no courtship. Young women hurried by, women willing to be good companions as in pre-war days, but soldiers let them go. Results were what were required, immediate results. No time for finesse. Friendship was not enough. Who the hell wanted a friend when he was dead? Young women wanted music-halls and dances, and young men wanted bed-mates. Gradually war sapped and mined at sex conventions, as women refused to allow only prostitutes to express the Eve in all femininity. To get out of mud and blood and lice and coarse blankets and sleep with a pliant girl, that was what the soldiers wanted. And all in eight days.

``Say, digger.''

``No. No. No.''

``But, New Zealander------''

``My dear, I'm broke.''

``The night's getting late. Come on.''

``But I'm broke.''

``Honest?''

``Haven't a bean.''

``Hard luck. But I can't take you home. I'm broke too. I've had trench fever. Good night.''

``Good night, sweetheart.''

Very late at night, footsore and weary and sleepy, he wandered back to his bed at the Club. He was full of self-pity and bereft of philosophy. The nervous casualty had got on top again. Tears ran down his cheeks, welling up from that disintegrated interior. Nearly all the beds were empty, booked but unoccupied. The soldiers were having a jolly good time, parley voo. And he, poor man, was forced into a career of chastity by the very poverty that forced many a girl into a career of vice. Life is so contrary. Like causes produced different results, science notwithstanding.
