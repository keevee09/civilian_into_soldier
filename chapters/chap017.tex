\chapter*{\textsf{Plugstreet}}
\addcontentsline{toc}{chapter}{Plugstreet}

T\textsc{he} Battalion started to walk on many a night from Romarin or Nieppe to the valley in front of Messines. Trenches had to be strengthened against a spring offensive. For a long time Messines to Guy was only a hill towering above a valley, for none could see the town through the dark. Nor did any realize that Messines, drenched as it is with the fertilizing blood of battalions which have marched through the pages of history, was soon to be drenched with much New Zealand blood, that the blood of a new country was to refresh the streams of other days. As they worked or loafed conversation was interlarded with the thud of pick and shovel.

``Over the hill is Messines.''

``Big town, Sarge?''

``I haven't seen it yet.''

John Guy first saw Messines from across the top of a parapet at the edge of Plugstreet Wood.

``No more fatigue for a while, boys. We're going to Plugstreet trenches to-morrow.''

All ears paid attention, for Plugstreet had seen desperate encounter while the New Zealanders were still on Gallipoli. Grown quiet for the moment, it was to be the scene of bloody conflict almost until the end of the war. The Canadians had left their imprimateur on all the trench country around Messines. Calgary Avenue, Medicine Hat Trail, Montreal Sap, Moose this and Quebec that, Tin Barn Dump, Stinking Farm,  from Plugstreet to beyond Hill 63 the country had Canadian titles. In the rear Hyde Park Corner and Red Lodge told of earlier English occupation, and now New Zealanders had arrived to create Otago Trench and Wanganui Avenue, to scatter a wealth of Maori names trying to the tongue of Scot or Cockney.

War was like that. Men came from ends of Empire to name trenches or roads so that Australians died in Wooloomooloo Sap and got shelled to bits at Wattle Farm, Canadians ate mouthfuls of earth in Calgary Avenue, New Zealanders bled in Wanganui Lane, and Tommy rested his dying head on the macadam at Hyde Park Corner. When John Guy first strode through the Wood it was still a wood of growing, if scarred, trees, unlike so many of the places designated wood on the map long after high explosive had withered and powdered branch and stem and root. Deservedly infamous, its trees concealed lines of breastworks and batteries of heavy calibre. The trees restricted passage of air, and the wood was always heavy with the odour of German gas.

They found the front line in good repair, testifying to the quality of the Canadians. A good deal of the line was harassed with Rum Jars and Minnies, but the Canadians had put the trenches together as diligently as Fritz had blown them down. Two hundred yards separated the opposing trench lines, and no-man's-land was well wired.

``Messines,'' said a Canadian, as he hurried away, ``is just a couple of miles in front of you. You'll see it on the first clear day.''

``Messines is somewhere over there,'' a dozen voices asserted during the next few days, pointing at the mist.

For mist hung low, retarding visibility and adding to expectancy, so that the denied village to the mind became a great city. Fritz dropped tons of Minnies in parts of the line with no great casualty, for men withdrew rapidly from the Minnie dump to the flanks until retaliation created silence once more. John Guy was promoted to the charge of a signal station, for he had specialist training. The task gave him a measure of liberty. It made the trench and not one bay his territory, for he could patrol the telephone line at his leisure. The line had to be in good fettle, for raids were expected.

Three days went by, days of heavy trench warfare and minor casualty. Minnies and Rum Jars shattered more nerves than limbs. For three days the country in front was obscure, even the trenches opposite being difficult enough to discern, days so grey that few sniper bullets came, though heads were held above parapet level long enough to offer fair targets. Discretion condemned such idle curiosity, but boredom challenged restraint. Little more than the tangled wire and shell-holes of no-man's-land could be seen.

``I don't believe there is a Messines.''

``What's that?'' his trench mate asked.

``There is no Messines,'' he returned to the signal bivvy.

And then from the trench a voice hailed him.

``Guy.''

``What's up?''

``There is no Messines.''

He emerged from the bivvy. It was as though the elements were affronted at his agnosticism, for the fog was rolling away before an uncovering breeze.

``There is no Messines. Have a look.''

For the fog had gone and sky was blue in a perfect day. A breeze had stripped away the pall as a hand tears a coverlet from a bed. He jumped on to the fire-step and feasted his eyes, a perfect target for a sniper. Three other comrades stood alongside of him so that a traversing German machine-gunner might have cut life out of the four of them had machine-gunners not been tired after a dreary night, or perhaps bereft of hatred that morning. The four stood erect and no bullet spat.

``Messines.''

The eyes of John Guy travelled not as a bird flies but as a pedestrian trudges. He travelled up a cobbled road that crossed  a green hill, for blades of grass still grew unchallenged, blades soon to be withered and devoured by the pestilential blast of cannon. Houses of two and three stories rose in the town on the edge of the hill, houses as yet only sporadically shelled.

``An enchanted city.''

For the uncanny stillness of the town equipped it with a spiritual vastness thereafter. A deserted town seen from afar for the first time is weird. He saw houses and no smoke rising from chimneys, he saw streets without traffic, footpaths without pedestrians. No farmer's cart wound up the hill with creaking axle. The city, and the village thereafter was always a city, and no city of millions ever was populated with the horror and fantasy of that village, had not yet been hardly dealy with. An odd wall had been flattened, a gaping hole or two showed in brick edifice, houses from afar showed contents of rooms where walls had fallen down, but on the whole the artillery had been generous. High explosive had swallowed a chimney here, has nibbled a wall there, had bitten off a cornice or a gable, had pock-marked roads and fields and houses.

But summer was coming and insatiable war was getting ready to devour everything, houses, blood, fields and blood, roads and blood, for with the inorganic there had always to be enough flesh to smear the front with a devil's paste. Men built cities, expending brain and flesh and blood. Men hammered cities about the ears of men, mixing brain and flesh and blood amid the tumbling masonry. Man loved and builded and hated and ground each brick to dust.

Perhaps some of all this grew upon them as they gazed upon the silent, enchanted city. But the city was not deserted. Life and death were at the prowl and on the ready. An army were deep in cellars concreting emplacements and basements against the summer crescendo of the guns. At night limbers rumbled over the cobbles as the town came to life. Like bats and owls men came to life. In the night. And then the Sergeant came upon the four of them, four spellbound fools, four men who had never seen the like before of a city where life was frozen.

``Get down, you bloody fools, get down! Do you think you are Aunt Sallies in a shooting-gallery?''

Reluctantly they got down.

``Fritz might weight your feather-heads with a bullet.''

``Messines.''

He stood on the fire-step late that afternoon and looked again towards Messines. Even at the risk of a bullet it was something that he wanted to do. It was an hour before Stand To and the sector was quiet, quiet as a room is quiet when the ticking of a mantelpiece clock shakes the air. And as he stood he wondered what it was that drove men to risk the sniper's bullet, and though a shudder went down his spine at the thought, still he stood. So intense were his thoughts that he looked at them rather than at Messines. The pressure of a finger on an accurately aimed rifle, he thought, and I will flop down dead. But the German lines were two hundred yards away and the forest at his back made him a difficult target.

He shook off his morbid fancies and let his eyes come down the road from Messines to the German lines, and as his eyes caught the parapet opposite his heart gave a terrific start. For there, two hundred yards away, head and shoulders above the parapet, was a German staring back at him. In all the miles of desolation which in truth concealed thousands, all he could see was one man. And that one man in his loneliness emphasized the sinister stealth with which the thousands skulked.

``Hullo,'' he said quietly as he stared back, as though his whisper could carry to those other ears. Thus stood two fools, forgetful of machine-gun and rifle bullets, the attention of each absorbed in the person of the other. The German was too far away to be more than a man, but Guy was sure he was a young man. He thrilled with friendliness. He wanted to meet, wanted to talk to, wanted to wave toward the German so near him. Two hundred yards were only a few steps. But he did not pause to think that the two hundred yards were across barbed wire and barrage, across hatred black as raging, pitiless sea, hatred in which a human goodwill was of no consequence, the hatred of the old fed by the virility of the young. He wanted to wave and didn't. He just stood and stared, so that the German was the first to bid him good cheer. For the Hun waved, a friendly, extravagant gesture, a sentiment that reached over the hatred and made Fritz seem a comrade.

``I've been watching you a couple of minutes. You're a funny fellow, Guy. I've been watching dreams chase one another across your face. But come down, you silly old swine.''

``Christ, Joe,'' Guy said to the voice at his rear, ``Fritz is just like one of us.''

``Who is? What the hell do you mean? Speak for yourself.''

``The Hun in front.''

``What bloody Hun?''

``The one I'm making eyes at.''

``Well, why the hell don't you shoot him?''

``I'm rearing him a pet.''

``Come down, you bloody fool, come down! D'ye think the bearers want to carry out dead goats?''

Guy came down and saw Fritz going simultaneously.

``Just had a staring match.''

``Did you?''

``A friendly looking bird like one of us.''

``Do we look friendly?''

Guy laughed. Joe's chin was covered with four day's of growth and four day's of grime. He seemed weary with the disenchantment of war. On his chest was a gas respirator, on his head a steel helmet, in his pouches bullets, at his side a trenching tool handle, in his hands a rifle he had been oiling. On the step alongside was a case of Mills bombs.

``A friendly looking bird like me.'' Joe felt the point of his bayonet.

``You have all the appurtenances of goodfellowship.''

``But I know what you mean, Guy, I have felt that way myself when war was quiet. As Omar says------''

Joe got ready to evacuate a line.

``No, you don't. No, you don't. Throw a slab of Omar at me and I'll bayonet you.''

``No appreciation of the beautiful. But quit looking over the parapet like that. One of these days a Hun will shoot you out of sheer friendliness.''

`` `Into this universe and why not knowing, nor whence like willy-nilly flowing.' '' Nothing pleased Guy more than to plaster Joe with his own poet, and then to go on his way, refusing to listen in return.

Across in the German lines a Minnie gun detonated and a huge mortar came floating over to intimidate, to wrench and tear at the already churned earth, to shatter the calm of evening.

``Your friend knocking at the door.''

Guy grinned.

The Lieutenant came running along and asked for retaliation upon the Minnie gun. Guy ran to the signal station and grabbed the phone from the man on duty to see how long retaliation would take. He bawled the order and the recipient was convinced as to its urgency. He heard the order called from phone to gunner by the artillery signaller a mile away, he heard the guns in the battery fire a salvo. He handed back the phone and heard the shell a second time as the salvo screamed over head, and then back from the German lines the crash of high explosive as shell fell about the Minnie gun.

``Thirty seconds,'' he muttered. ``If I had said SOS a barrage would have fallen down in twenty.''

In twenty seconds a curtain of flame and steel could be erected between himself and the German who had waved. Joe had followed him to the door of his bivvy.

``Quick work, Joe.''

``To protect us from our friends.''

`` `Ships that pass in the night.' '' Guy grew sentimental.

But Joe corrupted his sentiment.

``S--------- that pass in the night, you mean.''

Trust someone to corrupt any sentiment. Joe backed him into the bivvy and standing in the door, drew his Omar from his pocket and by the candlelight read verse after until Stand To at last reprieved the signallers.

John Guy may have invested Fritz with homely qualities, but next morning when he put his head above the parapet in the same place a machine-gun traversed and whining bullets sang around his ears. He got his head down unimpaired, but when afternoon came he did not bother to nod again to his German friend. How friendly they would be breast to breast, bayonet to bayonet! Sentiment was weak, wayward, fitful. Hatred was organized, disciplined, armed, powerful. And a very small pellet of steel could play hell with a very large heart.

`` `For in the potter's shop one dusk of day, I watched------' ''

``Save it for a week, Joe.''