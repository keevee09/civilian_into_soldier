\chapter*{\textsf{About it and about}}
\addcontentsline{toc}{chapter}{About it and about}

R\textsc{einforcements} were thrust out from the Base as training 
was completed. A stream of men from Empire and its outposts was 
pouring continuously down upon the camp to flow onward and fill the 
gaps in the line. Sitting in the train at Etaples, Guy read the verse 
that was read by the million.

\begin{verse}
``A wise old owl sat in an oak,

\ \ The more he heard the less he spoke.

\ \ The less he spoke the more he heard.

\ \ Say, why not imitate that wise old bird?''
\end{verse}

``Who wants to be a bloody owl?'' a satirical voice greeted his declamation.

Entraining, he wondered what the front line was like. Associating with men who 
had already been, he understood now that what he was going to was indescribable. 
Every attempt at description seemed to listening ears a confession of recapitulative 
paucity. The trench had seemed so simple in far-off New Zealand, a mere few hundred 
miles of ditch fronting a similar length of enemy ditch. But gradually the front 
had assumed a complexity at thought of which understanding fell down. Curiosity had 
to remain passive until informed by experience. For he had come to understand that 
trench life had a content beyond length, breadth and depth. Rapidly, as he listened 
to those who had endured, he came to understand that trench language could only be 
appreciated by bodies that had wallowed in trench mud, for the voices that tried to 
tell him were choked to silence by his wall of ignorance. Can anything be described 
to virgin ears when there are no comparative standards?

``You are quiet, John.''

``My mind is prowling, Robbie.''

Maybe prowling was not the word. His mind was always engaged in truancy. For in Etaples 
he had seen bewildering sights, and heard uncanny commands, and sometimes the soldier 
fashioned to kill and the philosopher fashioned to doubt fought battles for his soul. 
He had seen military hospitals. Below the hospitals he had seen a military cemetery, 
a field where crosses and coverlets of earth were legion, a glittering forest of 
freshly painted white beneath which youth rotted. He had chanced to cast his eye 
upon the white crosses as the sun had reflected from their white paint so that each 
cross had mirrored a separate futility, and his mind had reeled before the spectacle, 
and his heart had been caught in a great sadness. The sun had shone down on the 
snowy coverlets as he had tried to imagine the hell out of which boys had staggered 
to die.

More men than he had known in life lay freshly dead, buried in sterile sand, hungry 
for the million fertilizing bodies. From windows of hospitals men could gaze on freshly 
turned heaps of sand that would enfold them too, when merciful death at last blotted 
out the pain of heroic existence. He knew that the trench from Switzerland to the sea 
was a similar graveyard, except that there were no crosses, and more bodies, and that 
dropping shells denied each body fixity of tenure. Looking at those glittering crosses, 
he had realized that the front, the suffering out of which this wreckage had stumbled 
to die, was beyond description, something men could know, something men could feel, 
something men could talk about only to those who had known and felt.

``The reflection from those crosses on a sunny day would stop the cheers of the millions.''

``Sombre swine. Too serious to be a soldier.''

``I'm not Robbie. I'm a passenger.''

``It's when you're serious that you're funny. If you mean all you say you shouldn't 
be in uniform.''

In Etaples, as he had learned to split bags of straw, he had seen fatigues digging 
fresh graves. Always there were reserves of graves for the next stream. And he had seen 
stiff corpses trundled along under the Union Jack, so that arid sand might become rich 
and ``push up daisies.'' The Last Post would ring out, and the burial party would hurry 
away for more dead. Ambulances arrived daily with damned freightage, a freightage hesitated 
and reprieved a moment and then hurried on from hospital bed and mortal ache to cemetery and 
eternal military precision.

``But sometimes in the sun, Robbie, the white crosses seem to be boys' faces, smiling faces.''

``I've felt that too. The crosses always get me. One wouldn't. It's the endless rows.''

``And what do you do, Robbie?''

``I smile. When in doubt I smile.''

But before the reflected pain and futility John Guy was always humbled. Something always 
stabbed from the crosses at the hidden civilian spot to which feeling had fled. Maybe the 
tears dropped down somewhere inside when he laughed. For it was unmilitary to show grief 
except with Last Post and Reversed Arms and Dead Marches. An army could perish, but only 
civilians might shed tears. Soldier tears were only applauded when outlaws cried behind 
the granite of military jails. Tears were not only unmilitary, they were uncomradely. 
Sentiment had to be obscenely ridiculed. Obscenity was mental camouflage, masking tenderness. 
There was no time for tears for the youth that was ``pushing up daisies''.

``They can do everything to a man except put him in the family way.''

``Yes. Yes. Everything.''

From Etaples the hundred thousand went on to hell. To Etaples some returned to sleep in 
yellow sand. The living went to death, the dead to burial beneath the wooden crosses. But 
every casualty was not honourably dead or in hospital. On the hill behind the barbed wire 
was the army that had succumbed to venereal disease, poor devils who had left wives and 
sweethearts but not sex itself in the town of mobilization. Any soldier might end behind 
the wire, for disease was wayward. A jury of the honourably dead would have acquitted their 
barbed-wire comrades, for they would have made allowances for human frailty. Nevertheless 
the authority that provided prophylactics provided no pension for casualties broken in 
breaking the wrong commandment. Lust to know woman was more reprehensible than lust to kill.

What had he not seen in Etaples! A half-frozen, crucified soldier fainting against a waggon 
wheel, a servant of Mars who had enlisted to offer his life but who had omitted too often 
to offer obeisance. The collapsed offender had punched an officer in the chin, forgetting 
that he had sworn to be violent to men grey only. What had Guy not seen! A witty Cockney 
instructor telling of the supreme military virtue of succeeding a faulty bayonet stroke 
with a well placed kick in the stones, advice virtuous in war where men live or die as they 
kick or stab, advice so cruel as to make the listener wince back from the imagined pain of 
the kick. And then the joke that put even the squeamish at ease. ``There will be enough 
good Britishers left for Mrs. Fritz.'' The Cockney was a brave fellow who mingled instruction 
with obscenity, cocksureness with a comic pessimism. ``Old soldiers never die, they always 
die too young.'' Roars of laughter. ``Say, here's to a naval engagement, and put your foot 
into his belly.''

Guy's own voice had roared out its contribution to the laughter that reverberated down 
the hillside and on to the field where newly interred heroes were getting ready to 
``push up daisies.''

``The cemetery is full of men who didn't kick or shoot in time. War isn't a game.''

``Didn't kick or shoot in time.'' Good man, Sergeant, for when men come and go in the grey 
dawn, friendliness, humanity, is suicide. Sitting musing in the train, John Guy wondered if 
testing time would find him steady, human or beastly. His thoughts caused him to set his 
jaw. To be able to shoot a man as one would a rabbit, as one would a wild-fowl; to be able to 
put a bayonet in a human carcase with sang froid? And then, perhaps, shooting a man was 
cleaner than shooting a rabbit, for the man could shoot back.

``Funny fellow you are, Guy.''

``Why?''

``I watch your face. You get happy and then pugnacious and then mournful.''

``I can't camouflage my emotions, but sometimes I'd like to see behind that smile of yours, 
Robbie.''