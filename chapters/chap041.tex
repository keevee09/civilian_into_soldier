\chapter*{\textsf{The advance of the refinery}}
\addcontentsline{toc}{chapter}{The advance of the refinery}

R\textsc{easonably} safe, he witnessed the advance into La Bass\'{e}e Ville. Rain came with the barrage, so that khaki swiftly vanished behind mist, rain, smoke of powder. The rain made the whole country a slimy bog of craters filled to the lip, through which attacking men had to force a way. Clothing became plastered with vile mud, boots were filled with battlefield ooze, rifles were jammed to the muzzle, rifle bolts were encumbered with grit and refused to slide. Bandages were soaked and muddied, so that filthy bandages pressed dirt against terrible wounds. All that advancing men could do was to press on, naked breast to nickel pellets, until within bombing or bayoneting distance. The naked hands were more valuable than the dirty rifle. But New Zealanders on the flank had only to advance a short way. Advance was easier than elsewhere, though possession was contested more tenaciously by flank guns afterwards.

For from the River Lys to miles beyond Ypres men waded through that terrible bog, at its worst around the Ypres salient. Thus did High Command butcher armies in the Serbonian bogs of their own creation. Men fell in the craters and were drowned in mud, or men fell and managed to float until their wounds were filled with rottenness, so that they fell slowly into decay in hospital. For no member of the High Command knew what it was he asked the Other Ranks to do. Thus was the British army put to death in its own midden by incompetents enhaloed by propagandists.

Watching the disappearing backs, Guy wondered whether the mud and rain would let the attackers capture the elation he had possessed on the day of Messines. From the trench top as misty rain came and went he saw the forms of men going away, forms that flitted in and out of barrage smoke and drenching torrent. He saw men reach at last firm roads around the sugar refinery, running forward into tangled iron and brick, and he knew that bomb and bayonet were playing a part. For a moment that thought re-elevated him to his Messines savagery, but he could not sustain that mood in immobility. As a spectator his heart was more with his frail mud-soaked comrades than against the Hun. Of course his humanity was one-sided and took no cognizance of crouching men in grey grapped in La Bass\'{e}e Ville, separated from Warenton by a sea of mud. The drenching rain lifted for a moment, and he saw New Zealanders going on beyond the bricks to the limit of advance. Rain and shell smoke intervened again, and when the bricks were once more visible everyone had gone to earth.

That was all of that Ypres offensive he saw if he witnessed many of the results. Tired men plastered with mud came splashing and wading back, walking casualties, red blood oozing down muddied garments. He saw wounded men plod their way back against death, to get killed at the entrance to a firm communication trench. He saw huge shells sending filthy gushers of mud skyward. He saw a shell kill a man and a second shell inter him in vile mud except for an upraised protesting hand. He saw stretcher-bearers sliding their stretchers across craters like a sleigh, plastering the wounded occupant, getting him used in advance to an all-encumbering earth. He saw four German wounded come splashing back, and the four were smashed down by a German shell, two to death, two to casualty. The wounded men flopped about in the mud and yelled for aid, but who wanted to get flattened by a German shell while aiding German wounded? It wasn't common sense. It was trying enough to face peril for one's own.

The Germans had been making for the end of the sap, and that was precisely where the German salvoes were dropping. The two didn't live for long. A screaming shell cleaned the bloody swine out, stopped their flapping and crying for help in a foreign tongue, foreign tongue that spoke the universal language of pain.

``Serves the bastards right.''

In the heat of such moments soldiers are not philosophers. A week hence they might find it in their hearts to be sorry. Just now too many of one's own comrades were being disembowelled.

``Yes, serves the bastards right.''

Four more German prisoners came out of the rain and over the mud, each pair carrying a royal passport to life, a wounded New Zealander, the most precious freight ever borne, for no muddied bayonet could be cleaned against ribs that carried a friend to safety.

\emph{``Kamerad! Kamerad!''}

``I'd \emph{Kamerad} them if they weren't so useful!''

As they went down the sap the aggressive speaker reached for a German backside with his toe so violently that Fritz jumped, inflicting agony on the man he carried.

``Keep still, blast you!'' the wounded man yelled. But Fritz didn't understand the English language, while the impact of the toe registered instant response.

Later, when the German guns lulled their fury, German wounded were dragged to hospital across the mud. Kindness and brutality were sporadic, waxed and waned with the intensity of German drumfire. Men who did their best to shove a bayonet into Fritz's guts in the morning did their best to save human beings late in the afternoon. Showing more mercy than was shown by High Commands. For from end to end of that Ypres offensive thousands of British troops perished to commemorate the stupidity of generals who should have been hung, English gentlemen valued for their accents and their breeding.




