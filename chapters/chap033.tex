\chapter*{\textsf{Look, the cavalry!}}
\addcontentsline{toc}{chapter}{Look, the cavalry!}

``L\textsc{ook!} The cavalry!''

In a lull as he sat half asleep, with nightmare horror parading through his weary brain, a phrase came to his ears. It came again, restoring alertness.

``Look! The cavalry!''

He jumped to his feet in his place of refuge and looked. Horsemen were coming up a flank, were spreading out, were attempting to advance.

What bloody madness was this that Higher Commands had conjured up? Generals knew as little of war as war correspondents. This was attack according to ignorant plan. Or was it according to bloodthirsty indifference? A group of men on horseback were going over the shell-torn field toward the German line, against uncut wire, against the machine-gun, against artillery. What a shambles they created. For shells and bullets registered on rearing horses, and horses and men were chopped to pieces. Horses, riderless, galloped and plunged madly hither and thither, frantic with agony and terror. Horse after horse was battered down. As he gazed upon the plight of dumb brutes he lost faith in High Commands.

Those who had ordered cavalry were helpless incompetents, blundering murderers whose knowledge of war was only the knowledge of the inexhaustible reservoirs of flesh they were privileged to destroy. The humblest infantry private would never have been so asinine as to order horses against uncut wire and machine-guns. Looking at the cavalry convinced the infantryman for all time that the Hight Command knew only the arm-chair and not the trench. The infantryman knew most of all about strategy, but his genius was slaughtered by an imbecile Staff deep in the dug-out.

That sickening sight beyond diverted German wrath from the sap momentarily, brought Guy and his comrades together. Guns, busy killing horses, yielded infantrymen respite. Sick of the sight, they all got busy and fashioned a deep place to contain the four of them. And once he had company survival again seemed possible.

``Who's next?''

``Don't bloody well joke.''

The man with the leg split from ankle to thigh heard them and called. For hours his eyes had been searching the vacant heavens.

``I say. Is that bearers? Here I am.''

They sat in their hole and ignored the plea.

``Wouldn't be a bearer to-day for all the tea in China.''

``Poor cows must be tired.''

``I say, bearers, can't you hear me?'' came the voice.

``What time is it?''

``Late in the afternoon. Seems a year since we hopped off.''

But the afternoon was early. Shells came again, but not so frequently, as they dozed stupidly, four men and two pigeons. Tiredness possessed so that ra, ra, ra, almost became a lullaby. Weariness defied concussion, except that it was overhead or alongside. Sometimes they slept for a whole minute, slept on duty, slept on the battlefield. No R.S.M. commanded stark wakefulness. Brass Hats were in deep dug-outs. If shells were powerless to keep eyes open how could orders? Once Guy opened his eyes and thought the jaws of a comrade had been shot away, but his eyes fell shut again without satisfying themselves as to whether they lied or not. When he awakened again the man was merely reclining with an open mouth.

Two bearers came.

``Who will we take out?''

Since they couldn't take everyone they wanted a man who would outlive the journey.

``I say, bearers,'' the shell-hole called.

``No good. He's as good as dead.''

``Yes,'' said Guy.

Soon they were off with a load, hurrying from danger despite weariness. Shells could evoke muscular prodigies. The four men fell asleep again. Shells played merry tricks with the dead men around but left the four men and a basket of pigeons alive. They awakened to sip at water, to nibble at biscuits, four men splashed with blood.

Afternoon, late afternoon, and they lived. Drumfire ceased and gunners prepared for twilight. Advance had ended and even consolidation was being postponed for the shadow hours. There was a battlefield hush except big shells which searched out big guns. Once Guy fell asleep for ten minutes. When he awakened he was so drowsy that it took him a few moments to recall the happenings of the day. He looked at the sun in the heavens, at the hands of a neighbour's wrist-watch. Four o'clock, four o'clock. There was still time for atrocity. More than twelve hours had gone since they had hopped off from the lines, part of an organized unit. Now the organized unit was a few live men scattered about here and there in shell-holes. They had gained a few yards of earth, a few prisoners, a few guns. Organization had ceased to exist, had been devoured by high explosive. For a moment success or failure rested on individual stickitiveness, on the capacity of men to sleep fitfully in shell-holes rather than run away. Brutal discipline launched armies into offensive but after zero, for a time, human and inhuman qualities in men counted. A man couldn't run away. His own opinion of himself held him to the gun-muzzle. As bombardment spelled for a moment organization reasserted itself. There came a raising of heads, a taking of stock. From company headquarters in each direction a sergeant went taking stock of survivors.

``Hullo.'' Guy looked up and saw a sergeant standing above.

``Hullo. Where did you come from?''

``Two hundred yards left.''

``Thought you were all dead.''

``Lot of us over there. You've had the worst time. Been getting hell around this sap. Not many in platoon left.''

``Four and two pigeons.''

``Jesus Christ! What a bastard place.''

Two hundred yards further on the Sergeant found another platoon in good number. He paused on his return to gather in the four.

``No good of your remaining here. Too deadly anyhow.''

They went with him willingly enough.

``Any bearers?'' called the wounded man as he saw himself deserted.

``Soon now.''

``Shift him and he'll die in a few minutes.''

Guy followed the Sergeant carrying his pigeons. A salvo hastened their progress. All around the head of the sap were dead men. They got out of the area before their overland progress brought fresh shell and they passed to an area of reasonably firm country where the Captain was, where casualties had been few. Chance of life had come back and nervousness with it.

As late afternoon passed away a new drumfire commenced. Fritz, fearful of renewed concentration, started to shell with bloody deliberation, searching the whole area to prevent the massing of fresh attackers. Shelling grew to the violence of midday. The vacated sap head received more than its proportion.

``Top of sap, see!''

``Why the hell didn't we move away?''

``Not enough brains to come in out of the wet.''

``No time to be spectators.''

``Easy to be wise after the event.''

``When we were there shells seemed to be everywhere.''

The illusory safety of an old sap had been the reality of a measured target. Huge shells were going back to British batteries. A deluge was falling on the ruins of Messines. Drumfire was on the advanced line. The British artillery gradually swelled into a counter-bombardment to chop up counter-attackers.

``Deepen your trenches,'' the Captain charged his men.

But they were tired and few shovelfuls were lifted.

What a bombardment grew up as evening came. For the first time the observers realized the creeping hell of barrage the had followed in the morning. Then they had been too much part of the storm to observe it. They had ridden it as a bather rides a surf-board. They had been organized in harmony with its ra, ra, raing rhythm. The sky was raining high explosive; away over the sap German shell was rending dead bodies. Curtain of flame hung in the front horizon.
