\chapter*{\textsf{Angley}}
\addcontentsline{toc}{chapter}{Angley}

T\textsc{here} is always one beast in a military camp. For a moment in Sling a 
loathsome beast reigned. R.S.M. Angley was a red-haired, rat-eyed beast.
About him there was the ferocity of the lair. Angley's morning always suffered 
from the previous evening's alcohol. At night he sought intoxication in liquor. 
In the morning he found distraction in sadism. Dilating upon the glories of 
patriotic manhood, he at the same time strangled to death many a patriotic 
illusion that had inhabited a breast. Angley was feared. With his power over 
camp he loved to be feared and to be hated. The knowledge of the powerlessness 
of lesser mortals pleased him as he strutted on the square. No man is qualified 
to wield dictatorial power over his fellow-men, although many men have to wield 
that power and wield it creditably. Angley was the least qualified. R.S.M.s in base 
camps are mightier than God. Angley was omnipotent, omniscient, omnipresent.

In France rotters were tamed. In base camps they rose to promotion and honour. In 
trenches, where life was cheap, a loud-voiced swine had no place. But in Base Camps 
authority was impressed by verbal braggadocio, by well-filled crime sheets. Where 
men die, men count. Where men drill, the skunk can sometimes rise to great heights. 
The ferocity of a voice is superb advertisement. Martinets abhor friendliness. 
Familiarity with lesser mortals is a crime against discipline.

In New Zealand, rank notwithstanding, all soldiers were civilians. In Sling military 
gradations were emphasized. There was no common mess trough for N.C.O.s and men. 
Promotion commanded severance of friendship. Rank above the common herd had to 
possess detachment from the herd. In France, in the line at least, gradations 
within a company might be obliterated. N.C.O.s and officers might again become 
friends, welded against distinction of rank by common suffering, by the presence 
of the leveller death.

At the apex of the camp stood the R.S.M., mightier than the O.C. commanding, for 
he was the instrument to which power was delegated. Officers, N.C.O.s, men of 
reinforcements were a flowing quantity, no officer associating enough among men 
to know them and protect them from the brutal. The R.S.M. had fixity of tenure. 
Fine fellows were too good for the barrack square and were worth a dozen swaggerers 
in the line. Good fellows stayed a short time in camp, the Angleys managed to remain. 
Angley swung a cane and vented his alcoholic spleen whenever he found a target. And 
the target could not resist. Only fools rebel against the all-powerful, fools and 
vagabonds. The Sergeants and the N.C.O.s could not resist. When Angley borrowed, 
Sergeants generally loaned.

Angley had a friend of similar kidney, a Provost Sergeant. It was common knowledge 
that both would descend on the Klink in the midst of a drunken carouse to beat up 
prisoners. Some men swore to pierce Angley's ribs with a bayonet if ever he went to 
France. Strong men, reduced to tears, longed for opportunity to grapple with Angley 
where manhood was of more account than rank. No one ever did beat him up. When at last 
his atrocity penetrated to the ears of the O.C. and he was drafted, he became a 
mysterious casualty. For Angley was a coward if for a moment he walked over the 
faces of the helpless. None would have bayoneted him had he gone to France. Such 
hatred could not have persisted amid high explosive. But many there were who 
swore they would. John Guy was such a one.