\chapter*{\textsf{Fretful argument}}
\addcontentsline{toc}{chapter}{Fretful argument}

J\textsc{ohn} G\textsc{uy} stumbled from the ditch to his dark corner of the hut and squatted on the bare boards. Equipment had been labelled and packed away, and would be returned to survivors only after Messines. As he sat in his corner there came to the hut a small group of men who were gathering to worship and invoke the aid of Almighty God. As they gathered a private who had once been a captain in the Salvation Army questioned Guy.

``You don't mind, Jack?''

``No. Go ahead.''

He lit a candle.

Actually he lied, for he did mind. Inwardly he was furious, for he objected to the easy assumption of a few members of the group that God was a Britisher whose passions were gratified and whose name was glorified when German briskets were spitted. He listened fretfully to the prayer of men who, failing return to Romarin, desired at least elevation to a sinless, painless eternity.

``Oh, Lord, our Father . . .''

They prayed for deliverance of the flesh from the earthy lusts of common men, and Guy, half drunk, felt their prayer as an accusation against himself and the hilarious tipplers outside in the ditch.

``Grant that we may be pure of heart and flesh.''

The sincerity of the voice seemed indecent, more profane than the deliberate obscenity of his comrades outside, his comrades whose voices had reverted from plantation melodies to blistering refrains once again, so that husky obscenity and subdued prayer intermingled. A voice in the hut started to sanctify the bayonet, to see God as a helmeted British warrior. And Guy was glad that he was an unclean sinner half fuddled with beer and wine, who lusted for a woman and yet loathed butchery and the harness of butchery in which he had been caught. Why, the God of some of these men was a mud-and-blood-dripping Junker.

``Our Father, which art in Heaven . . .'' Guy's fuddled mind caught at the words, and his lips joined automatically in the words. The prayer seemed more refreshing than what had accompanied it.

How funny it all was. His mates outside were heading to perdition astride a bottle of intoxicants, or they had already plunged to hell on Mademoiselle's bosom. But these fellows in the hut were going to heaven on a sharp bayonet thrust, on the wings of a nickel pellet, upon the groaning of a dying Hun. The juxtaposition tickled his hilarious mind.

``Ha! ha! ha!''

``Hush!'' rebuked an offended voice.

``Sorry.''

``God's will be done.''

The more assured they were of eternity, the more they seemed to pray for the earthy carcase. They wanted to wear God like a coat of protective mail. No. Far rather would he go with the tipplers to hell. For he did not believe that God would direct pellets and bayonets from a place in the skies. A God that derived satisfaction from the competitive supplication of Hun or Britisher was a monster bereft of the idealism of many a distraught victim. Sometime when he was sober he'd sit down and think it all out. But if there was a God, surely it was a sad father whose heart was wrenched at the downfall of brotherhood.

``Help us to dethrone the Antichrist.''

The God of these prayers was an archaic medicine-man to whom the suffering of victims was joy, to whose ears the groaning of the damned was ineffable music. God the Father belched his love from a British howitzer. Again he forgot the other inmates and chuckled.

``Hush, Guy! Hush! There's a good fellow.''

In the name of civilization and God they were crucifying manhood. Other men came shuffling into the hut and sat in their places and the prayer meeting was speeded to a conclusion. A very drunk fellow whose views on religion were a byword staggered in as the meeting concluded.

``Hullo! Hell-dodging, eh?'' Exclamation that caused a strained silence.

When the meeting ceased desultory conversation of a religious character came to Guy's sobering ears as he wondered whether he had not made a bit of a nuisance of himself. His mood was contrite and apologetic as intoxication ebbed. But a military padre came to the hut, a man in the uniform of God and Mars. The Padre paused a moment in the doorway.

``Sorry, boys, I could not get here in time.''

He was a personage, a military personage, a Sam Brown captain of the souls, for his religious mission had been fortified by authority. For men stood to attention in military formation while padres stressed the value of godliness, imposing salvation, as it were, ``by numbers.'' A padre was one of the great, able to command clicking of heels and precise saluting, able to talk about the meek and lowly and down from a high military elevation, as like as not commencing his sermon, ``Officers and Men,'' an apostle of the meek who dined at Officers' Mess to avoid the hardships endured by Other Ranks. For it was the Padre's task to goose-step his flock to the gates of Heaven in unbroken formation. The man of God came using the language of the beast.

The plump form of the visitant was illuminated by the puny flicker of candlelight. He was well rounded, well groomed, well cleaned, slept between sheets somewhere in a good bed. In shelled locations he occupied the deepest dug-out with the Colonel. His carcase was lice-free, filth-free, as he expiated to the filthy and lousy and the nervous upon the joys of virtue. The road to Heaven for his flock was the soldier's death. To die for the right was to die kicking a Hun in the face, in the teeth, in the stones. The Padre was secretary of the Officers' Mess as a side-line, was considered a sport because he drank a deep tot of regimental rum occasionally, because he didn't mind a few hearty damns.

``Good evening, Sir.''

The parson, who was a fire-eater if ever there was one, was soon affirming that God was on the Allied side.

``Not much of a General, then.'' The drunk was in the mood for a verbal brawl.

``Hush!'' said half a dozen voices.

From the corner Guy talked. He couldn't resist the impulse. The words were emitted spontaneously.

``You think God is happy when we stick a bayonet in Fritz?''

``Oh---er---I wouldn't put it that way, my man. But we---must---er---resist wrong. Christ drove the money-changers out of the temple, young fellow---er---'' The parson poised a plump admonitory finger.

Murmurs of approval greeted his effort.

``Thou shalt not kill,'' said Guy.

``But---er---my man, others are killing---er---you can't allow evil to triumph.''

``But why call killing God?''

``You don't understand. An eye for an eye.''

``But how about vicarious atonement?''

``Precisely.'' The parson swelled in triumph. ``For self-confessed sinners.''

``Judge not.''

``You're---er---verging on blasphemy, and certainly you're unpatriotic. But you may be honestly mistaken. For you are---er---a young man, a very young man. Surely you want to face God after having done your duty.''

``With blood dripping from my bayonet?''

``The Padres have a jolly good time, Parley Voo,'' interrupted the drunk.

``This is too bad,'' said the ex-Salvationist. ``The Padre did not come here to be insulted.''

``That's right,'' someone else agreed.

``But parsons should not be glorifying war,'' Guy persisted.

``Remember the Padre's rank,'' a Corporal of the meeting group warned Guy.

``Ah, no. Ah, no.'' The Padre adopted his most patronizing tone. ``Before God we are all equal. I don't mind. Young man, if I were you I would---er---attend more to faith and less to argument.''

``If God is the warrior monster you people make him, I could run him through the guts with a bayonet.''

A horrified silence greeted that, silence broken by the drunk, who always rallied to the man in the minority.

`` 'Ear! 'ear!'' he said in a tone that made Guy's argument seem part of a drunken brawl.

``Oh, Guy,'' murmured a comrade Guy respected. ``Why did you say a thing like that to-night?''

``Because God isn't a Junker.''

``When the guns are thundering you'll soon be on your knees asking pardon.''

``Because I believe in God the Father and not in God the National Soldier.''

``He'll be on his knees anyhow. The fellows who sneer at God are yellow when it comes to a pinch.''

``Perhaps we are.''

``No doubt about it.''

``Well, I'm not on my knees now.''

``Everybody's yellow when there are shells about.'' The drunk had a lucid moment.

``Er---er---well, well,'' from the Padre. ``Let us cease disagreeing to-night.''

Joe, who had been very silent under his liquor, tried to quote Omar ``about it and about,'' but his voice lapsed for want of energy.

Guy lapsed into silence, half-ashamed of his fretful, unmannerly pugnacity. What did their ideas matter to him? They had not interfered with him in the ditch---why should he trespass upon their private affairs? For they were good fellows, if bitter verbal antagonists. As he lay back his mind conjured up a picture, and he smiled at the top of the hut, felt again a desire to chuckle. For he saw the plump Padre, a testament in his pocket, riding straddle-legged on a shell. The picture was so real that he assented aloud without thinking.

``Yes.''

``Yes what?''

``Aw, nothing.''

``You're not getting rattled.'' Everyone, even the Padre, laughed at him good-naturedly.

Guy lost track of the conversation as he thought. Zero would end a hell of a lot of arguments and a hell of a lot of the contenders. Pro and con would blow up. He knew he was as confused as he thought the Padre was. He was a godless agnostic who believed God was not as bad as He was painted by the Padre and other fire-eaters. But it was the Padre's smug assurance more than anything else that irritated. If there had been suggestion of doubt------ And then he was caught up and listening to the voice again.

``It's wonderful, men---er---it's miraculous, the way shrines avoid destruction by shells. To-day I passed two in shelled country and both were undamaged. Er---I think the angels must guard them.''

Guy heard himself speaking again in harsh, jarring, offensive tones that vibrated over the murmurs of the faithful. He was so outraged that he literally exploded into protest. Special dispensations for shrines, for bricks and mortar, angels deflecting shells from the inanimate but careless of the living, men pulped and bloody before protected images, flesh a godlier target than masonry, savagely, with crazy deliberation, all these reproaches tumbled out of his mouth, accusing the Padre.

``You bloody fool! Since I've been in France thousands of men who could build shrines have been killed. Men of flesh and blood, not shrines of brick and mortar. Where are your bloody angels?''

Tense silence succeeded his outburst. He had called a captain, fortunately a padre, a bloody fool. The far-away chattering of a machine-gun came to listening ears. A limber rumbled by toward somewhere. The Padre did not stir, refused to articulate, and everyone knew that it was his turn. The well-modulated procession of words that exuded from the well-clothed paunch were halted, seemed to have been thrown back in confusion.

``Er---er---er------''

And Guy interrupted again.

``Some of these men were my friends, but the angels were absent without leave.''

``Guy,'' the friendly voice again rebuked, ``let's cut out argument to-night. We'll be in it soon and we'll all be sorry.''

``He'll be yellow when the whips are cracking.'' The Corporal was positive.

``We will not---er---discuss the matter further to-night, men. Good-night. May God be with you in Messines.''

``The angels willing,'' Guy muttered.

The Padre shook hands all round, Guy shaking because he lacked the courage to outrage convention any further.

``I trust, old man,'' said the ex-Salvationist softly to him afterward, ``that if you get wounded you have time to ask forgiveness.''

``Well, who knows? I do a lot of things I wonder at. Even get drunk.''

``If you are facing death------'' The Salvationist reached a kindly hand for his shoulder.

``If I am numb with pain heaven knows what I will do,'' Guy interrupted him. ``But I've been at the edge twice now. All I knew was tiredness, anxiety to go to sleep.''

``You may have been saved for another chance.''

``Rats! But a little pain makes people queer in the head.''

They were talking quietly. ``But surely what I say in health is of greater importance than what I say when health is at an ebb.''

``You're a bit drunk. And are you in mental health?''

``I suppose we are all a bit feverish because we have been cut off from yesterday and don't know about tomorrow.''

``I know. Whatever happens to me.''

``I am glad you have that comfort.''

``Turn it up. Turn it up,'' said the very drunk one. ``Turn God up and go to sleep.''

Someone had put out the light so Guy and the Salvationist shook hands in the dark as kinsmen and comrades, though not as co-religionists. They both had faith in one another although there was disagreement about whether God was a Junker or the Father of all humans.

``I'll pray for you,'' the Salvationist whispered.

``Save your breath, except that you pray to the Prince of Peace, the man of compassion.''

The drunk rounded up the night with a verse of that great song inspired by the occasion.

\begin{verse}
``The padres have a jolly good time, Parley Voo;


The padres have a jolly good time, Parley Voo.


The padres have a jolly good time,


With Mademoiselle behind the line


Inky Pinky Parley Voo.''
\end{verse}

Chuckles came from the gloom and the strain was relieved. Laughter was the solvent of all bitterness.

``How quiet it is.''

How quiet it was. But the mind was active. As Guy lay on the boards the voice of the Corporal came back. ``Unbelievers are yellow.'' He would see. Perhaps he would be more afraid of the guns than of spiritual hell. He would see. If he was bricks and mortar he would have angels to ward off the shells. Being human, he was vulnerable. With that absurd thought in his mind he fell asleep smiling.