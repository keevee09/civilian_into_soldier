\chapter*{\textsf{Vicious appetites}}
\addcontentsline{toc}{chapter}{Vicious appetites}

B\textsc{ecause} St. Omer was the eve of giant offensive, because war made a mockery  and a bitter memory of real love, St. Omer had its problems. St. Omer catered for the lusts of cannon fodder. No civilian passion could get in the way of glory, but if men stood in the queue they might have a fleshy release that was almost martial. For lust expressed induced a moment wherein one attained forgetfulness, even if forgetfulness was succeeded by shame. Just for one instant before the crashing thunder men wanted to be nearly men. The disintegrating shell would obliterate the shame of all contacts. Men wanted to embrace an overworked prostitute as a prelude to embracing shell.

Excited by proximity to St. Omer, with its catering for vice, men talked crudely of the abodes of love. And Guy's curiosity was inflamed, and maybe his emotions were stirred a little too, for if not altogether a satyr he was yet of the earth and earthy. And if for no other purpose, he wanted to go to the mart where flesh was for auction and play the part of the spectator. After church parade on Sunday the Captain had a few words to say.

``You are preparing for an offensive.'' The slightly irritable and assertive voice dwelt on the offensive and its possible consequences. ``You are in a friendly neighbourhood.'' The voice stressed the need of restraint from temptation to pilfer livestock and vegetables. ``You are men, and there are many brothels in St. Omer. Ten men from each platoon can have leave every night. Number Nine is out of bounds. You can get preventatives from the Medical orderly, who will remain in his tent until midnight to treat the wise. Dismiss the parade, Sergeant-Major.''

John Guy always wondered thereafter whether cowardly shame or a lingering sense of decency kept him away from the Red Lights of St. Omer. Certainly he had grown mentally incontinent. He lusted in the spirit as much as any sinner. But that longing never was transported into doing. Vice was not sufficiently furtive. He wanted sex experience to be a private matter. He stayed away from the queue outside and associated with the rare mortals who were purged of earthy dross or who, like himself, wanted to sneak in quietly. Denying himself, he exaggerated the attraction, for disillusionment was rapid at the end of the brothel road. There was neither romance nor joy about contact with overworked French girls who were driven to prodigies by Madame and war.

None who had gone were very boastful about the women, many were apologetic. But all talked with delight of the pictures on the wall, gained perverse thrill in assessing the battalions that marched on to death over each woman's bosom. For Madame Proprietress was adamant and drove her slaves along, fattening her banking account on patriotic lust and the flesh of her sisters. Joyless, automatic, sexless love only appeared to be better than self-denial until it was caught breast to breast. Listening to his comrades, Guy remained, like most young men, the centre of a battle between disgust and attraction. They had attained obscene disenchantment and were the better soldiers. And what right had he to be afraid of the queue? Cannon fodder has no time to be punctilious. He wondered if only the brave could be vicious. For militarism, mother of prostitution, had given him lust and robbed him of courage. Parade-ground love was beyond him. Yet he wanted to dare before he was spreadeagled by high explosive. Of course he only thought of the Red Lights some of the time. St. Omer was in a sense a holiday, interlude, clear sky and green fields, hedges and tress, before churned earth and the rumble of guns. And drill that was rehearsal for real killing was free of militaristic dross. And coming adversity brought officer and ranker together.

High Commands out of their generosity ordained that men should be killed in clean shirts and with clean skins, so baths were visited. Equipment was examined, iron rations and medical dressings renewed and reinforced. Brass buttons were dulled so that no glittering reflection could attract machine-gun and sniper's bullet. Officers donned privates' tunics, fine feathers being only a boon when they brought privileges, a curse when they acted as magnets for bullets. Church parades occurred more regularly, men prayed sincerely for a steady arm and forgiveness at zero. True believers went and doubters too, who ``wanted a little bit on in case there was something in it.'' Unbelievers grinned and blasphemed. Many an earthly soul in a quandary went to church parade on Sunday morning, to bayonet practice on Sunday afternoon, and to Mademoiselle on Sunday evening, having a moment of passion lest zero obliterate not only the immediate but the eternal as well.

Bayonets were sharpened until they had razor edge, until slight pressure would split any German, made sharper than butchers' knives, for they had tough, wiry, unfattened carcases to slice. John Guy held his bayonet against the grindstone and visioned the use to which he would subject its glinting beauty. He thrilled to ecstasy as he realized how easy the sharp, clean and beautiful blade would pierce the guts of wilting man. He visioned agony in enemy face as naked hands grasped at his sharp blade and tried to force it away from a gasping chest, from a gaping throat, from guts, a blade so sharp that it would cut protesting hands to the bone.

The killing lust called by martinets ``the spirit of the bayonet'' came upon him as he held the blade to the grinding wheel. The blade made a pleasant thrusting sound when tested against sack of straw, cutting crisply, like a sharp hay-knife. He wondered if a bayonet made just that sound when driven against a hot, living body. Fury of hand-to-hand encounter possessed him and in prospect he saw himself swell to a demonic size as he visioned the onslaught of enemy. Yes, by Jesus, yes. By the Almighty, yes. In a tight corner he would thrust, he would kill, he would maim, he would listen for the music of the cut, the sound of cloth and flesh giving way, he would listen to wounded lungs whistling for air through the hole he had made in the chest, lungs strangled with blood, overwhelmed with air that came through unaccustomed apertures. His elation was succeeded by a nausea, a prayer that he might fall by an impersonal bullet and not by the slicing of a bayonet, a prayer that he might be able to kill the same way, to topple grey figures to damnation with a nickel pellet, let out life through a bloodless puncture in the forehead.

Battle was pending so gambling increased and stakes grew in size. Life was at hazard, so why not first of all hazard one's possessions? Who wanted to die possessed of a pocketful of paper money? Spinning coins helped to cure the fever within, and if a man lost his all he lost a few francs, and if fortune smiled he grew incredibly rich, able to have a royal boozing in St. Omer. Champagne appeared when the money was plentiful. Hens laid if one caressed their owners' palms with francs. Money, enough money, could lift a lusting soul from the public queue to the bedroom of some clandestine caterer for passion, a woman who, unlike Madam, did not believe in smaller profits and quicker returns.

Of course the price of love, like the price of other commodities, was regulated to prevent undue profiteering. The humble ranker needed a tariff not beyond very slender resources. But the man with a hatful of francs could find a less public vehicle for his lust, could avoid standing in a queue in which Algerians, Chinese, French, Portuguese, or Britishers, testified to the universality of depraved appetite. The secret whore risked arrest, so had to charge a little more as compensation. Many a soldier took a cap full of francs from the ring to Mademoiselle's lap, and maybe received a good deal more than that for which he paid.

But soon the training was completed and the swelling thunder of the guns was demanding the return of the Battalion. Packs were shouldered and feet were swinging to the strains of the regimental march. And the moment of serenity was gone again. Infected with the fever of high explosive, Guy regretted opportunity lost. He would be a long time dead.

The country was never more alluring than along the road of that farewell match. Waving wheat, white houses, civil life was departing. And converging on tumult, Battalion was charged with high-spirited carelessness, a mental lightness that caused even feet to scarcely notice cobbles. Footing it to chaos, Battalion sang as never before. They sang:

\begin{verse}
``The infantry had a jolly good time, Parley Voo,\\
With Mademoiselle in Number Nine, Parley Voo.''
\end{verse}