\chapter*{\textsf{Arrival}}
\addcontentsline{toc}{chapter}{Arrival}

A\textsc{t} a wayside station they heard on the air the faint, far-off 
rumble of guns, sound that rose and fell, rose and fell like the wind driven 
crash of thundering surf. There was no intimate bark of individual gun, no 
clear detonation of shell. Guns and bursting shells were too far off to be 
more than blended in the rising and falling, the rumble, rumble, rumble.

``The orchestra is playing.''

``Yes.''

``The hymn of hate.''

``Yes.''

``The thunder of a heavy sea on rocks.''

``Yes.''

``All they say is rumble, rumble.''

``Yes.''

He said no more than ``yes'' lest he say that more wrongfully. He would know 
this world before he talked of it.

There came another station, and rumble mumble was a louder rumble mumble, and 
glowings and flickerings like far-away lightning lit up dark skies. Maybe the 
new reinforcement wondered what war was like between the opposing cannon mouths. 
Then came rumour, and from an officer confirmation.

``It's the Kaiser's birthday, boys, and there's a big bombardment and 
counter-bombardment going on in his honour.''

German and British corpses were falling a little more rapidly. Even in war humble 
trench flesh had to yield bloody tribute to the purple.

Suddenly they were at a railway station which had known severe shell-fire.

``Hope sparks from the shack don't send a few shells this way.'' It was the voice of 
an old-timer.

Another station was halted at, and weary men were detraining, shuffling along cobbled 
roads, whispering to one another under cover of night; absurd whispering, for they were 
three miles from German ear.

``Light no matches. Smoke no cigarettes. This road is under observation.''

It was the strangeness that induced the whispering among the new-comers. An old hand 
laughed at the command, talked in a loud voice, and then, cowed by his glum associates, 
said no more. Guy saw shell-pits alongside the cobbles, in the distance saw a beautiful 
circling light, saw the sky flicker as guns fired.

``Dear Mother,'' Robbie said, writing an imaginary letter, ``there is a heavy bombardment 
on the Western Front.''

``Heavy,'' sneered an old-timer. ``Heavy. Call this heavy? You wait a while.''

Guy was sure Robbie was not smiling in the dark at that savage and unmerited rebuke. But 
that rebuke stifled the pretensions of new reinforcements. It was easier to conceal ignorance 
by saying nothing.

Shuffling along cobbles, they arrived somewhere. Guy tumbled into deep straw and slept. 
Welcomed and serenaded by the song of the guns, that song was not yet too intimate to 
overcome tiredness. Journey had been long, cobbles wearying, and understanding was only 
indirectly pregnant with the terror endured by other soldiers who crouched under the 
Royal Hail fired from Birthday guns.

At Fleux Baix they were merged into a brigade, and reinforcement comradeships of a year's 
duration were dissolved.

``From here to the right, right turn,'' said an R.S.M. with a list.

``Good-bye, Robbie, I'll miss the smile.''

``Good-bye, Jack. We'll meet again.''

They were to meet once again while Robbie's smile was taking wings and leaving his body forever.

``Quick March!'' said the Sergeant.

They were moved away to become permanently merged in a battalion. Overhead in the blue sky a throbbing 
plane was being driven higher by a spitting gun. Miles to the front a German sausage balloon was 
hanging in the heavens observing British movement.

``We're at the war,'' asserted a voice.

A shell screamed over into an abandoned nunnery and sent up a spurt of red brick dust.

``Like spray against rocks,'' he answered.

Angry patches of smoke continued to pock the heavens around the German plane.

``We're at the war,'' another voice agreed.

Down the road came a wheeled stretcher containing a sprawled figure that dripped red blood on 
the dry road.

``War's right,'' said the fourth voice.

``Fleux Baix is fun,'' the Sergeant informed his new quartet.

Everyone came and went unconcernedly. In Fleux Baix traffic was heavy in broad daylight. 
If every house had been spattered with shrapnel the sector was still accounted quiet, and 
jostling life hurried openly on its lethal mission. Shells were dropping in the nunnery 
because the nunnery had a tower. Shattered shops, churches, the nunnery, all were tumbling 
down but men were untroubled.

``Seems lively enough,'' Guy hazarded to the Sergeant.

``Very quiet sector.''

``Why do the Germans hammer hell out of the churches?'' said a man at the back.

``Hell shouldn't be in the churches,'' he was answered facetiously.

``But why?''

``Steeples are good observation points.''

``The British guns are quiet, Sarge.''

``Tired after the birthday celebrations.''

The platoon he joined was billeted in a two-storied house badly damaged by shell-fire. 
He wondered why men inhabited targets. He was to learn that in some sectors a policy of 
live and let live was adopted by German and British artillery, a satisfaction with a 
steady toll rather than a reaching toward the maximum harvest. No one bothered about 
the frailty of the house in which they lived, and the village crumbled as much from the 
insidious under-mining of its inhabitants as from the hammer-strokes of high-explosive. 
Shell destroyed occasionally, soldiers operated incessantly.

Soldiers desolated. Soldiers wanted wood. Boards, beams, doors, anything that would burn 
was torn down. Shell hammered down. Men gathered and burnt up. The snow on the fields was 
frozen and soldiers were humans who loved to doze in the smoky drugging warmth of a brazier.

When a section of soldiers descended upon a deserted house destruction commenced. First, 
then, the furniture. A chair would feed the flames, a table would follow. An axe would tear 
polished mahogany off a grand piano. Piece by piece, leg by leg, the instrument would 
disappear. Beds, couches, wardrobes, dressers, all yielded a moment's warmth. When the next 
section arrived there was no furniture, so down came a mantelpiece, a cupboard door, the 
shelves of a pantry. The human family had crouched by a fire since it had crawled out of 
the slime, and out of a battlefield slime soldiers came to crouch by glowing piano leg or 
cupboard door. Plank by plank the lining and ceiling went to feed the flames. Then a beam 
here, a beam there, the stairs, so that the second story was reached by an improvised ladder. 
Lastly, plank by plank, the floor, until the billet was uninhabitable, until the unsupported 
walls were ready to fall before air concussion alone. Thus it was that ceilings, beams, whole 
houses, humble deal tables, and aristocratic grand pianos, beds that had weathered the love 
and birth of centuries, were fed to the flames. Warmth was warmth, and the soldier would be 
dead a long time. Property had lost its sanctity but not its power as a fuel.

In his billet after nightfall, John Guy sat beside a fire that glowed in an oil-drum. The 
freezing air beyond was carefully excluded from the billet, and the smoke of the wood drugged 
his lungs and caused his eyes to weep. But he was warm as a purring cat by a hearthside. His 
ears were attuned for all that was novel.

``The heat causes 'em to bite,'' said a grimy fellow, scratching his buttock.

``You can't smoke the little bastards out.''

``No, you can't.''

Over the brazier flame two men toasted kippers they had purchased. The smoke and the aroma 
grew thick.

``Cooking or curing?'' a voice in the corner asked.

``Need a gas-mask to cook 'em.''

``Need a Victoria Cross to eat 'em.''

Voices from grimy countenances fell on Guy's ears. Sentences, silly and sensible, were 
uttered to the room, to the brazier, to any listening ear, words were fired into the 
smoke like shell fired at random, words that might or might not find a billet, words 
uttered because it was pleasant to exercise vocal chords rather than to remain glum, 
a stream of sounds that could be as meaningless as bird chatter, as meaningful bespeaking 
life. Smoke, space, war, devoured sound greedily, the sane and the inane were swallowed in 
void. No one marvelled that all talked and few listened.

``We want more wood.''

A rough, unshaven giant adjusted a scarf and vanished, to return in a few minutes with a 
closet seat riddled with shrapnel. He returned wearing the wooden seat around his neck like 
a giant ruffle. His face beamed with delight at his own joke, and everyone roared gustily. 
How pleased was the giant to bring laughter as well as warmth. He removed the seat and 
chopped at it with a bayonet, throwing the pieces alongside the brazier. Crude wit was 
unleashed. Guy listened delighted. What good fellows these were.

``That seat was made to fit your head.''

``What this piece of wood has seen!'' The Giant sighed in envy as he chopped away, happy 
at his fortune in spreading happiness.

Laughter and blunt allusion dominated the room. Everyone attempted an obscenity. The closet 
seat had the virtue of keeping frozen air at bay and of invoking a mental atmosphere that 
banished thought of war. A plank would have been a piece of wood. A closet seat had a droll 
personality. Guy laughed with his unshaven comrades. ``We,'' he thought, for already he 
wanted to flatter himself by considering himself one of them, ``are children.'' And as 
they made play with that piece of wood they were the sort of children that in other days 
had raided orchards and played around bays. Their profanity was naked of culture, but 
nakedness was unashamed. Cannon fodder was taking life easily. Life was too unsubstantial 
for sobriety. Apart from laughter Guy hesitated to express himself. He wanted to know his friends 
before he ventured. Sufficient for the moment was his joy at being incorporated among them. 
Good fellows they were.

He heard them talk of sex and he heard them talk of food, the two great topics after war 
itself. Of sex, for they were young and life was fleeting. Of food, for they were lean and 
rations were slender. The unshaven, brown, grimy faces reflected comradeship when the brazier 
flickered. Kippers, the brothel, the closet seat, war itself managed to get discussed. He 
gazed upon a red ember and saw in its glow the fair hair of the New Zealand girl he loved, 
and imagination rioted until the dying ember put an end to crystal-gazing. Only living embers 
are populated with radiant memories. Maybe he realized that John Guy wanted all the sensations 
for which his comrades lusted, that he wanted them now before impending casualty, that all 
he loved was so far away as to be beyond reach almost for ever. For he had appetites too, 
earthy appetites to deny or feed on husks and substitutes. But he shuddered back from the 
thought of the queue.

War, he thought, as he lay down to think and not to sleep, is the human family going back 
to primeval slime. War is reversion. Chaos out of cosmos. Shells tear down cities house 
by house, soldiers tear down houses plank by plank. War gathers the world's living to 
sprinkle Flanders with the world's dead. Men leave home to stand in a brothel queue and 
disease is multiplied. People go hungry to feed guns, and the prodigious maw of cannon is 
yet unsated. Individuality among growing things is surrendered for discipline in the mud. 
Order is the new beauty, even if lice inhabit its singlet. Atrocity is the new patriotism.

Flesh and its works were falling down. Mantelpieces, libraries, craftsmanship, construction, 
were smashed by shells, burned in braziers. The world was ransacked to create chaos. Wool 
from Australia, butter from New Zealand, steel from America, wheat from Canada and Argentina, 
timber from the Baltic, commodities from all parts of the world were being pulped in the 
mill. Mars was the God of Chaos, but deluded his worshippers by giving orderliness to 
disintegrative processes. The neutral nations that did not fight for territory or high-sounding 
principles, or greed or stupidity, fought one another commercially for war profits. Carrion-like 
they battened on victim flesh. Warring nations prostituted themselves before Mars, and the 
neutrals, like procurers, fattened on the proceeds. Ships came from the ends of earth bringing 
the young and the healthy to become torn and gangrened. And at last he was at the centre of 
things, and his mind was playing verbal truant.

The embers were dead and the lights were out, but conversation came from the reclining men.

``Fleux Baix is quiet.'' A member of the new draft was speaking.

``Yes, quiet. But three men were gassed in the billet next door last night.''

A third voice said something from a muffled face.

``Too quiet. After the Lord Mayor's Show comes the sanitary cart.''

``Dear Mother,'' went on another in the language of the communique, ``it's a peaceful night 
on the Western Front.''

Conversation and desultory phrase-making at last ceased. Guy's mind stopped philosophizing. He 
fell asleep.

From lying asleep he passed spasmodically to wakeful terror. ``What's that?'' his heart asked. 
``What's that?'' his voice feared to ask, for his comrades slept on. He could hear their 
measured breathing. Was he the only startled man in the room, the only one afraid of a something 
that had awakened him and which he couldn't place? Cautiously he raised his head, but in the 
darkness all forms seemed recumbent, unconcerned. But his heart was beating wildly. From 
outside came the shuffling sound of the feet of the gas guard. Nothing seemed to have happened 
and yet he had awakened afraid. He started to settle back on the floor, but as he lowered himself 
a something caused him to arrest his movements. Half-reclining, terror left him as understanding 
came, although he was still excited by the intensity of his waking. He heard a long-drawn sigh, a 
sigh that seemed to wobble, a sigh prolonged with menace, yet passionless and weary, a sigh 
indifferent as to anxieties of listeners. An age after the sigh had passed away he heard a deep, 
far-off explosion.

A man got up to go outside and relieve himself, muttering aloud to himself, to any listener.

``Fritz is putting them a long way over to-night.''

``Putting what over?'' Guy asked.

``Long---distance-shells.'' The answer came as though the words were wobbling a long way back too.

``You fire your words like the shell sighs.''

``All long-distance shells sigh and wobble at the end of the road.''

``Shut up, for God's sake. Shells are bad enough without cackle.''

Someone was trying to sleep.

``Have one for me,'' said another voice. ``It's too cold to go outside.''

John Guy's nerves were stimulated by novelty. He followed the other man out to the can, shivering 
with the cold. It seemed that the moonlight was playing a magic melody with the snow and heaps of 
tumbled bricks. ``Shirt-tail magic,'' he thought and smiled into the night.

``Cold.'' The gas guard stood close to the can, relieving himself of garrulity only.

``Cold it is.''

Guy returned to his blankets again. He slept until another splitting shock rocked the billet. And 
again silence reigned except for the beating of his heart, and again he could not say ``What's that?'' 
for fear of laughter. He knew that understanding would soothe, that the terror which beset him was 
merely of the unknown. He did not know that the silence of other men was not the silence of 
indifference but of helplessness.

Crash! Crash! Crash! went a salvo, and the house rocked once more.

``Blast the battery. Why can't they do their dirty work in the day-time?'' There was disgust in the 
voice.

``Yes,'' a voice equally annoyed said, ``we want an eight-hour day and a half-holiday.''

``War is going to the bloody dogs.''

Guy knew that the crashing of a battery and the wailing of long-distance shells would never worry him 
again. Only novelty had startled, although all noise would trouble him until he learned to distinguish 
between far-away noise and dangerous noise. Each crash caused a slight start, but in between salvoes 
old hands snored. And artillery soon grew tired of tearing moonlight to tatters, and shelling ceased. 
And then he lay awake, alert, listening to deep breathing, to the far-off trench chatter of a machine-gun.
For all the frozen snow outside he was warm, and could still sniff the odour of kippers. His mind had 
the wonderful alertness that comes at times when the body is passive but sleep has been interrupted. 
Off it went at its old game again.

``Everyone except me is asleep,'' he thought, happy at that as though wide open eyes could have pried into 
his active mind. They have lost all care. War is as far away as sleep is from consciousness, an incredible 
distance, for mind, knowing only wakefulness, cannot comprehend the vastness the spirit travels on the 
road to sleep. And yet that vastness can be bridged by the reverberation of a gun. In his mood groups of words came tumbling on to his tongue. Sleep and wakefulness were as far apart as were life and death, as were black 
and white, and yet none of these were distinct but all blended. A few minutes before a gun had terrorized him and now the same atmosphere soothed. In that alert wakefulness that is yet drowsy, words and ideas assembled in his mind like the legions of a drunken army.

Somewhere men were being killed, murdered, dismembered. Somewhere on clean moonlit snow there was death, ache and agony; somewhere guns played death sonatas. Slaughter and deep sleep, moonlight and machine-guns, snow and 
sharp bayonets, of these formed a drowsy and poetic compound. Men fought to eat, to love, to survive, against Germans who fought for the same reasons. Was it pleasant to thrust a bayonet into a body and see it emerge wet and shiny, dripping red drops on white snow? Drowsiness made killing beautiful until the crashing of the battery brought sanity again.

``Damn!'' murmured a voice as a figure turned on its back. ``Damn!''

But that final shot sent Guy's castle of verbs crashing, and pillowing his cheek in a grimy palm, he fell away once more.

For a few days the company remained in billets. Guy got to know his comrades. He found that comradeship was of greater importance in fireside talk than possession of university degrees. When his comrades talked they solved world problems, and if their solutions were wrong so much the better, for correction was still left for subsequent occasion. Opinions were judged as much by the quality of the man as by the quality of the man's opinion. Argument was weighed when the voice was the voice of a good soldier and comrade. The intelligence of the new chum was discounted in advance. What did a man who knew nothing of the trench know about politics, wine, women, food, of art or science or literature, of life? Exposure to shells made opinion oracular. Lack of experience repulsed the most incisive contention. The new-comer who vocally insisted on his opinion was a gas-bag. Appreciating this, John Guy thought a lot but added little to the verbal pool.

In time he acquired the contemptuous familiarity that smashed down this illogical mental wall. But he understood. He sensed that this stupid loyalty had much to commend it. Men who suffered and risked together trusted one another more than they trusted the stranger. Out of suffering there grew a common loyalty with high survival value that transcended pure reason. If a man had demonstrated steadfastness his views were entitled to respect. A good comrade spoke down from an elevation, and a voice that speaks from above has the added force of gravity. The new chum spoke from a depression. The quality of the instrument had to be ascertained before the quality of what the instrument gave forth was appreciated. Words were poor things on which to rely when men wanted comrades.

And, after all, is that not the rule the world over, the cause and the curse of every culture? Clannishness is as old and as vicious and as virtuous as the herd. Clannishness induces co-operation, helps a boot to find the chin of a prostrate foe.

``Who's down, Bill?''

``A Hun.''

``Kick him in the bloody teeth!''