\chapter*{\textsf{The brass hat who was a Mad Hatter}}
\addcontentsline{toc}{chapter}{The brass hat who was a Mad Hatter}

T\textsc{he} offensive was over. The company had advanced to possess a portion of the captured territory. Shells fell on the concrete roof, light whizz-bangs which made the occupants flinch and their ears ring. But each explosion was harmless, the light shells making only slight indentations in the reinforced concrete before scattering into fragments. The inmates were safe even when Fritz got going with armour-piercing shells, for heavy guns found great difficulty in connecting with small targets. The shells fell into the soft earth around, making the pill-box oscillate, but only spraying it with gushers of mud. Indeed the slow fuses of the armour-piercing shells frequently slid into the soft earth without detonating at all.

A shell landed in the trench outside and made a mess of the drainage, and the torrential rain backed up and flowed under the steel door into the pill-box. Everyone scrambled on to the top deck of the large pill-box and hoped that the water would not reach that height.

The inflow of water annoyed the Colonel, who had been comfortably seated in a part of the pill-box where there was no decking. He was a tall man and was disinclined to compress his six foot four into four feet of floor space, for that was the distance from the decking to the concrete roof. Shells were falling outside along the old German trench, whizz-bangs, coal-box, high explosive, and the armour-piercing beauties were sliding deep and churning around mud and harness and the remains of dead men.

``Any shovels?'' the Colonel asked, anxious to keep his feet dry and warm, ready to order out a working-party.

Fortunately there were none. The wave that had gone to the refinery had taken everything that looked like a shovel. The ranker inmates didn't mind how much water filled the Colonel's boots, preferred a cramping posture under concrete to danger and sloppiness outside. There were enough crouching out on the fire-step already, wet and filthy with soaking rain and mud, crouching to dodge the fragments that whistled in the air.

Guy sat by the phone with boots full of water but congratulating himself. If his feet were wet the roof overhead was satisfyingly thick. He had come along a heavily shelled sap to the phone and had managed to avoid even the rain. He swung his legs backward and forward in the dirty water, but above the knees he was dry, and as he contemplated the flow through the iron door, he felt sure it would not rise more than a couple of feet.

The Colonel sat at his elbow. The advance had brought the Colonel to the front the better to control his regiments, though that was laughable, for one company was in La Bass\'{e}e Ville on the right, and one company was forward on the left, and no communication saps connected them with Battalion, nor could trenches be dug any more rapidly than they filled with water. And all phone wires had been cut. For all he knew of his company's disposition the Colonel might as well have been on Jupiter, though that was not his fault. He was something of an irresponsible factor when his organization had been leap-frogged into chaos.

Outside on the fire-step, when not crouching to dodge spinning fragments, the sentries were keeping eyes to right and left fronts for SOS signals which the Colonel could relay to batteries, or if the wire was cut he could send up a rocket or two himself. Miraculously the phone to the rear had held all day and through the evening bombardment, answers coming back regularly. Until the Colonel made up his mind he wanted to use it.

``It's dead, Sir.''

``You can't call anyone up?''

``No, Sir.''

Of course wires were like that. The greater the number of shells the more wires were cut to pieces. But the Colonel was almost reproachful to the signaller. Someone hammered on the steel door.

``Open! Open!''

The door was opened and a runner came in with a surge of water that raised the level another three or four inches. The runner was filthy with brown mud and his dark eyes shone out of a weary face, one of the most weary Guy had ever seen under steel helmet.

``What's that?''

``Message from La Bass\'{e}e Ville, Sir.''

``Where? The Headquarters?''

``Beyond the refinery, Sir.''

On the heels of the runner came the Colonel's batman from the rear, from a cook-house a mile away, with a hot meal for the great man. The Colonel perused the message, the batman unpacked the container, spread the meal on a square of decking. Two hot boiled eggs, a jug of hot coffee, a parcel of buttered toast. The food was spread out on the bench before the Colonel, who toyed with it and made up his mind about the message.

``Ah!'' The Colonel distinctly grunted deep satisfaction as he swallowed a mouthful. ``Ah!'' He became aware of food rather than of the message and exhaled toast aroma.

A dozen pair of eyes shone with hungry greed as the Colonel ate. A dozen rankers held their breaths as the Colonel knocked the top off an egg. Nostrils dilated and caught appreciatively at coffee aroma. Everyone sighed as the spoon went deep into soft, hot egg-yolk. The men were as crouching camp-fire dogs ready to pounce upon and fight for thrown fragments.

John Guy found himself breathing deeply, distending his nostrils as though some of the sustenance might reach his stomach via his air passages. His excited and denied gullet repeated the swallowing process each time the Colonel swallowed in the same way that an audience clears away an imaginary huskiness when a speaker has a cold, in the way in which observers smack their lips when someone is seen eating a lemon. Well nourished, and intent upon problems of high strategy, the meal to the Colonel was but a sideline. He was unaware of the disturbance soft eggs, coffee, toast, had evoked in the tantalized gullets of his audience. The Colonel was a dainty eater despite his inches, seemed to trifle with what should have been a serious performance. He scraped the shells clean, drained the last drop from his jug as though it didn't matter. The Colonel used the food to fill his stomach. His audience would have used it as well to sustain the soul. So Guy swallowed and swallowed and, wet and tired and cold, the runner from in front eyed the Colonel almost as though he might suddenly snarl, pounce upon a fragment and make off into the night.

Nor did the Colonel realize that his every movement was being followed by a battery of shining eyes until he was pushing away his cup. Then, in a fit of appreciation and like a schoolgirl, the dainty man of six foot four blushed, and was discomfited more by those eyes than by shell, for with all his daintiness the Colonel had courage. He turned his eyes to the note from the front and read its contents a dozen times.

``Humph!'' He cleared his throat and subdued his agitation. ``You can take this message back, and since the phone is out of gear you can go with him, signaller, and learn the route for further messages.''

``Yes, Sir.''

Thunder and lightning were shaking and illuminating the air as Guy and the runner splashed out into the rain past the steel door. Shells were shrieking and howling and crashing and they both felt frail as they emerged from the concreted. A spell under concrete always unnerved for the open. They jumped the parapet and hurried away from the dangerous proximity of the pill-box and Guy was wet to the waist before they had proceeded thirty yards.

``You can work round by the bloody sap if you want your bloody head blown off,'' said the runner. ``But if you jump the bloody trench and keep slightly to the bloody right you come to a bloody building, and if you go straight ahead you will find the country bloody well firm.''

``You're pretty good on the bloody.''

``It's a rotten bloody night.''

``Greedy cow,'' said Guy almost to himself, changing the subject.

``Yes,'' answered the runner, whose mind was not far from the meal. ``Greedy bloody cow.''

``Nice job this he's given me. Run over and get to know the way to La Bass\'{e}e Ville.''

``But I came alone.''

``You learned the way in daylight.''

``I was pretty funky.''

``Those eggs were appetizing.''

``To hell with the Colonel.''

They heard a ringing smash in the rear and wondered whether a shell had fallen on the roof or broken against the door. Another shell fell out of the night and sent up a huge gusher of earth, the falling of which steadied progress.

``That's close. But the shells go deep in mud.''

``Too bloody close. If we miss that old building in the dark we'll flounder around all night. Can you bloody well swim?''

``Do you bloody everything?''

``Not a saint, are you?''

They went waist-deep into a vile shell-hole and Guy got his rifle filled and plastered with earth. Mud got inside of his clothes and caked against his skin.

``I'm a bloody sand-bag,'' he remarked.

``Why bloody everything?'' the runner parodied.

Darkness and drenching rain were barriers to progress. Fitful red flashes shone out when shells exploded, and now and then when rain grew thin flares encircled up around the front. The flares, as always when men are on the side of a salient, seemed to be going up from all around.

``Easily get lost to-night.''

``To bloody easily.''

Ground grew firmer as they moved from shelled trenches, but around the shelled ruin the morass was again waist-deep. Night and rain obscured everything and they plunged through the deepest holes instead of taking the shallows.

``Greedy cow.'' The runner again reverted to the pill-box.

And that was the last that runner ever said, for a fragment of steel from a shell that had burst a long way off sang a shrill song over fifty, sixty, seventy yards of mud and impacted moistly into his throat, nice smack that anywhere else but on the throat would have meant a treasured wound. But this sharp, spinning fragment cut an artery and life gushed out into the mud. A few pints of blood and the runner lay dead on a crater lip. Guy bent over the dead man and shook him as if to awaken him from sleep.

``I say! I say! I say!''

A second time he shook the carcase.

``I say! I say!''

And he grew very much afraid when his companion did not respond. To the terror of the moment was added the collapse of the months. It required a mental effort for him to tear himself away from the warm body. Even the dead were better company than the morass. He pushed ahead to try and find the building and he thought he would find it all right, but when he had gone a few yards he remembered the message and slopped back to grope for the corpse again. Finding the message, he lost his sense of direction completely. And when the rain ceased for a moment flares came from everywhere. He was lost. Round and round he splashed, looking for a ruin, buy he might as well have been hunting for two eggs and a plate of buttered toast. Irritation, frustration, funk, possessed him, and he cursed aloud and talked to himself.

``You silly bloody swine, getting lost.''

He paused and shivered and murmured ``Cold,'' and then sneered at that explanation. He took careful note of the circling flares, will-o'-the-wisps pregnant with illusion, and abandoned all attempt to find the ruin, the brick-tumbled island in the sea of mud. He would go straight ahead and try and find the cobbled road that ran to La Bass\'{e}e Ville. That was a broader objective. Unwittingly his feet took him to the worst of the bog. He tried to direct himself by the chug! chug! chug! of Fritz machine-guns, but like the flares that sound seemed to come from all around too. He floundered into and splashed about in a thousand craters. He hurried his progress with all of the exhausting frenzy of fear. Suppose it wasn't much of a road. Suppose it had been shelled until it wasn't plain where field ended and road began. Suppose he went over the road without being aware of it. Suppose he blundered into a German trench and got knocked on the head. All the possibilities of night haunted his steps.

He started to move forward with excessive caution, straining his every nerve, pouring out the energy of all his senses in the effort to pierce the veil ahead. And he was scared. He expressed his fears to himself aloud.

``I'm scared.''

The compassion he had for himself strengthened, refreshed him. He went on, plunging, slipping, peering, listening, intimidated by the sucking progress of his own limbs as he withdrew his boots, by the laboured noise of his own breathing. He paused to glance here, hesitated with his foot half-lifted to listen to something out there. With eyes blinded by darkness, ears overwhelmed with battlefield and imaginary noises, pores saturated with mud, he saw felt, heard, every terror France contained. Night, dark night, conferred one boon. If it created, it also hid his funk from the eyes of men. None could bear witness.

``Greedy swine!'' He paused with foot half raised to sniff inconsequently at the recollected odours of coffee and toast.

To a stranger on a road the way is always long. To John Guy, slopping about lost in a crater field, half a mile seemed an eternity. He had almost given up all hope and was ready to lie around until daybreak to get his bearings when he came out upon the cobbled road. The cobbles were pitted, but there was firm walking for a man who picked his way. But he wasn't careful. Shells were bursting and machine-gun bullets were traversing the road. He went down on his belly a number of times with bruising precipitancy. Howling shells would have kept him funking had not loneliness driven him on. A few months earlier and he had been in the habit of going to earth and arising automatically. He still flattened instinctively, buy a mental effort was necessary to erect and drive himself on. Bullets whined and cracked and spat overhead, German and British bullets, German cracking low down on the road, British high up, and whining to an unknown target.

Then what? He had a blood-curdling thrill. Once he went to earth, buy paused half-way back to the erect posture. He paused with his knuckles resting against the earth as an athlete would poise on a mark awaiting a pistol-shot. He had sensed something. He did not know what, but his heart raced wildly. He poised tensely. He turned his head carefully to the right and he turned it back to the left as if something could see his movement, as if quick movement might get itself heard by the night. He looked ahead, lifting his head stealthily,, as though the head worked on a creaking, stiff hinge. What the hell was it? He strained and listened, and his eyes watered. The howl of salvo and the pattering of mud was all he could hear. He was about to hazard a step when from behind the wall of night he heard the sound of sucking mud. And he lost his fear in his instant concentration. In a flash he was once again a grim creature for anyone to encounter on a dark night. Still as a runner on his starting-mark he poised himself and straightened. Was there some other figure poised and straining toward him? And was it in khaki or grey? Did someone stand with rifle or bayonet, with bomb or club at the ready for his approach? Danger at hand filled him with all the desperate stuff of heroic resistance, fortified the soul. And as he crouched the sound, for sound it had been, came again, increasing his tension.

``Ugh!'' Only the groan of a wounded man somewhere to the right or left or in front. Was it German or New Zealander groaning? For the wounded man might fire at him out of the night if this accent was alien, even now might be ready for the menace that had clattered along the road. Again came the sucking sound and a grunt as the wretch tried to ease his posture. He listened and thrilled. He had no pity for the wretch walled in by pain and night, not yet, until he knew whether the creature was nationally dangerous.

And then a voice spoke to the night, to itself, to the guns, to the heavens, and its accents of pain filled Guy with joy.

``Oh God damn. Oh, God damn. I think I'll die. Oh, God damn.''

``Where are you, digger?''

``Here in the mud since morning.'' What and eager voice! ``Been batty most of the time. Are you a bearer?''

``No, a runner.''

Guy floundered to the fellow's side. The man was lying with his legs doubled up under him. Guy made no effort to straighten the legs. He would leave them as they were until bearers arrived.

``Think a shell dropped something across my legs. They're broken up but not cut. You won't leave me, will you?''

``I've got to go to La Bass\'{e}e Ville.''

``Send bearers.''

``We'll get you in before daybreak.''

``You won't forget.''

As Guy hurried away the man groaned and called again.

``Don't forget.''

Once again as he went on his way he grew scared. As he went on a flare from the low side of the Lys cast up the shadow of tangled brick and steelwork. It was the refinery. He approached with care, for he was in strange country. He didn't want to stop a New Zealand bomb or bullet. He groped and knocked against a tangled piece of iron that swung loudly against another piece. He emerged, paused, crossed a street, and stood in a second ruin. He thought he heard a voice but, not sure, tested the locality with a universal syllable that might be British or German. He coughed quietly.

``Who goes there?'' the challenge spat at him.

``Runner,'' he called, omitting passwords.

``Righto. Come on.''

It took them a few moments to find each other in the blinding dark. Rain was falling again.

``You gave me a hell of a scare coughing. Where are you?''

``Here.''

They walked twenty paces up a shattered street and Guy was directed down into a cellar. There were German dead as well as alive New Zealanders in the cellar. The Captain was talking to a small assembly of sergeants.

``Lie low all day. You can't leave your places if they get on to you until you get trenches dug. If they get on to you and you try to clear out they'll chop you up with machine-guns. Make yourselves snug before Stand To. If anyone wants to urinate he can use his shell-hole. Better than get killed.''

The sergeants left and spread out to their posts. The Captain talked to Guy and gave him an up-to-the-moment statement of his strong-point positions. Guy gathered most of his message verbally. He sat and rested for about an hour, and the Captain's batman gave him a mug of hot tea they had managed to brew. As he drank it the dead Germans were dragged upstairs out of the cellar and thrown into a shell-hole. The Captain gave him final advice.

``Tell the Colonel we don't want daylight messages. Our position is directly under observation.''

``Yes, Sir.''

The Sergeant returned along the road with Guy to show him the best point of departure on the return route.

``Hurray.''

``Good luck.''

Shells and bullets came to stimulate his tired muscles, but he travelled without incident over firm earth, managing to cover the return in a few more minutes than he had been away hours. True, his too great haste sent him stumbling, falling, plastering his mud-soaked body afresh. But he reached the trench without hurt. Drumfire was intensifying for morning was nearing again.

``Who's there?''

``Me.''

An infantryman waved him direction to the pill-box.

``Fifty paces along the sap.''

There were no shells falling and he splashed up to the iron door, tugged it open and entered. And it was always meal-time in the Colonel's world, for again the Great Man sat in front of coffee and toast brought from afar. Guy's mind reverted to a visit paid in infancy to an ocean-going steamer. The bosun had told him of the marvels of the deep, of the intricacies of ship organization. ``The captain gets fifty meals a day, the first officer twenty, the second ten, the third five, the brass boy and the sailors get fed once a day. How would you like to be a brass boy?''

``Only one meal a day?''

``Yes, my lad.''

``I don't think I want to be a brass boy.''

The absurd dialogue came back as Guy faced the Colonel. Exhausted as he was he wanted almost to laugh in the Colonel's face. The eyes of the onlookers still glinted in hunger from the top deck. The Colonel, like the mad hatter, seemed always to be moving to a fresh place at the table.

``Well?'' He breathed coffee aroma in Guy's face.

Guy handed up the note while the Colonel ``ahed'' and refilled his cup. He climbed his mud-plastered carcase on to the deck while the Colonel read, sipped, meditated. The other occupants of the deck withdrew as far from his soaked body as possible. They were dry. Tired to a point at which coffee failed any more to excite his lust, he started to nod in the warm atmosphere. But the Colonel was rightly curious and prodded him and asked a hundred questions, essential questions, but asked with a carelessness of manner that exasperated a sleepy man to the extreme. At last the inquiring mind let him fall asleep. It seemed that he scarcely had closed his eyes before they were at him again. Actually he had had a fair rest.

``Here,'' the Colonel said, ``take this message across.''

``Christ,'' Guy replied, half asleep, half awake, ``I'm tired.''

``I know you are. Take a pull on this. You know where to find the company.''

Guy swallowed a generous mouthful before he spluttered at the undiluted firiness of army rum. Refreshed a little he took the message across. The road seemed only one-thousandth the length of his first journey. Greying sky and knowledge of direction gave confidence and ease. Mist stayed late and he came and went without incident apart from dodging shells and bullets that came out of the dawn. But when he returned to the pill-box he knew a weariness of muscle and spirit known to him only once before in his life, a weariness that had come to him when, untrained, he had entered for a half-mile race, and had run on to collapse. He climbed on to his bench to sleep but couldn't sleep. And somehow what kept him awake was an incomprehensible something about the Colonel. Something was missing and he couldn't tell what it was nor sleep until his mind had answered the riddle. The effort to answer the query was as exhausting as slopping through mud. And then he had it and smiled like a child and curled himself down. Revelation was vouchsafed at last. With the riddle answered he breathed deeply and fell away to rest, fell a long way.

The Colonel had been without a fresh meal of coffee and eggs and toast. That lack had troubled his tired mind. He slept until late that day.
