\chapter*{\textsf{Good-byeee}}
\addcontentsline{toc}{chapter}{Good-byeee}

S\textsc{he} was frail but her capacity for exhausting him mentally and physically was prodigious. For she could not talk to him and he couldn't ask her questions all the time. And as she exhausted him mentally and physically she bankrupted him financially. He couldn't be niggardly and the pay-office wouldn't allow him to be generous. That granite citadel would resist any second assault. Unable to be interested in her mind required of him to frequent an interest in her body. Soon he had no questions to ask and no more money to spend and many days of his leave untouched. He had to save enough from the wreck to feed himself on Y.M.C.A. cocoa and unpalatable biscuits. So it ended in a disillusion that was mental, physical and economic.

So ends everything in war. Love and profane love, religion and rationalism, patriotism and pacifism, realism and idealism, life itself---in satirical madhouse laughter. He sat up in bed one morning, put his arm around her, rested his unshaven cheek against her soft and powdered one, and crooned:

\begin{verse}
"Good-bye-e. Don't sigh-e.\\
Baby dear, wipe the tear from your eye-e.\\
Though it's hard to part I know,\\
I'll be tickled to death to go.\\
Don't sigh-e. Don't cry-e.\\
There's a silver lining in the sky-e.\\
Bon Soir, old thing. Cheerio, chin chin,\\
Napoo, Tooraloo, Good-byeeeee."\\
\end{verse}

She didn't argue. She was planning her day so as to get an early start after a new soldier. She regretted only the state of her purse. The procession of men had long caused her to accommodate herself to departure when companionship became unprofitable. She was too poor for sentiment. He had been a good sort, but that was all. They sang the song together as the maid brought the toast, a third time as he dressed. Light-heartedly he kissed her.

``If you come again------''

``If------''

He kissed her when he was dressed and walked down the stairs. She stood on the landing and sang the absurd verse down to him. He stood with his hand on the knob and sang it back. Three other prostitutes came to the landing and joined in the singing. The ``coarse bloody swine'' came into the passage and kept time with a nodding head.

\begin{verse}
``Good-bye-e. Don't cry-e.\\
Baby dear, wipe the beer with your tie-e.\\
Though it's hard to part I know,\\
I'll be tickled to death to go.\\
Don't sigh-e. Don't cry-e.\\
There's a silver lining in the sky-e.\\
Bon Soir, old thing. Cheerio, chin, chin,\\
Napoo, Tooraloo, Good-byeeeee.''\\
\end{verse}

They all laughed and waved and the ``coarse bloody swine'' said a ``good luck.'' Guy opened the door and stepped out. He slammed it hard as though he were closing down on an event.

Crash!

He felt the few coins in his pocket. The sun was shining. He was well, a carefree pauper. The rest of his leave could look after itself. His heart was so bright he went on his way singing:

\begin{verse}
``Bon Soir, old thing. Cheerio, chin, chin,\\
Napoo, Tooraloo, Good-byeeeee.''\\
\end{verse}

He radiated happiness. His mirth was infectious. Straight-faced strangers grinned in friendliness at him as he swaggered down the street. A tall-hatted pillar of society distinctly inclined his head.

``Good morning.''

``Good morning, young man.''

Again the absurd lines possessed him and he gave them voice:

\begin{verse}
``Bon Soir, old thing.\\
Cheerio, chin chin,\\
Napoo, Tooraloo,\\
Good-byeeeee.''\\
\end{verse}

\begin{flushright}
\textit{Other books in the NEW ZEALAND SERIES are The Children of the Poor, The Hunted, and Shining with the Shiner, all by John A. Lee. In New Zealand, inquiries to Vital Books Ltd, 15 Mount Eden Road, Auckland C.3.}
\end{flushright}