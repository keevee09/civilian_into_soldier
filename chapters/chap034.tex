\chapter*{\textsf{Counter-attack!}}
\addcontentsline{toc}{chapter}{Counter-attack!}

H\textsc{e} opened his eyes and listened, for his nerves had sensed something in the tone of a voice even if the meaning of the words had not yet become apparent. His reclining body was stiffening, his nerves were tingling with expectation, fatigue was unhanding him and energy was returning. ``COUNTER - ATTACK! COUNTER - ATTACK! COUNTER - ATTACK!'' The words went through his clearing mind. ``COUNTER-ATTACK!''

He heard the words again. Where did they come from? What did they mean? He shook his head as if to clear his mind. He filled his chest with air. He breathed the air through distended nostrils as though the words hung in the air, emitted a distinguishable odour. The voice came again but he was alert, for his own mind was taking up the refrain. It was the authoritive voice of the Sergeant, not nearly so authoritive as the voice in his own breast.

``Counter-attack!''

He jumped to his feet and searched to the front but he could see nothing. The air in front was clouded with the smoke of bursting shell. Day was merging to twilight. The sky was clouded. Nerves had conjured up the counter-attack, but of that he knew nothing. To him the words meant what they said, became not ground for sceptical investigation but for dogmatic certainty.

``Counter-attack!''

Many a voice reiterated the words, but his heart kept yelling the phrase so that it seemed to explode from within rather than clamour from without.

``Counter-attack!''

Tired he had been, but he was tired no longer. Fearful he had been and again he was eager for the fray. The stupid, indomitable hardihood of zero was again upon him. If he had twitched in helplessness before the onslaught of the intangible thunderbolt, fear of advancing men quickened him again to resist.

``Counter-attack!''

The words were a tonic, meant something human, something that could be grappled with, something of bone and flesh and blood. Shells were differently constituted. Shells had mechanized efficiency. Shells were callous despite molten fury. Shells came from an impersonal cannon and dropped down out of heaven, and neither courage nor point of bayonet could deflect them in their fall. But counter-attack had to be forced along by flesh and blood. Counter-attack spelled other breasts and other bayonets, point to point, fury to fury, blood to blood, hate to hate, khaki to grey. All this he sensed. And he knew he could win in that encounter, for he would out-beast the German beast, outrage hell itself.

``Counter-attack. Stand To.''

The silly bastards to say ``Stand To.'' He was standing to. He had been standing to ever since he had breathed the delicious odour of that phrase. He was staring into the smoky veil ahead, staring until his eyes watered. Out of the heavens two barrages were falling, curtains of reddening flame as day vanished. Like the Aurora Australis, the Southern Lights, which he had seen in the midnight air of southern New Zealand. Red shells were flashing across heaven and hammering down on earth, but he could discern no man in grey. And yet he was possessed of that phrase. His every nerve yelled at him.

``Counter-attack!''

They would advance behind the smoke, grey on the heels of barrage, humans behind steel and advancing night. All that he had known that day had fallen from him. He had forgotten his frailty. He had forgotten legs snapped like brittle carrots, men shot in the guts who clamoured for water. Fear, fatigue, resignation, all these were past dross. He had again swollen to heroic stature until his bayonet commanded the approach to Messines. He was clad in the greatest armour plate in the world, faith in his supreme beastliness.

``Counter-attack!''

That was a food which expanded a man in all ways, in length, breadth, purpose. In a world of detonation and chaos once more his body was anchored securely to an idea. Men in grey were coming. Let the bloody swine come. Let the bloody swine come. Bayonet to bayonet he would greet them. Ecstasy was his mood as he thought of his sharp point.

``Jesus. Haven't we had enough?''

He was in no mood to heed such a whimper. Enough? He laughed to himself, as does a drunken man. To hell with such thoughts. Let them come, let them come. He had grovelled and cowered and abandoned hope before the impersonal, but he would grovel before nothing of flesh and uniform. For they were only men in grey and he was an iron beast. The blond beast would crumple under the point he would drive at it. He dropped his eyes from the gathering gloom ahead out of which they must come. His bayonet was red, red with the blood of friendly gushers. Queer tricks his mind played, for in the millionth of a second he became introspective, stood outside of himself and saw John Guy standing with a bayonet on the battlefield. And then he lost sight of himself in the mesmeric beauty of his point. He would make it shiny red this time, clean New Zealand blood away with German. He saw the blade, wet, satiny, dripping. He wouldn't kill because he had to but because he wanted to. He would run bodies through before they ran him through.

Possessed by the crazy killing lust, courage induced by gun thunder and fear, he grew irritated at the delay. Why the hell didn't they come? Some men grow to such moments. Some are overwhelmed and cower. The most sensitive not infrequently are the most adaptable. He was born to grow steady when most others lost their heads. he stared and tried to will grey men out of the smoke. When they came he would not wait. Waiting for men would be like waiting for shell. Who waited cringed in spite of himself. When he sighted a grey man he would jump out of the trench and run yelling at the bastard, and while it flinched he would have his point through its guts. And he would run a second and a third blond beast through.

It never entered his head that the man in grey might be the more furious beast. Might shoot him down as he ran. He would win because the German would be more human, compassionate, afraid, because the German would think of home while he would know of nothing except the joy of the thrust. The German would be the man and he would be the beast. In a fatal moment of humanity the German would be brushed aside. He, John Guy, was worth a thousand blond beasts because in that moment life held only killing purpose. His mind raced around all the clash that was coming, imagination hypnotized the human into the great soldier, caused him to see rare beauty in bestiality. He was mad, deliriously, joyfully intoxicated with the lust for warm, wet German blood.

``COUNTER-ATTACK!''

Ah, the bayonet! What a wonderful instrument that was for a killing man. What a means of imposing mortal terror before it imposed a mortal hurt. He balanced rifle and bayonet in both hands, and decided that a rifle and bayonet were supremely beautiful, could only be more beautiful when the blade was wet red. If a dozen came he would race to kill, as a footballer goes for the touch line. His elation at thought of murder obliterated thought that a few men in grey might be swollen too, might, like himself, be scourged by despair to iron-clad resolution.

Why didn't they come? Why didn't they come and get it all over? Why must he wait? Why must he wait and stare? In his anxiety to get at them he could have cried as does a vexatious child. In irritation he asked the question aloud.

``Why don't they come?''

To his amazement, his thought brought back an answer.

``Surely you don't want them to come?''

Of course he did, but something in his neighbour's' tone checked his utterance, a doubting something, as if his neighbour had heard the unbelievable. He looked to the other side and saw another neighbour, and he seemed unhappy too. And both men were looking at him as though he were queer. And again, as earlier in the day, contact with something more human made him human. Once more his heroic hysteria ebbed and left him limp. And as he answered his words became no longer true.

``Of course I want them to come. Get it over.''

Too late. He lied. He had wanted them to come a second ago but the timid concern of a neighbour had had the force of a cold douche. And he was tired once more. Why couldn't Fritz leave them alone? They had endured enough already.

``I wonder if they will get this far.''

Counter-attack had become a menace instead of an intoxicant, for human sanity had thrown down beastly delirium. But the figures in grey did not come. German and British artillery battered because both feared. Guns fired at imaginings and not at persons. Furies did not come, although the darkening horizon caught fire with the scattered oil of incendiary shells, although SOS rockets hung brilliantly in the sky when twilight had abandoned earth to star and shell light. The yellow, red, and green eternal brilliants shot up to heaven to mingle with the eternal brilliants, and men in shell-holes acquired and forsook earthly pain. Guns thundered and earth vibrated and squirmed and shuddered. Everything vibrated, and out of the organic and the inorganic flowed a destructive essence.

Across the heavens hung two curtains of glowering steel, and from under the curtains pulpy heads peered fearfully forward, trying to catch a glimpse of oncoming men called enemy. Brittle heads had the stuff of reason and unreason battered out of them. Legs and arms and heads were torn from living and dying carcases and thrown around in the night. Organization and mechanization reverted to chaos. And once out of the awful night, like a clarion heard in a breathless lull, came the whinny of a horse, a horse whinnying from a soft muzzle, a horse chopped down somewhere nearly to death, a horse which, like a wounded man, wanted bearers. With memory only of past kindliness and forgetful of a race of men that had set it against nickel and steel, the horse whinnied for company. And that whinny caused Guy to reach automatically for his basket of pigeons. Men were such dumb suffering brutes themselves.

Stars in the sky, red glowings in the heavens and earth, reverberation until the universe seemed caught in voluntary disintegration and fell to pieces beneath human thought and feet. Wonderful, beautiful, terrible SOS rockets celebrating the high revelry of Mars, telling of hope and fear and religion and horse and man being broken in the night. SOS. SOS. Save Our Souls. Save our brittle heads. Save our fragile limbs. Save our reason. Late into the night death revelled, painting earth red with crimson essences. There was no escape from the thunder of the guns. Men had got to die, had got to die. Mars sewed a crop of flesh and ground it to paste. Men crouching in shell-holes lost human semblance mentally, lost human form physically. No one saying, ``God save the King.'' No rest for dying eyes upon ``the pinnacle of sacrifice towering like a rugged finger unto heaven.'' Howitzer shells perished in a blaze of light, veritable thunderbolts. The men who made howitzer shells were put to death like squibs.

Exaltation and fear and despair and resignation, courage and cowardice, possessed in their turn the soul of John Guy as he flinched in the night. When no tangible foe came and only shells howled through the night, ardour fled, fear fled, all fled at last except unutterable weariness. White and red shells burst. Cracking and whining bullets came. Over the sap where they had known hell, shells were falling, falling, falling. Night over there would have been vigil in hell. In the morning British guns dropped shells into Messines; in the night German guns dropped shells into Messines.

``Dig,'' said the Captain, revived with a whisky.

Wearily until exhausted flesh revolted, they strengthened and deepened trenches for counter-attack or counter-bombardment in the morning. And then they rested their carcases in postures as ungainly as those adopted by the dead.

Night sped.

``Stand To!''

Drearily in the greying dawn they grew ready to repel counter-attack. A day ago, an epoch ago, they had been launched over the top, thrice as many as would return unscathed. Now they awaited another dawn that would renew the thunder. There was no hush and sudden out-flaring of guns. All night the guns had shelled. But fury multiplied to drumfire intensity with the greying of the sky. Dawn and shell dispersed the mist that gathered around day-old corpses as the sun climbed the sky.

When no advance heralded dawn the guns steadied. Corpses started to swell and bloat in the sun. The guns started to throw shell far back, knocking out batteries. The guns started to drop shells on trenches newly dug in the night. Men grew hungry, thirsty. A thirsty soldier dragged a dead man out of a pool, a wounded man that had crawled to quench his thirst, who had drowned in a pool as big as a large basin. None hesitated to drink the water a man had drowned in.

A runner from Battalion Headquarters brought a message and a bottle of whisky for the Captain. The officer could get whisky while the rank and file thirsted for water. The order commanded the company to retire to the fringe of Messines and be ready for a possible effort to repossess the hill. The company hurried in small sections. One section fell to pieces crossing the sap. Attached to the Captain, Guy tumbled into a well-concreted cellar, while the less fortunate set to work to deepen a line of trenches cut the day before by another company. Because the door of the cellar fronted the German guns he went with the Captain's batman to the rear and dug a deep, narrow trench into which all might crouch beneath a concrete parapet.

The ``counter-attack'' of the previous evening turned out to have been provoked by the partial retreat of some Australians who had overrun their objective and were being shelled by German and British guns. Returning in the twilight a few hundred yards, far-away observers had mistaken them for Germans. They had wilted under the concentrated savagery of the two artilleries.

``The luck of the game.'' The Captain shrugged his shoulders and had a whisky.

