\chapter*{\textsf{Not so tough after all}}
\addcontentsline{toc}{chapter}{Not so tough after all}

``T\textsc{ough} men,'' said the Corporal, handing them over to the guard. 
``Tough men.''

Guy knocked on the wall.

``Says we're tough. To be honest, I'm scared stiff.''

``Me too. We were fools to get cheeky.''

``We'll get hell to-night.''

``Been getting it all day.''

Silence was broken again by the knocking of the man from Wanganui.

``Listen!''

``Yes.''

``You sang my favourite song yesterday morning.''

``What's that?''

The voice sang:

\begin{verse}
``Rule, Britannia; Britannia, rule------''
\end{verse}

The Sergeant came hurrying along and knocked on the door.

``Don't. I like the pair of you. But if you try to sing with that 
voice, I'll shoot you at dawn.''

``You're a good sport, Sarge.''

Guy's chest was bad all right. When he got out in the morning he would 
report sick. He developed a racking cough when the frost settled on 
the tin roof. On bare boards and in shirt-tail and two blankets he 
could not sleep. The tin above was a perfect frost conductor. And 
he had a great mental as well as physical weariness. Body and mind, 
over-weary from monotonous repetition, refused to rest. Sometimes he 
reached the borderland of sleep. But the leering, rat-eyed face of 
Angley always intervened with some snarling command on its bestial 
lips. The tired mind would set the body off in double time, feet 
always out of step, the turn always going the wrong way.

``Hell,'' he thought in a lucid moment, ``must be a place where body 
and spirit do it wrong automatically, despite the desire to do right.''

``Say, Guy, are you asleep?''

``Yes. Feather bed. I'm talking in my sleep, as a matter of fact.''

``I've been drilling all night.''

``Me too.''

Hell, he was tired. ``Go away, Mr. Angley,'' he said to himself. But 
the leer always floated back into nightmare. The voice always jarred 
on weary ears. Even chattering teeth seemed to say jerkily, ``Left, 
left, left right left,'' except when he listened carefully. If drill 
he must, why could he not drill to order? Why must every movement be 
wrong? The footsteps of the sentry on the veranda now and again 
helped to disperse nightmare.

``Christ, it's cold.''

``By Christ, it is.''

The Sergeant came along.

``More blankets when Angley has gone to bed. You can have a hot drink when 
we change shift.''

``I hope I meet you in France, Wanganui.''

``Me too, Guy.''

They never met.

``I hope the Tommy jail is no worse than this, Wanganui.''

``Thanks. But it is. They knock 'em down and kick 'em up. They call it 
breaking hearts. They can break my head first.''

``If I were you I would cave in from the start.''

But Wanganui was only boasting to keep up his spirits. The lion-eater 
wanted to make a brave show. He knew there were no limits to martial 
cruelty, that the only regulation was to make the knee bend, the heart 
to whimper. Punishment must be so in war. Malefactors must be crucified 
to wagon wheels, men in cells beaten up. Outlaws must cringe and come 
to heel like whipped curs. If punishment were modest, the misbehaved 
would inhabit cells while the well-disciplined died in the front line. 
Jail had to be bad enough to make the braving of bomb or bayonet an 
easier hazard. Where millions risk agony voluntarily, what is a little 
organized cruelty? War gives the soldier a slight chance, the military 
policeman must give him none. There is no room in military jails for 
Marquis of Queensberry rules.

Guy's developing chill made him less concerned about Angley. Gradually 
he was acquiring much of the irresponsibility of high temperature.

Angley had not forgotten the men in the cells, but liquor came before 
lust. Angley consumed much liquor a little cost, for sergeants passing 
through camp found it easier to treat the R.S.M. than to ignore him. 
He could tell good canteen stories. His technique had been perfected 
upon many reinforcements. He told stories of how he handled the Other 
Ranks when discipline had to be vindicated. The mess was warm, the 
reinforcement had money; across a foaming tankard Angley had it in him 
to be a good fellow. But as night advanced he was left against the 
bar with his cronies, two Provost Sergeants. And then the good fellow 
went from his face and he remembered his debtors.

``Drink up, boys. Let's go down and inspect the guard-room.''

They went. They were in the habit of going to humble men who boasted 
their contempt for authority. Guy heard Angley's voice speaking to the 
guard outside, the low warning of the man from Wanganui.

``Good luck, old man. Don't squeal.''

``Squeal?'' Guy was unafraid, a lack of fear he did not connect with 
the irresponsibility of high temperature.

In the guard-room Angley chatted boisterously to the Sergeant, who 
hated him but who was studiously polite. He uttered a hearty joke that 
brought a murmur of approval from the cell next door.

``Good-natured to-night.''

``Wait,'' muttered Guy.

``Um, we shall have a look at the prisoners.''

Footsteps came along the passage, a key turned, a bolt jarred, a door 
swung open. It was not Guy's door, but Guy felt the beating of his 
heart growing until it conquered the chattering of his teeth, grow 
until it seemed scarcely to have room to beat in the cell. He knew 
fear. Weariness dropped away and he was listening, listening.

``SHUN!'' yelled Angley.

There was a moment of silence.

``Come on. Never mind the blankets. Parade in your shirt-tail.''

A Provost Sergeant laughed.

``Yes, Sir.''

Guy fancied he could see his comrade in misery standing in the cell 
in his shirt-tail. And as he knew his own turn was coming, fear gave 
way to elation, to a spiritual intoxication the like of which he was 
to know in France again beneath the barrage. Evidently the man from 
Wanganui was shivering with cold, for Angley's voice made play of 
the fact.

``Look! Look! So brave on parade to-day. Now the cur trembles. Yellow. 
That's what he is. Catch him, you chaps.''

To Guy's ears came sound of scuffling as the prisoner struggled to 
keep the Provost Sergeants from pinioning his arms, resistance that 
gave legal title to reciprocal brutality. They had him.

``Let go! Let go! You'll twist my arms out.''

``Shut up!'' said Angley, ``or we will.''

The sound of a blow on face or body came to Guy's ear, and then a 
word.

``Apologize.''

Silence and a second blow. Grunts, but no apology.

``Apologize---apologize---apologize.''

After each smack in the face or punch Guy heard that word. He 
thrilled to the contest. He wanted to go to the rescue. To hit 
Angley in the teeth with a boot, to hit until his flesh broke, 
to knock out the beast's teeth, to hammer blood down the beast's 
face into his mouth. The man next door broke away, but he was no 
match for his opponents. One stood on his bare feet. Hitting 
recommenced.

``Apologize---apologize---apologize.''

Mighty the power of a closed fist. To hurt of impact was added 
at last the humiliation of apology.

``I apologize.''

``Sure.''

``Yes, Sir. Yes, I do, Sir.''

``All curs do.''

The man who ate lions in Wanganui was crying.

``Give him two extra blankets. We don't bear malice.''

Oh, the unrivalled magnanimity of Angley to the strong man made 
a sobbing child.

Then Guy heard his own bolt shot back, and a voice jeered at him. 
He had lived so vividly through the duel in the next cell that he 
had forgotten the cold, but distinctly he heard his own teeth chatter 
again. Elation at contest left him, and he found himself facing 
what was coming with limp weariness. He only wanted to be left 
alone. He was suddenly less afraid of Angley in the flesh than 
he had been of the nightmarish Angley who had driven away sleep. 
He did not know that he was in the slightly delirious land of high 
temperature. Weaker than his neighbour, easier apparently to 
subdue, a touch of illness had charged him with a listless 
tenacity.

``Get up you swine.''

He didn't move, but heard a voice he knew was his own.

``Aw, go away, go away.''

A boot came against his ribs, but seemed to jar rather than pain. 
Almost, it appeared, he was drugged against hurt. The blankets 
that were wound round him were grasped, and he was rolled 
unceremoniously out.

``I'll give you go away, you bastard.''

Guy stood on his feet and the air did not seem to be so cold, or 
he forgot that it was cold. Angley bent forward and laughed at 
him. Then for a moment there was silence, broken by a sob from next 
door, a more complete silence when the man next door forgot his 
hurt to listen. Angley's mouth opened and he bent toward Guy, 
whose arms were clutched very tightly by the Provost 
Sergeants.

``Good evening. Angley the dirty rat, has called around to pay 
his respects.''

Guy didn't struggle, for he was void of strength. Angley's face seemed 
to float in front of him as though supported by no body, floated, 
bloated and obscene. From the mouth came and alcoholic breath. 
Guy didn't feel the pressure on his arms, but the breath was 
real, vile, inescapable, seemed to strangle.

``Go away and let me sleep.''

``Take that, you rat.'' Angley spat in Guy's face. ``And that 
and that!''

Three times he spat while his friends laughed. A slime that 
seemed to pollute the very soul ran down Guy's face. Deep down 
in him the outlaw and vagabond prompted a daring something that 
was yet passive.

``You dirty swine, Angley.''

Angley closed his fist and struck hard. And again impact jarred 
where it should have pained. And the blow made Angley's face 
dance more obscenely than ever, made it spin like a top, move up 
and down.

``Swine,'' said Guy.

He had a sense of wrongdoing but was too weary to achieve correction. 
He didn't try to say the word. It came automatically. And its 
repetition seemed as instinctive as breathing, a fixed idea that 
was mechanically reiterated. The fist came again.

``Swine.''

Between blows he did not feel, between concussions that made the cell 
spin, the word came again and again.

``Swine! Swine! Swine! Swine!''

He was too tired to change the word. To find something else 
to say required concentration, and everything mental was in 
chaos. In the nightmarish cell his mouth kept forming the word 
long after his voice ceased to pronounce it. It had to be battered 
from his face. His arms were twisted and his body was limp and 
his face bled so that spittle and blood mixed, but he didn't care. 
Angley, afraid of the consequences of subduing such vocal 
determination, suddenly controlled his fury.

``Let him go.''

Guy slid limply to the floor.

``Better revive him.''

Angley seized a fire bucket of icy water that stood in the 
passage and tipped it over him, making the cell sloppy.

``Swine.'' Half consciously the fixed idea spurted from his lips.

``Is the Sergeant there?''

``Yes, Sir.''

``Dry him and give him all the blankets he wants.''

``Yes, Sir.''

``You heard him call me a swine and a bastard?''

``Yes, Sir.''

``You know what I could get him for that?''

``Yes, Sir.''

Guy lay on the floor and started to cry, but the tears were of 
exhaustion. He was crying as a child cries, for he was tired of 
bugle calls and drill, of Angley's face and obscene breath, of 
the cell that twisted and turned and spun. Angley withdrew. A 
voice came from the next cell.

``By Jesus, you're made of iron.''

But he didn't hear. He pillowed a head in his arm. He 
thought he was caressing a head on a pillow, head of a 
golden girl in far-off New Zealand. Outlaws have no place 
in an army. The Sergeant came with blankets and hot cocoa. 
He threw military law to hell.

``Come out by the fire, boys. Cheer up. You'll soon be 
in France.''

``Is France much worse?'' Guy heard himself saying.

``More dangerous but better fellows. Angley's the worst 
cow in the New Zealand Division. That's why he stays put.''

``How did you get your face knocked about?'' asked the M.O. in the 
morning.

``I lost the fight.'' Guy was learning not to complain or argue.

``Some of you fellows should never go near a canteen.''

There was no doubt about the temperature, Guy spent the 
Christmas in hospital.

``Full diet, boys, pudding to-day and Epsom salts to-morrow.''

Guy was eager to escape to France. If men suffered in France 
it was because the Empire was at war. If they suffered in 
England it was because militarism was the new Joss and was 
served by an odd bully. Early in January he was discharged 
from hospital. For some unknown reason Angley had been 
drafted and had become the victim of what soldiers said was 
a self-inflicted accident. Some rumoured that news of his 
beastliness had at last reached up to the O.C.'s ears. Guy 
escaped to France, marching out of camp in the dusk of 
evening. The draft travelled all night to arrive in Folkstone 
in the dreary dawn. In the afternoon the ferry took them 
to Calais.