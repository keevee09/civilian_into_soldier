\chapter*{\textsf{Up and over}}
\addcontentsline{toc}{chapter}{Up and over}

H\textsc{e} was flung violently against the side of the trench by the oscillation of the country as a dozen mines vomited German earth and men to heaven. But, like one in dementia, he felt no bruising contacts. From the rear zero came in the belch of a thousand guns.

``Up and over!''

Did he say ``Up and over!'' or did everyone? Somehow he was out on the trench top with shovel and pigeons and ammunition and life itself sitting airily on his shoulders. He was pressing forward, and comrades on both sides were pressing forward too. On they were going, and he was going with them, all were going, exultantly anxious to add the screaming and yelling of men to the screaming and yelling of shells. For soldiers, like shells, screamed and yelled before life fell asunder, if death gave them time.

On, on, on, on the heels of barrage, on the heels of hell, on with the jubilation of a surfer riding a big breaker, except that a soldier was part of the wave too; on, careless about comrades, about Joes and Bills, and only concerned about bayonets. On fiendishly, for such a molten wave is the conveyor of devils; on, on, to tear and strike and claw and kill. On to get at German throats before German barrage came down, on to catch grey men in grey whimpering in shell-holes, men scourged and terrorized in barrage and confronted immediately with a moving hedge of inhuman bayonets. On. By God, if it were only to men to have the velocity of bullet and shell, how they would get on! No one talking, everyone hastening, eyes front, never an eye for the men that were twisting and dropping as German nickel cut at the dawn.

``We'll get across before the German barrage comes,'' someone did try to yell in his ear.

Fear and horror, what were they? What the hell were they when frenzy sat in possession? He was an iron man, a giant. He was irresistible. Heaven nor hell had no terror for him. His heart beat surely. He was callous, and yet not cool. He had absorbed the temper of the furnace called Red Morning. The song of the guns that spit German soldiers into German corpses had tuned his soul to their great harmony. Never had life had for him so great a purpose. To get on. What urge in all experience equalled that? On, on. He fell over a piece of tangled wire into a shell-hole, crashing heavily. He was up and away, half conscious only of his fall. On, for he was free from all hurt and all ache, he was drunken with the killing lust, as merciless and as efficient as the barrage.

Since leaving the trench he had sloughed off a million years of social repression. He had freed the beast and atavism came shambling out of humanity. To kill and to wound. That was a purpose. To be part of mechanically incubated chaos, to be red in tooth and claw, except that primeval claws were shod with steel and nickel. To kick at German faces with feet steel shod. Once he had been afraid of death because life had promised fruitful continuity, but now life or death were paltry beside the urge to advance. To get on, on, on, to chase the whirlwind with a pointed bayonet.

``It's daylight.'' Somewhere a voice was yelling that in his ear.

``Yes, daylight.''

He was irritated at having to assent. Anything which got in the way of his urge was a nuisance. For night had gone but he didn't care. Night had been blown away from the muzzles of guns. There was grey smoke and grey mist on the air. The guns that had blown the foolish things of life, like love and compassion and laughter and philosophy and wives and brothels, out of the way, had smashed night to smithereens. Grey mist and acrid vapour were commingling.

A first wave from the trench that had been fifty paces in advance of their own had captured and killed what the barrage and mines had left in the German front line. Guy saw a companion put a bullet into a hot dead carcase. His reaction was only irritation. Why couldn't the bloody fool keep his bullet for a living body? Why waste anything that had power to smash a Hunnish skull? Up the hill they swept. Up the Messines Hill that started to stand out from behind its shell-tattered shroud of fog. At the first objective men from another battalion were digging in. Advance was retarded a moment, for men were ahead of schedule and had to wait until lifting barrage sent them on again. Every foot of earth had been whipped and churned.

In a shell-hole he saw Robbie, whom he had not seen for months, dying. And the beautiful smile was still on the face if the heart knew doubt. He saw but did not understand what he saw until quieter moments. His eyes registered the impression, but his mind postponed thought about it. He only awaited the lifting barrage to get on. Friendship was a meaningless thing with red in his eyes.

And, miraculously, Germans had survived to resist the onslaught, men had weathered the churning barrage, the creeping barrage, to face bayonet points as they lifted their heads. And although he knew it not, the few who had survived to man the nickel-spitting machine-guns had taken heavy toll of his comrades. He knew it not, for he had no eyes for the rear, not even as he crouched in waiting. The German positions had been lightly held in expectation of the attack, for defence had been one of depth rather than stubborn denial. German High Commands were prepared to let the door swing and then bloodily come back. But there were dozens of German dead lying around, crumpled like sacks of soft, unresisting material. Blood still dripped from hot dead bodies. Guy crouched beside one whose face had been ploughed from chin to forehead by a bullet, saw another whose eyes gazed in mournful horror at his spilling intestines. But Guy saw all these things incidentally, unpityingly, scarcely understood them as yet. They were impressions that were to be delayed in their effect.

And he was off and away again, chasing the creeping barrage, on, on, on. He saw a New Zealander stoop to rat a Hun who had lost his backside, and he saw the back straighten and the face smile as the ear tried to hear a watch ticking amid the gun thunder, schoolboy joy at the live watch, unconcern for the dead Hun. ``It lives.'' Did the lips of the forager try to make that phrase penetrate hellish din, more anxious about the life of the souvenir than about his own.

On, on, on, trampling on the heels of high explosive. Pressing up the hill he laughed aloud, yes, aloud. Down a spur on the left a party of Germans were running, escorted by a private. They had their hands aloft, and he knew what the bastards were yelling.

\emph{``Kamerad! Kamerad!''}

Everyone was laughing. How ludicrous grown men, husbands, fathers, seem running with arms aloft, pleading, afraid of red-eyed capturers. For advancing men are not dove-like. Childishly funny was the mortal fear of humans. Laughter came spontaneously, hysterically.

\emph{``Kamerad! Kamerad!''}

The prisoners ran to keep ahead of the escort who ran behind. Guy saw a neighbour drop on a knee and aim at a prisoner as though aiming at wildfowl, and one of the prisoners dropped his arms and went rolling head over heels down the spar. The rifle cracked a second time and another rolled over.

``Stop, you bloody fool!'' the escort was yelling, but none could hear him as he passed on. He didn't want to be shot down by a British bullet.

Couldn't do a thing like that, Guy thought. And yet it didn't seem to matter much. Certainly he felt no inclination towards remonstrance. On, on, he was going. Shooting a prisoner wasn't criminal at the moment, seemed merely a breach of good taste. What were a few Germans more or less? What the hell was anyone?

In a few months Guy might see more dispassionately if he lived that long. For the moment his eyes were red. He knew homicidal mania. He had patriotic delusions.

On and on, and the bloody barrage crept along, too slowly for vengeful men. The shooter of the prisoners let elation get the better of discretion and he hurried on and overstepped the barrage. Suddenly he became part of the chaos and fell to fragments. Something split his body before the very eyes of following men and hurled the pieces up and around and down to the churning earth. A German, a solitary, jumped out of a shell-hole, a twitching German without a scratch, a living miracle, and elevated his palms. Surrender was inevitable.

\emph{``Kamerad!''}

But it was too late. The twitching apparition had no power to take red out of on-running eyes. A bayonet in the guts brought his hands down to its sharp edge while the mouth still stuttered \emph{``Kamerad!''} A few seconds earlier and there would have been time for reason. He would have lived. He fell or was shaken off the wet, crimson blade and the killer went through his pockets. Half a dozen men mildly rebuked the man with the red bayonet, but they had not known what it was to have a twitching Hun start suddenly upward from their very feet. And what the hell did it matter? What did matter was whether the bastard had a ticking watch.

On, on, on, and Guy ran over a spreadeagled German fatigue party, dead but warm, a party washed out by the creeping barrage as it struggled toward the German front line with sealed cans of coffee for the defenders. ``There's many a slip `twixt the cup and the lip.'' He kicked the top off an iron dixie that had weathered the shrapnel and had a swig. The liquor was palatable, refreshing. Others paused to swig and away they want again, on and on. Concentration on the barrage made them unaware of crackling machine-guns that thinned the advancing men, that kept toppling man after man to earth. The deadly litter was left behind as they swept along.

On the right a New Zealand company was working through the tumbled heap of masonry that had been Messines, running from concrete emplacement to concrete emplacement, shooting bayoneting, bombing, being shot, being bayoneted, being bombed. Men with red in their eyes were running at men with fear in their eyes, the aggressor imbued with the destructive surge of the barrage, the defender weak under its bloody hammering. For men on the heels of barrage are fiends and men attacked with hysterical frenzy, men defended out of despair. Men in grey defended, and shot and shot until they were overwhelmed, and then wondering why surrender at the bayonet-point had not been accepted. When men resist and kill so that their excited breath fans victors' faces, the fiends who have advanced despite toppling comrades cannot turn in an instant to apostles of meekness. The instant it took a finger to press a trigger, an arm to jab a bayonet, a hand to release the clutch of a bomb, was not enough to metamorphize the beast into the dove, to bridge the abyss between the primeval reversion and the twentieth-century human. For war is war and war is atrocity, and only the splash of blood can wash blood from the vision. In the ruins men squirted nickelled death out of barrels. What else could they do? To even run away they had to retire through the British barrage. In the ruins New Zealanders battered them, thrust them down to death.

Going past Messines Guy had a rational moment. He realized that with his signalling impedimenta he should be alongside his Captain, from whom he had been outdistanced by his impetuosity. He turned and saw the Captain fifty paces to the rear, to the left. When the advancing line dropped down to await another lifting of the barrage, he turned and moved toward his allotted place.

``Where are you going, Guy?'' the voice yelled in his ear as he paused. It was the Corporal who, many hours before, had interpreted his anti-religious views as apologia of a coward and the Corporal who had told him he would be yellow.

``Going to join the Captain.''

``No, you don't. No, you don't. Stay where you are.'' The Corporal fingered his revolver.

``The bloody bigoted foll thinks I'm afraid,'' Guy thought. ``Thinks I'm running away.''

``Oh, very wall.'' Guy dropped into a shell-hole, happy to be relieved of all responsibility.

There was nothing he wanted to do more than chase the barrage. the bombardment was breaking on the next objective in rare fury, breaking like the frothing sea against an iron-bound coast, except that the explosive froth did in seconds what sea does in centuries. What the hell did it matter what the Corporal thought? The right to go ahead freed him from a curb. And little incidents like that vanished at once from mind, could only be thought about in the calm of weeks afterward. The Corporal didn't irritate. What did irritate was the fretful waiting at a barrage that would not lift. He wanted to move after the steel curtain and see what was left for his sharp point. He never gave the foolish Corporal another thought.

It dawned upon him that the fog was all gone, and that the sun was shining, and that he was sweating and panting as though he had run a hundred yards at high speed. That already morning was growing late, that there was still much to do. So far his own line had not come to line had not come life and death grips with the Huns. Again as they waited an impetuous one forged ahead prematurely and was broken to bits for his too willing patriotism.

The Corporal lit a series of ground-flares to show the exact position of the line to the aeroplane overhead which circled dangerously low to be sure of the line's movement. As the flares burnt green the plane called its message to the guns with blasts on a Klaxon horn. ``Look!'' so many voices yelled that something was heard.

All looked for an instant. One of the thousands of shells passing overhead, a German shell of a British shell, had clipped a wing off the plane. There was no visible explosion, insufficient resistance in a trumpery wing to detonate the shell, which probably went on its way. The plane spun in a helpless corkscrew and the pilot fell out of the cockpit, plunging head first to earth, terrific dive to oblivion. Ge reached earth moments before the plane came down.

``Some dive,'' someone yelled in Guy's ear.

``Some don't,'' he grimly muttered.

And the barrage lifted and he was away again, forward, forward, forward on the churning earth, breathing acrid smoke, ready with his point to clean up defenders before they could recover from detonation, before they could regain assurance and resist bloodily. Toward him, staggering out of the barrage, came a German who had no wall to his belly, and the bloody foll was trying with his hands to keep his inside from bulging and spilling, futile effort, for it spilled nevertheless before he fell down on his face to die. the barrage crept along, earth leaping to each piece of pattering steel, as the sea does to raindrops. On he went with two companions, a ling way in front, dangerously near the moving barrage.

There were more German dead about and more New Zealanders were being let, but neither shot nor shell could hit him. Of that he was sure. Nickel and steel had no chance. The bullet wasn't cast to get him that morning. He knew it. Why did he know it? Is it not enough that he did know it? That others knew it until the pellet made liars of intuition. He had invulnerability. The only way his blind, stupid optimism could be knocked out of him was for a chunk of steel to let it out. He saw the head of a German with a shrapnel-hole in it and grinned. For his head was not like that. No piece of steel could hammer a hole in his skull.

On, on, on. His bayonet was pointed at the corporate breast of all Germany, and Germany would wince and bend before his resolution. On and on, with the velocity of his purpose and the weight of his equipage hurling him along, driven by a mind and heart that no longer reasoned, on he went. Emotional power nerved his muscles to prodigies. With all the irresistible weight of a locomotive sliding down iron rails, on he went. The German line, or the barrage, had gravitational pull. Bullets, pieces of steel, clods of earth, fell around his ears, but menace added to determination whereas it generally brought fear. On, superbly on, extravagantly on, movement that was the breath of death was the necessity of his life. On. Militarism had forged him for this hour. On. The anti-militarist was of a sudden forged in hell to the great soldier.