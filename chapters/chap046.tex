\chapter*{\textsf{Rest camp}}
\addcontentsline{toc}{chapter}{Rest camp}

``S\textsc{train} and worn out,'' the doctor said when he reported sick.

The medal on his chest was a passport to careful consideration. Thus equipped he appeared before the M.O. without the initial disadvantage of being thought to be a malingerer. And he was ill, physically as well as nervously. For some reason his temperature had fallen considerably, and he was overcome by a mental and physical inertia. He had been sitting with the thermometer in his moth, wondering whether he would register a high temperature, and he had been astonished to find that it was the other way round. Instead of knocking the top out of the thermometer, the bottom had fallen out of himself.

``It's low,'' he said in amazement, as the orderly read the figure to the doctor. He had sufficient experience of hospitals to know a good deal about temperatures.

``Couple of weeks' rest and good food will fix you up.''

Guy went to a rest station and was ordered a varied diet supplemented by a few pints of stout each day.

``Fatten you for the killing,'' the Rest Camp M.O. said.

Since he had never consumed stout in his life, and since he only cared for alcohol in the tobacco-thickened social atmosphere of an \emph{estiminet} throbbing with conversation, he gave his stout to other men who liked the stout for itself. Nor did he want to get well too rapidly. There were many thirsty souls who loved a drop of Guinness so much that they would have imbibed even if the liquor had brought their temperatures to bubbling point.

``You're lucky to get on a stout ration.''

``I don't care for it. I don't want it.''

``What!'' It was as if he had denied the existence of daylight.

``That's how I feel about it.''

``Give it away, maybe.''

He became a popular person whose goodwill was to be cultivated. The rest station was the rest station for an Army Corps, so he jostled shoulders with men speaking all the dialects of Great Britain, metaphorically, that is, since for several days he was kept in bed with a hot bottle at his feet.

In the marquee there were skilled malingerers, resourceful fellows who could develop high temperatures, who could feign aches and pains and hold the trenches at bay with astonishing success. Some were clever, bright fellows, anti-social to the degree that they had violent objection to mass suicide. Some were cunning while seeming dull and ox-like. These skilled practitioners of the lead-swingers' craft came to his bedside, as they went to every bedside, to diagnose carefully honest sickness so as to add to their reserve of ailments. They were unintentionally good militarists, and kept an ailment in the front line, one in support, and a dozen in reserve.

When Guy's temperature was taken the method and the result created sensational envy among the lead-swingers. The thermometer was placed under his tongue and the mercury failed to lift to near normal blood heat. The instrument was then placed under his arm with a similar result. The M.O. came.

``Don't think that's an accurate reading.''

``How will we get this one, Sir?'' the orderly inquired.

``Place the thermometer in his rectum.''

The reliable reading confirmed the previous reading. The M.O. showed concern, and hot-water bottles were brought.

The M.O.'s concern and the novel method of procuring a temperature-reading excited the skill and ambition of a dozen malingerers, who set out to produce an abnormally low temperature. At last that novelty was attained in some mysterious manner by a proud trier who was put to bed with extra blankets, hot-water bottles, better food, and a couple of pints of Guinness. The malingerer beamed and held high court surrounded by a group of less skilful but ambitious mortals.

``Chicken broth.''

``Ah, chum, how did you do it?''

``Got water-bottles?''

Envious murmurs.

``Two bottles of stout a day.''

Unstinted recognition of a commendable performance.

``And when they want to take my temperature they have to insert the thermometer in my rectum.''

Profound sensation.

Thus was honest illness shamed into speedy recovery, thus was the rest station made safe for the lead-swinger.

Too soon Guy permitted himself to be pronounced fit and dragged back to the Division. He did not realize that the malingerer caused the M.O. to discount everyone's complaint in advance, and that an honest statement would let him down. To make a good claim to illness he must needs have engaged in superlatives; positives were lame. So he went back to his Division, while a malingerer held court because the malingerer had learned to compel the freakish taking of his temperature. And Guy was pleased to get back, as most men were pleased to get back from rest camps. A wound was an honourable release entitling one to rejoice and attempt to delay recovery. But honest illness stifled among the unmanly subterfuges of the few rotters. Guy did some unmanly things, but they were involuntary acts, the acts of nervous and physical disability. He drew himself aside to sob for himself. He didn't know why. The tears indeed seemed to ease an inward pressure, had about them a perverted pleasure. This state he concealed, believing it to be exceptional. If he had known enough about nervous breakdown to know the tears as the usual reaction of his state, he would have explained his case to the Battalion doctor and steps might have been taken for a very long spell. But his ignorance of nervous trouble, his hatred of admitting that he wept like a baby without being conscious of the weeping's cause, drove him to concealment, and nearer and nearer to the edge of the complete breakdown.

Ever since Messines he had been haunted by a growing fear of life. In billets away from the line he could shake off the twitching fear of shells, although not the nervous consequences of that fear. But he could not shake off the fear of ceasing to be a man in the eyes of others. And he was also afraid for his reason, afraid he might become an unclean slobbering idiot. This fear exhausted his nerves. He had lost his fear of death, that had gone long ago, indeed death reached out hands that promised peace. He had become afraid of life with half a mind, with half a face, or with half a body. He would rather die than be botched. Fear of bullets was nothing. Fear that he might run from the danger of bullets was everything. Death that smacked at a soft chest with a crushing fragment, that was nothing either, but a life might be spent beating itself out in slobbering futility upon the walls of a padded cell. Yes, death promised peace. But other men died, companion after companion, and he remained upon the torturing rack.

So he cried because the tears welled up involuntarily, because he was afraid of his continued existence. Fear of death was fitful, fleeting, the big guns could scourge him to an elevation beyond that, sheer tiredness could compel indifference, but the fear of continuing life and the knowledge of his growing inward chaos was persistent. Fear of life made him rest his forehead against the caressing, cooling silkiness of the muzzle of his rifle. Fear of life caused him to meditate upon the slight trigger pressure that would end him and his fears. Fear of life made suicide alluring.

Someone told him that suicides were court martialed.

``Surely not! Why?''

``To condemn them and to discourage others.''

He didn't know whether to believe it or not. But the dead man would have the laugh on the Brass Hats at the grisly post-mortem. There grew up in his mind the determination to die in the manner of his own choice. Why not? Why wait until war left him without his faculties? Why? Suicide was beautiful. Suicide opened the door to mental peace. He cried over his plight. He would suicide.

For a time further thought about the problem was postponed. For a short time, for the duration of London Leave.
