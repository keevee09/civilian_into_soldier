\chapter*{\textsf{Fatigue and fire-step adaption}}
\addcontentsline{toc}{chapter}{Fatigue and fire-step adaption}

A\textsc{t} the end of a week Guy was lousy and had lost his sense of clean superiority. Like every new chum he was shocked to find that his carcase harboured vermin. That lice were a matter of determinism and not of free will. Like every superior new-comer he was unusually lousy before he admitted the affliction. He shrugged his shoulders and twitched his skin in effort to keep the parasites on the move, concealing his state from his comrades, for he was disgusted with himself. And then in a moment of wisdom he joined in open warfare, going down the seams with a match and candle. He only felt degraded while he retained a keen sense of the value of cleanness. When dirt became a mental as well as a physical habit, lice no longer mattered. Sanitation was not possible for an infantryman. He could reach at his chest and pick a louse off. Even clean clothes from the bath-house contained a supply of readily hatched eggs in the seams. Sometimes Guy's thumbs would grow red as he cracked the plump vermin.

In the trenches he settled down. Adaption was a matter of imagination, and the most imaginative were the soonest at ease, although as like as not the first to become nervous casualties. He was soon pitying those whose nerves were never soothed but always on the jump.

Night was different, for then ennui never possessed him. On many a night around Fleux Baix he stood with head and shoulders above the parapet and he was never bored. The frozen no-man's-land might always contain menace. His rifle lying on the parapet and a nest of Mill's bombs to hand, he was always too diligent to be bored. A quiet watch in the day-time dragged. A night watch never dragged, although it seemed to have power to exhaust his nervous forces. Day-time in quiet sectors was monotony, night-time always had possibilities. The walls of night that shut out desolate no-man's-land destroyed no-man's-land desolation. For only at night did no-man's-land become some-man's-land. Night watch sped because he always strained to pierce the sinister purpose that might lurk a few yards away. Night watch in support or billets was interminable, for then attention was only bent to passage of time. But on the fire-step, with a huddled comrade at foot ready to be kicked into wakefulness, night hastened away.

On the midnight parapet he learned to know the value or a mouthful of cold, sugarless tea. The tannin and the iron from the dixie restored the exhausted watcher, so that weariness fell away as fatigue does before a glass of champagne. Hands grasped very eagerly for the dark liquor. Standing when all was hushed, when no star or star-shell could be seen, when no machine-gun chugged or chattered or spat, for every bullet was toned by the gun it left or by the journey it pursued, standing on the fire-step in that uncertain world he would spend his vitality like a profligate. He could feel black, silky night absorbing his nerve force, drawing the essence out of him and into herself. Night like a prostitute absorbed his vitality, left him limp, gave little in return. His eyes strained to penetrate the impenetrable, his ears listened to catch the unhearable, the alien foot-fall or the creak of wires, tried to affix meaning to noises that came from anywhere, from nowhere other than lying imagination. For he was a good time in Flanders before he saw an enemy, and then he only caught a glimpse of field grey running across a gap in a trench. But was not the fact of so many thousands being so near and yet unseen the cause of black night always being heavily pregnant with danger?

Sight of foeman was not necessary to advertise the presence of danger. Already some of his reinforcement had died. Thousands died who never saw enemies, struck down by pieces of shell, bombs, whiffs of gas, whining bullets. An enemy seen was an enemy located, guarded against. An enemy heard but not seen was an enemy anywhere, hundreds of yards away on patrol, standing close parapet ready to respond to raiding signal, an enemy awaiting barrage protection, a figure somewhere in the night with poised bomb. Silence after noise, nervous tension, made the steel bayonet that lay negligently on the trench top seem a friend; long, slender, beautiful in its power to kill. Silence and the beating of his heart after unexplained noise could cause him to lovingly caress the icy corrugations of a Mills bomb.

Before he had come to France he had doubted whether he could face killing with equanimity, but one night he had grown aware that if need be he could kill with exhilaration. He had realized that if a figure or figures came at him out of the night, he would kick a comrade awake and jump forward with his clean, long bayonet. With an exultation that would be born of fear itself he knew he would drive a bayonet at a wincing opponent and man to man he would win. He was sure he would win because he was sure he could rid himself of compassion. He would become a beast and his opponent would remain all too human, and he would spit the carcase in front as though it were a bag of straw. Oh, yes. He had it all worked out imaginatively. He would cut through clothes and flesh and ribs. Fear would drive him to victory, the same fear that would cause his antagonist to flinch. Kill or be killed was the rule of the game. Very well, he would kill. Toying with his mental and word pictures, he could get viciously excited. His mind dramatized such conflict so frequently that when it came he had already rehearsed his part. When danger had him by the throat imagination swelled him to bestial proportions. Fortunately for him.

Night sped as he strained to distinguish between foreign foot-fall and his own heart-beat. His eyes tried to tear the veil from no-man's land, tried to see only desolation. Always he felt that his precious nerve energy, the vital fluid of life itself, was flowing outward through his eyes to a devouring glutton. Black night was a vampirish hypnotist that fixed him in mute spell and sucked away his energy. Was something moving out there? What intention did Fritz conceal behind such silence? Why was he not interrupting blackness with flickering silver flares? Why were no nickel bullets traversing the trench top? Why were none pattering into the entanglement in front? Was no-man's-land filled with patrols or working-parties or raiders? It mattered not that each dawn exposed the country in front as the abode of rusting wire and tree-stumps and bleached skeletons. At night all knew that the area was a land of stealthy men, of quiet working-parties, of shadowy patrols, indeed the trysting place of murder, where death could reach out of a haphazard thunderbolt, or out of darkness, causing men to grapple with subdued fury. Death could occur in front of his eyes as he stared, could stifle even a groan while he stared. Unending was the battle for ghoulish supremacy.

Staring into the night, his eyes might dim with moisture at the strain. Crack! would go a German flare pistol. Circling up and flickering down would go the beautiful illuminant, covering shell-hole and tree-stump with silver radiance, making a land of horror a land of enchantment, a fairyland of loveliness, as light shone on stump and corpse and bully-beef tin. But behind the man who fired the flare crouched other men with fingers on triggers, alert to drill discovered marauders, eager to do death in fairyland.

How tense was silent dark night! How packed with tautness! Tension at incredible absence of noise, as well as at too much noise. Tension because a million guns had lulled their fury. When the chug! chug! spit! spat! whine! whine! of machine-guns had paused, when no lonely shell sighed weary travail through high tunnel as it pursued its passionless mission of destruction, night was crammed with tension. For night might be stilled with its finger on the trigger, death might be crouching at the ready, and in an instant, and paradoxically, heavens might rain hell. One touch and men might crumple and fall as he tried to read its intention.

And when the world grew uncannily still and lonely, what solace it was to feel for the comrade that sat or slept at one's feet. To thrust out a foot cautiously, to feel regular breathing, to know a connection with all that lived when everything that killed was about. That touch with the foot was sufficient. To hear old Bill coughing his way along from post to post, maintaining the contact of all groups, that was a human rather than a military something also.

There were millions of lives in the British and German trenches that fraternized in a way denied of humans; the millions of rats that kept the battlefield clean. No body or piece of unburied flesh rotted. Teeth and appetite came to the harvest from German and British line. Sharp teeth tore dainty flesh, left skeletons white as snow, discerning teeth that scurried and fought for choice portions. Many a wooden cross stood erect over a skeleton and paybook. The rats could not devour the identity disc. Men were rat and lice fodder. One night, standing on the fire-step peering into the night, trying to penetrate its purpose, something came that sent a chill down his spine, that saturated body with fear, that set his heart battering against restraining rib and flesh.

``What's that?''

In his excitement he whispered at himself, a faint whisper that came faint from his lips as the sound of rustling blades of grass.

``What's that?''

Something, something, something, something incredibly stealthy. Something too impalpable to be grasped definitely, yet something discerned by all senses, a strange unaccountable threat somewhere. His heart beat hammerstrokes as it tried to break free of encircling ribs. Was something crouching with a bomb? Was something aiming with a rifle? What was it? What was it? What was it? It materialized. It was a rustle, a scuffle, a squeal, an exit across a parapet as a rat went out to dine. Many times this happened and every time he was startled, although he never again knew the terror of that first scuffle. For the listening man fell into a wide-awake trance. Wide open of eye, he fell at last into tense immobility. Concentrated on the beyond, he repeatedly forgot the parapet rodents and the shock of stealthy sound came and made huge inroads in nerve force. Each second as he listened had been an age, a huge vastness in which the beating of the heart had been mile-posts. He nearly fell at the anti-climax.

``Jesus. A bloody rat!''

But everyone fell for the unseen rat on still, dark nights. For the noise might be from far, from a crawling body, might be a brushing against a taut wire, might come from the breathing of an enemy. Nerves thrilled, the skin bristled, when the trench tormented its inhabitants with a scurrying rodent.

``Ever been scared by a rat, Bill?'' he asked the Sergeant.

``Who the hell hasn't? First time I nearly died of funk. Don't call me Bill, call me Sergeant.''

But everyone called him Bill. His goodfellowship was greater than his power as a disciplinarian.

The first night Guy stood on the fire-step he fired his rifle at a tree-stump. The morning after he had a filthy rifle to clean. He was a practical soldier.

``I'll never fire again until I put a bullet at a man. I'm not going to clean my rifle for fun.''

He kept to his resolution for the duration of his life.

He learned to spare his comrades. Some men stood on the fire-step and plugged the night full of holes. These men saw a Hun in every shadow and blazed away. They constantly aroused sleepy comrades, compelling the resting man to rub sleep from his eyes and stare into the night for a movement. Thus did strong men lose rest to carry the burden of weaklings.

``I say.'' Tense whisper.

``What?''

``Wake up.''

``I can see something out by the wire.''

He grew to curse the man who saw things that weren't there to see. Or rather the man who saw such things aloud. For that man set his nerves tingling and robbed him of precious sleep. There might be reason in the expressed fears, so others had to join in the watch and confirm or confound. To be asleep and to be jarred awake by the timid accelerated heart beating. And if a wide-awake man saw menace in shadow, it was difficult for a newly awakened man to see aught else. In time knowledge of unreliability discounted timid alarmists, but this, too, constituted danger. A time might come when the timid saw clearly. To see peril in shadow, to peer until the eyes watered, to watch for movement when the timid one fired, all this spelled wear and tear.

For four nights Guy was alone in a bay with a man who shot away hundreds of rounds at his shadowy fears. Guy never fired a shot himself, but each crack of his comrade's rifle robbed him of nerve force, helped shoot his indifference to pieces.

``Give's another mate, Bill.''

``I know. The only way to kill what he sees is to knock the seer on the head.''

Stand Down and Stand To had as yet no special significance for him. They only told of the passage of hours. In Fleux Baix they were not moments at which grey dawn or twilight erupted into red hell. The sector was quiet. Death took steady toll only. A regular percentage. Vigilance was unremitting but violence was moderate. Awfulness only ushered in the grey dawn when Brass Hats were in gambling mood and the big plunge was being planned for the spring.
